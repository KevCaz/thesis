Au royaume des idées, nous transformons les fragments de notre existence
en songes. Chaque conversation est une caisse de résonance offerte à nos
propres songes. Chaque lecture est une conversation, un échange avec
d'autres, qui bien qu'absents, nous livrent leurs pensées. Comment alors
ne pas croire que tout le travail ici présenté n'est autre qu'une
histoire, la narration de ma rencontre et des échanges avec tous ces
autres.

L'aventure du doctorat est bien plus collective qu'elle n'y parait.\\
Longue est la liste de ceux qui ont apporté une pierre à l'édifice
décrit dans les pages qui suivent. Des pierres les plus solides dont
j'ai fait mes fondations aux plus petits cailloux qui donnent un rien de
charme et d'originalité à l'ensemble, nombreux sont les contributeurs au
présent travail. Tous ces autres qui parfois ignorent ce qu'ils m'ont
offert, je souhaite qu'ils en soient chaleureusement remerciés. Bien
entendu, certaines rencontres laissent des empreintes plus profondes et
je me dois de les nommer.

Pour m'avoir laissé ma chance et donné une confiance sans faille, je
veux remercier vivement Dom, Nico et David, sans qui l'aventure n'aurait
pas été la même. Pour faire ces premiers pas sur un chemin incertain,
mieux vaut partir avec ceux pensent qu'on en trouvera la fin.

Un chemin qui m'amena sur deux continents, dans deux pays, a travaillé
dans deux universités au sein de deux laboratoires. Le laboratoire, ce
n'est pas seulement un lieu, ce sont aussi des hommes et des femmes,
autant de sources de réflexion et d'inspiration qui jamais ne tarissent.
Pour de précieux échanges à Rimouski, je remercie Hedou, Idalflex,
Jacquette, Jo Brasco, Pippin le Solver, Tim. Pour de passionnantes
conversations à Montpellier, merci Alain, Andreï, Claire, Clara,
Florian, Isa, Marie, Pierre, Simon. Solarik, les \emph{behind the
scenes} que nous avons abondamment échangés m'ont été d'une précieuse
aide dans les derniers moments.

Dans les couloirs des laboratoires, il y a aussi des liens forts qui se
nouent autour d'intérêts convergents et qui se transforment en une
amitié très appréciée. Pour tout ce qu'ils m'ont donné, je remercie
Sonia (et Thomas), Albouy (et Séverine et Léo et Louis) et Legagneux (et
Aurélie et Margot et Juliette et Zélie et Romane). Dans ces même
couloirs, on croise parfois ceux qui se frottent des yeux fatigués par
la lumière des écrans. À mes geek préférés, ceux qui connaissent l'appel
du clavier, quand l'envie de coder devient trop forte, je dis merci.
Merci Casajus, Team Zissou, Dave et Flor pour les morceaux de code et
bien plus: votre enthousiasme et votre insatiable curiosité.

L'aventure n'est, bien heureusement, nullement monochromatique. Le rose
de certains instants cède régulièrement la place à un gris parfois bien
sombre. Partagés ces questionnements, ces colères, ou encore sa
tristesse avec d'autres voyageurs est un soutien des plus précieux.
Marion, Matoche, Clem, pour ce soutient qui a été si important, je vous
remercie mille fois.

Pour ne pas faiblir face aux péripéties, le lieu de repos soit être
choisi avec grand soin. Le temps du doctorat il est important de trouver
son havre de paix. Pour avoir été les co-locataires de cet endroit
merveilleux qu'a été la maison des courges, un immense merci à Camille,
Élo, Gigi, Jean, Jerem, Jerem, Lau, Léo, Marie, Palardy, TriTri.
Palette, bien sur, un remerciement tout particulier pour toi et pour
tout ce que nous partageons.

Certains alliés étaient là avant même que l'aventure ne commence et ont
été des repères essentiels dans un voyages quelques fois déboussolant.
Ces étoiles du sol qui nous guide quand la nuit vient de tomber. A mes
amis du TerTer de Nanterre, qui me rappellent où j'ai grandi, les joies
de mon parcours en banlieue, à ceux que j'ai connu il y a parfois plus
de vingt ans, à vous Ariane, Bibi, Cendrars, Gomar, Gronico, K-wik,
Matos, Miste, Rufo, Tinico, Gomar, je vous dit un immense cimer! Un
spécial cimer pour toi Rhum, mon frérot n'a jamais cessé de me
surprendre.

Il y a ma famille, ceux dont je partage parfois une bonne partie de mon
patrimoine génétique, mais souvent tellement plus. Un grand et
chaleureux merci à Jean Louis, Josette, Jean Claude, Yvette, Tatoche,
Tonton François, c'est toujours un immense bonheur de vous voir.

Papi, Mami, éternels soutiens, des grands-parents comme vous tous les
petits enfants en rêve\ldots{} Je vous le promets, regardez, je fais de
la science, j'essaye d'être docteur, c'est forcément vrai. L'aventure
sans vous aurait été bien plus pénible.

Papa, chercheur et papa, c'est quand même pas mal! Ton soutien au court
de ce long voyage m'a été d'une aide au combien précieuse. Tu as été un
éclaireur génial vers qui j'ai pu me tourner quand l'horizon semblait
encore bien loin.

ClaCla et Nico le haricot, merci pour votre générosité, votre folie, et
pour le bonheur que vous respirez, je vous dois pas mal de sourire au
moments les plus bas. Merci Pépette pour ta spontanéité et ta fraicheur,
je sais que j'étais absent ces derniers temps, mais j'ai souvent pensé à
toi!

Maman, la plus belle des mamans. Ton fils n'est pas souvent passé à la
maison ces dernières années, et oui, l'aventure m'a éloignée de mes
terres d'origine. Rassures-toi, ton fils ne t'oubles pas, il sait très
bien tout ce que tu as fait pour lui jusqu'à aujourd'hui et il t'en est
éternellement reconnaissant.
