Au royaume des idées, nous transformons les fragments de notre existence
en songes. Chaque conversation est une caisse de résonance offerte à nos
propres songes. Chaque lecture est une conversation, un échange avec
d'autres qui bien qu'étant absents, nous livrent leurs pensées. Comment
alors ne pas croire que tout le travail ici présenté n'est autre que le
récit de ma rencontre et des échanges avec tous ces autres? Un récit qui
à son tour deviendra une nouvelle source de songes pour des lecteurs que
je ne connaîtrai peut-être jamais.

L'aventure du doctorat est bien plus collective que ce qu'elle ne semble
être. Longue est la liste de ceux qui apportèrent une pierre à l'édifice
ici présenté. Des pierres les plus solides pour bâtir les fondations aux
plus petits cailloux qui donnent un rien de charme et d'originalité à
l'ensemble, nombreux sont les contributeurs au présent travail. À tous
ces autres qui parfois ignorent ce qu'ils m'ont offert, j'adresse un
chaleureux merci. Certaines rencontres laissent des empreintes plus
profondes et méritent que je leur destine des remerciements plus
personnels.

Pour m'avoir donné ma chance et offert une confiance sans faille, je
veux remercier vivement Dom, Nico et David, sans qui l'aventure n'aurait
pas été la même. Pour faire ces premiers pas sur un chemin incertain,
mieux vaut partir avec ceux qui pensent qu'on en trouvera la fin.

Un chemin qui m'amena sur deux continents, dans deux pays, à travailler
dans deux universités au sein de deux laboratoires. Le laboratoire, ce
n'est pas seulement un lieu, ce sont aussi des personnes, autant de
sources de réflexion et d'inspiration qui jamais ne tarissent. Pour de
précieux échanges à Rimouski, je remercie Hedou, Idalflex, Jacquette, Jo
Brasco, Pippin le Solver, Tim. Solarik, les \emph{behind the scenes} que
nous avons abondamment échangés m'ont été d'une précieuse aide dans les
derniers moments. Pour de passionnantes conversations à Montpellier, un
grand merci Alain, Andreï, Claire, Clara, Florian, Isa, Marie, Pierre et
Simon.

Dans les couloirs des laboratoires, il y a aussi des liens qui se nouent
autour d'intérêts convergents et qui se transforment en une amitié très
appréciée. Pour tout ce qu'ils m'ont apporté, je remercie Sonia (et
Thomas), Albouy (et Séverine et Léo et Louis) et le joyeux Legagneux (et
Aurélie et Margot et Juliette et Zélie et Romane). Dans ces mêmes
couloirs, j'en ai croisé se frottant des yeux fatigués par la lumière
des écrans. Pour mes geek préférés, ceux qui connaissent comme moi
l'appel du clavier, quand l'envie de coder devient trop forte, je désire
taper un \texttt{\textbackslash{}\textbackslash{}huge\{merci\}}. Merci
Casajus, Team Zissou, Dave et Flaul pour les morceaux de code et bien
plus: votre enthousiasme et votre insatiable curiosité.

D'un pays à l'autre, d'un laboratoire à l'autre, d'un couloir à l'autre,
d'un projet à l'autre, l'aventure n'est, bien heureusement, nullement
monochromatique et le rose de certains instants cède régulièrement la
place à un gris parfois bien sombre. Partagés ses questionnements, ses
colères, ou encore sa tristesse avec d'autres voyageurs est un soutien
plus que précieux. Marion, Matoche, Clem, pour ce soutien qui a été si
important, je vous remercie mille fois.

Pour ne pas faiblir face aux péripéties, le lieu de repos soit être
choisi avec soin. Le temps du doctorat, il est important de trouver son
havre de paix. Pour avoir été les co-locataires de cet endroit
merveilleux qu'a été la Maison des Courges, un immense merci à Camille,
Élo, Gigi, Jean, Jerem, Jerem, Lau, Léo, Marie, Palardy, TriTri.
Palette, bien sur, un remerciement tout particulier pour toi et pour
tout ce que nous partageons.

Certains alliés étaient là avant même que l'aventure ne commence et
constituèrent des repères essentiels dans un voyage quelques fois
extrêmement déboussolant. À ces étoiles du sol qui nous guident quand la
nuit vient de tomber, j'adresse un grand merci.

Ainsi, à mes amis du TerTer de Nanterre, qui me rappellent où j'ai
grandi, les joies de mon parcours en banlieue parisienne, à ceux que
j'ai connu il y a parfois plus de vingt ans, à vous Ariane, Bibi,
Cendrars, Gomar, Gronico, K-wik, Matos, Miste, Rufo, Tinico, je vous dit
un immense cimer! Un spécial cimer pour toi Rhum, mon frérot, qui n'a
jamais cessé de me surprendre.

À ma famille, ceux dont je partage parfois une partie de mon patrimoine
génétique, mais souvent tellement plus. Un profond et chaleureux merci à
Jean-Louis, Josette, Monique, Jean-Claude, Yvette, Tatoche, Tonton
François, c'est toujours un immense bonheur de vous voir.

Papi, Mami, éternels soutiens, guides indispensables de mes premier pas
à aujourd'hui, merci. Des grands-parents comme vous tous les petits
enfants en rêve\ldots{} Je vous le promets, regardez, je fais de la
science, je vais être docteur, c'est forcément vrai. Sachez que
l'aventure sans vous aurait été bien plus pénible.

Papa, père et pair, c'est quand même pas mal! Ton appui au court de ce
long voyage m'a été d'une aide au combien précieuse. Tu as été un
éclaireur génial vers qui j'ai pu me tourner quand l'horizon semblait
encore bien loin, merci papa.

ClaCla et Nico le haricot, merci pour votre générosité, votre folie et
pour le bonheur que vous respirez, je vous dois pas mal de sourire dans
les moments où j'étais au plus bas. Merci Pépette pour ta spontanéité et
ta fraicheur, je sais que j'étais absent ces derniers temps, il faudra
qu'on se rattrape!

Maman, la plus belle des mamans. Ton fils n'est pas souvent passé par la
maison ces dernières années, et oui, l'aventure m'a éloigné de mes
terres d'origine. Rassures-toi, ton fils ne t'oublie pas, il sait très
bien tout ce que tu as fait pour lui jusqu'à aujourd'hui et il t'en est
éternellement reconnaissant.
