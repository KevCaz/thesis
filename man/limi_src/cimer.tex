Au royaume des idées, chaque rencontre est la promesse d'un échange.
Chaque conversation est une caisse de raisonnance offerte à nos propres
songes. Chaque pensée nait de la recontre de notre propre histoire avec
les mots des autres . Chaque lecture est un monologue que d'autres nous
ont laissé. Comment alors, ne pas penser que tout le travail présenté
n'est pas ma propre naration des longs échanges avec tous ces autres qui
m'ont donné par l'écrit ou par l'oral. L'aventure du doctorat est bien
plus collective qu'elle n'y parait.\\
De fait, je ne puis faire la liste exahaustive des gens qui ont apporté
leur pierre à cet édifice. Des pierres imposantes et solides qui servent
de fondations au plus petit cailloux qui donne un rien de charme,
beaucoup sont ceux qui m'on apporté durant ces quatres dernières années,
qu'ils en soient tous chaleureusement remerciés.

Certaines rencontres sont plus mémorables, plus marquantes et je veux
les remerciées personnellement. Pour m'avoir laisser ma chance et donner
une confiance dans faille, je veux remercier chaleureusement Dom, Nico
et David, sans qui l'aventure n'aurait pas été la même.

Une aventure bien particulière car elle a été sur deux continents, dans
deux universités, dans deux laboratoires. Un laboratoire, c'est aussi
des collègues qui ont considérablement alimenté mes reflexions, ainsi je
remercie vivement Hedou, Idalflex, Jacquette, Jo Brasco, Pippin le
solver, Tim que j'ai connu à Rimouski et aussi Claire, Clara, Isabelle,
Flo, Alain, Pierre, Marie, Simon, Andreï que j'ai recontré à
Montpellier. Solarik, les \emph{behind the scenes} que nous avons
abondemment échangés m'ont été d'une précieuse aide dans les derniers
moments.

Dans les couloirs des laboratoires, il y a aussi des liens plus étroits
qui se forgent autour d'intérêts convergents, pour nos échanges aussi
passionant qu'aggréable je Sonia (et Thomas), Albouy (et Séverine et Léo
et Louis) et Legagneux (et Aurélie et Margaux et Juliette et Zéli et
Romane). Dans les couloirs on parfois ceux qui se frottent leurs yeux
fatigué par la lunière des écrans\ldots{}. A mes geek préférés, ceux qui
connaissent comme moi l'appel du clavier, quand l'envie de coder est
trop forte. Casajus et Team Zissou, j'ai été bien heureux de partager
code et autres astuces. Flo et Dave, vous êtes officielement entrés dans
cette joyeuse catégorie.

L'aventure n'est, bien heureusement, nullement monochromatique. Le rose
de certains instants cède régulièrement à un gris parfois bien sombre.
Partagés ces questionnenements, ces colères, ou sa tristesse avec des
voyageurs sur le même chemin est un soutien des plus précieux, Marion,
Matoche, Clem pour tout ce que vous m'avez apportés, je vous remercie
mille fois.

Pour nous lancer dans nos péripéties, il faut établir un havre de paix
pour se reposer dans lequel nous souflons entre deux combats. Le temps
d'un doctorat il est important d'avoir une famille locale pour faire de
chaque retour au camp un bonehur quotidie, pour ce havre, un immense
merci à Camille, TriTri, Gigi, Élo, Léo, Lau, Jean, Jerem, Marie.
Palette, biensur, un remerciement tout particulier pour toi et pour tout
ce que nous partageons.

Certains soutiens sont toujours là pour moi, autant de repères dans un
voyages quelques fois déboussolant. A mes amis du TerTer de Nanterre,
qui me rappellent où j'ai gradi, les joies que j'ai connu en banlieu
avec ceux que je connais il y a parfois plus de vingt ans, Ariane,
Cendrars, K-wik, Miste, Bibi, Rufo, Gronico, Tinico, Gomar, Ariane,
Matos, un grand cimer! Un spécial remerciement pour toi Rhum, mon
frérot, que je connais depuis tant d'annés déjà \ldots{}

Il y a ma famille, ceux dont je partage parfois une bonne partie de mon
patrimoine génétique, mais ils sont tellemnt plus. Merci Jean Louis,
Josette, Jean Claude, Yvette, Tatoche, Tonton François, c'est toujours
une immense joie de vous voir.

Papi, mami, éternel soutien, des grands-parents comme vous tous les
petits enfants en rêve\ldots{} Je vous le promets, regardez, je fais de
la science, j'essaye d'être docteur, c'est forcément vrai.

Papa, chercheur et papa, c'est quand même pas mal, ton soutien au court
de ce long voyage m'a été d'une aide au combien précieuse\ldots{}

ClaCla et Nico le haricot, merci pour votre générosité, la fraicheur que
vous apporté, le bonheur que vous respirez, à l'avenir. Elisa, j'sais
bien que j'ai que le grand frère n'a pas toujours été là, mais bone tu
sais que je pense bien à toi même si je suis pas là, on s'rattrapera,
promis!

Maman, la plus belle des mamans. Le fiston il est pas souvent à la
maison, et oui\ldots{} c'est ainsi. Le fiston il sait pourtnat tout ce
que tu as fait pour lui et il t'en ai infiniment reconnaissant.
