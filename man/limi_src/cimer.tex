Au royaume des idées, chaque songe est un examen de notre propre
expérience. Chaque conversation est une caisse de résonance offerte à
nos propres songes. Chaque lecture est une conversation dans laquelle
d'autres nous livrent leurs pensées. Comment ne pas croire que tout le
travail ici présenté n'est autre que ma propre narration des tous ces
échanges avec tous ces autres.

L'aventure du doctorat est bien plus collective qu'elle n'y parait.\\
Longue est la liste de ceux qui ont apporté une pierre à l'édifice qui
tient dans les pages qui suivent. Des pierres les plus solides dont j'ai
fait mes fondations aux plus petits cailloux qui donnent un peu rien de
charme et d'originalité à l'ensemble, nombreux sont les contributeurs au
présent travail. Tous ces autres qui parfois ignorent ce qu'ils m'ont
offert, qu'ils en soient tous chaleureusement remerciés. Bien entendu,
certaines rencontres laissent des empreintes plus profondes et je me
dois de le signaler.

Pour m'avoir laisser ma chance et donner une confiance dans faille, je
veux remercier chaleureusement Dom, Nico et David, sans qui l'aventure
n'aurait pas été la même. Pour faire ces premiers pas sur un chemin
incertain, mieux vaut partir avec ceux qui y croient.

Un chemin qui m'amena sur deux continents, dans deux pays, a travaillé
dans deux universités au sein de deux laboratoires. Le laboratoire, ce
n'est pas seulement un lieu, ce sont aussi des autres, des sources de
réflexion et d'inspiration qui jamais ne tarissent. Pour de précieux
échanges à Rimouski, je remercie Hedou, Idalflex, Jacquette, Jo Brasco,
Pippin le Solver, Tim. Pour de passionnantes conversations je remercie
Alain, Andreï, Claire, Clara, Florian, Isa, Marie, Pierre, Simon, que
j'ai rencontré à Montpellier. Solarik, les \emph{behind the scenes} que
nous avons abondamment échangés m'ont été d'une précieuse aide dans les
derniers moments.

Dans les couloirs des laboratoires, il y a aussi des liens forts qui se
nouent autour d'intérêts convergents, pour nos échanges aussi
passionnants qu'agréables, je remercie Sonia (et Thomas), Albouy (et
Séverine et Léo et Louis) et Legagneux (et Aurélie et Margot et Juliette
et Zélie et Romane). Dans ces même couloirs on croise parfois ceux qui
se frottent des yeux fatigués par la lumière des écrans. À mes geek
préférés, ceux qui connaissent l'appel du clavier, quand l'envie de
coder est trop forte, je dis merci. Merci Casajus et Team Zissou pour
les morceaux de code et bien plus, merci Flo et Dave pour votre
enthousiasmes et votre curiosité.

L'aventure n'est, bien heureusement, nullement monochromatique. Le rose
de certains instants cède régulièrement la place à un gris parfois bien
sombre. Partagés ces questionnements, ces colères, ou encore sa
tristesse avec d'autres voyageurs est un soutien des plus précieux.
Marion, Matoche, Clem, pour tout ce que vous m'avez apportés, je vous
remercie mille fois.

Pour ne pas faiblir face aux péripéties, le lieu de repos soit être
choisi avec grand soin. Le temps du doctorat il est important de trouver
son havre de paix. Pour avoir été les co-locataires de cet endroit
paisible, un immense merci à Camille, Élo, Gigi, Jean, Jerem, Jerem,
Lau, Léo, Marie, Palardy, TriTri. Palette, bien sur, un remerciement
tout particulier pour toi et pour tout ce que nous partageons.

Certains alliés étaient là avant même que l'aventure ne commence et sont
restés, toujours là pour moi, autant de repères dans un voyages quelques
fois déboussolant. A mes amis du TerTer de Nanterre, qui me rappellent
où j'ai grandi et les joies que j'ai connu dans ma banlieue, à ceux que
j'ai connu, il y a parfois plus de vingt ans, Ariane, Bibi, Cendrars,
Gomar, Gronico, K-wik, Matos, Miste, Rufo, Tinico, Gomar, un grand
cimer! Un spécial cimer pour toi Rhum, mon frérot, que je connais depuis
tant d'années déjà et qui n,'a jamais cessé de me surprendre.

Il y a ma famille, ceux dont je partage parfois une bonne partie de mon
patrimoine génétique, mais souvent tellement plus. Un grand en
chaleureux merci à Jean Louis, Josette, Jean Claude, Yvette, Tatoche,
Tonton François, c'est toujours une immense joie de vous voir.

Papi, Mami, éternels soutiens, des grands-parents comme vous tous les
petits enfants en rêve\ldots{} Je vous le promets, regardez, je fais de
la science, j'essaye d'être docteur, c'est forcément vrai. L'aventure
sans vous aurait été bien plus pénible.

Papa, chercheur et papa, c'est quand même pas mal! Ton soutien au court
de ce long voyage m'a été d'une aide au combien précieuse, je ne peux
m'empêcher que si je suis là c'est que j'ai été franchement influencé,
mais tu l'as fait discretement, à ta manière.

ClaCla et Nico le haricot, merci pour votre générosité, la fraicheur que
vous apporté, le bonheur que vous respirez, à l'avenir. Elisa, j'sais
bien que j'ai que le grand frère n'a pas toujours été là, mais bone tu
sais que je pense bien à toi même si je suis pas là, on s'rattrapera,
promis!

Maman, la plus belle des mamans. Le fiston il est pas souvent à la
maison, et oui\ldots{} c'est ainsi. Le fiston il sait pourtnat tout ce
que tu as fait pour lui et il t'en ai infiniment reconnaissant.
