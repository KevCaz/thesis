Biogeography is concerned with the study of mechanisms shaping the
geographic repartition of species. Although the list of mechanisms is
clearly identified, biogeographers still search for a consistent theory
to structure their interplay. Among the challenge frequently pinpointed
is the integration of ecological interactions in species distribution
models. On the one hand, the literature points out evidences underlining
the controlling role of interaction on local communities structure. On
the other hand, relatively few studies unravel imprints of interaction
in species distribution data. Developing a clear explanation for this
apparent vanishing remains a major challenge biogeography needs to take
up. The main issue regarding the lack of a clear answer regarding the
role of interactions at broad spatial scales is that most of scenarios
of biodiversity changes assume that interactions can be neglected. If
this hypothesis is often proved false, then the scenarios must be
reviewed and sustain the development of methodologies including
relationship among species.

The thesis herein presented is a brainwork on the role of interaction on
species distribution. I start with a theoretical investigation on the
topic as classical theories push ecological interactions into the
background. Hence, in chapter \ref{chap1}, I present the integration of
interaction into a a theoretical model of species distribution coming
from one of the most important theory in biogeography: the theory of
island biogeography. This work shows how the joint effect of biotic and
abiotic factors affect the expectations derived from the classical
theory. Based on this first chapter, in chapter \ref{chap2}, I show how
interaction can affect co-occurrence data. Such data contains the
presence or absence of several species for a similar set of sites
dispersed along a large geographic gradient. Using a probabilistic
model, I obtain theoretical results linking co-occurrence data and the
information included in ecological networks. I clearly demonstrate that
interactions shape co-occurrence data. I further show that the higher
the number of links between two species the harder the detection of
interactions. Similarly if a species experience many interactions, it is
then challenging to detect any sign of interactions in co-occurrence
data.

In chapter \ref{chap3}, I print the analysis of five datasets of
co-occurrence for which the description of interactions was available.
This analysis confirms the hypotheses made in chapter \ref{chap2}. I
show that interacting species co-occur differently from non-interacting
one. My results also point out the abundance of interaction must
preclude their detection in co-occurrence data. However, when accounting
for abiotic similarities among sites, signals of interactions are
weakened. Therefore, my results suggest that using abiotic factors to
infer co-occurrence probabilities, a part of the links between species
is captured, but this part remains uncertain. This questions the quality
of the prediction classical species distribution models make.

My findings bring new theoretical element on the contribution of
ecological interactions on the shape of species geographical repartition
and also introduce a original method to study the species co-occurrence
: examining them in the light of ecological networks. Before concluding,
I propose in chapter \ref{chap4} a promising avenue to go further in the
integration of interaction in biogeography: introducing them through
energetic constraints. This last chapter presents the hope I base for a
metabolic theory of biogeography.

\begin{quote}
Keywords: Biogeography, biotic interaction, ecological networks, abiotic
constrains, co-occurrence, the theory of island biogeography, metabolic
theory of ecology.
\end{quote}
