Biogeography is concerned with the study of mechanisms shaping the
geographic repartition of species. Although the list of mechanisms is
clearly identified, biogeographers still search for a consistent theory
to structure their interplay. Among the challenge frequently pinpointed
is the integration os ecological interaction in species distribution
models. On the one hand, the literature points out evidences underlining
the controlling role of interaction on local communities structure. On
the other hand, relatively few studies unravel imprints of interaction
in species distribution data. Developing a clear explanation for this
apparent vanishing remains a major challenge biogeography needs to take
up. The main issue regarding the lack of a clear answer regarding the
role of interactions at broad spatial scales is that most of scenarios
of biodiversity changes assume that interactions can be neglected. If
this hypothesis is often proved false, then the scenarios must be
reviewed and sustain the development of methodologies including
relationship among species.

The thesis herein presented is a brainwork on the role of interaction on
species distribution. I start with a theoretical investigation on the
topic as classical theories push ecological interactions into the
background. Hence, in chapter \ref{chap1}, I present the integration of
interaction into a a theoretical model of species distribution coming
from one of the most important theory in biogeography: the theory of
island biogeography. This work shows how the joint effect of biotic and
abiotic factors affect the expectations derived from the classical
theory. Based on this first chapter, in chapter \ref{chap2}, I show how
interaction can affect co-occurrence data. Such data contains the
presence or absence of several species for a similar set of sites
dispersed along a large geographic gradient. Using a probabilistic
model, I obtain theoretical results linking co-occurrence data and the
information included in ecological networks. I clearly demonstrate that
interactions shape co-occurrence data. I further show that the higher
the number of links between two species the harder the detection of
interactions. Similarly if a species experience many interactions, it is
then challenging to detect any sign of interactions in co-occurrence
data.

In chapter\ref{chap3}, I print the analysis of five datasets of
co-occurrence for which the description of interactions was available.
This analysis confirms the hypotheses made in chapter \ref{chap2}. I
show that interacting species co-occur differently from non-interacting
one. My results also point out the abundance of interaction must
preclude their detection in co-occurrence data. However, by integrating
the similarly of abiotic factor for the different sites, I show that
signals of interactions are weakened. My results suggest

Mes résultats indiquent également que l'abondance d'interactions est un
frein à leur détection dans les données de co-occurrence. Cependant, en
intégrant la similarité des facteurs abiotiques pour les différents
sites, je montre que les signaux de co-occurrence s'affaiblissent e
réduisent pour parfois disparaitre. Mes résultats suggèrent qu'en
utilisant des facteurs abiotiques pour inférer les probabilités de
co-occurrence entre espèces, une partie du lien entre les espèces est
capturée, mais cette part capturée est entachée d'une grande
incertitude. Ceci vient questionner la capacité des modèles de
distribution d'espèces à prédire correctement la distribution des
espèces.

Mes résultats apportent des éléments théoriques nouveaux sur le rôle des
interactions écologiques dans les modèles de distribution d'espèces en
plus de proposer une démarche originale pour étudier les données de
co-occurrence d'espèces~: les regarder à la lumière des réseaux
écologiques. Avant de conclure ma thèse, je propose au chapitre
\ref{chap4} une démarche prometteuse pour aller encore plus loin dans
l'intégration des interactions en biogéographie~: intégrer ces dernières
par le biais des contraintes énergétiques. Ce dernier chapitre livre les
espoirs que je fonde sur une théorie métabolique de la biogéographie.

\begin{quote}
Mots clés: Biogéographie, Réseaux écologiques, contraintes abiotiques,
co-occurrence, théorie de la biogéographie des îles, théorie métabolique
de l'écologie.
\end{quote}
