La biogéographie est l'étude des mécanismes qui influencent la
répartition géographique des espèces. Bien que ces mécanismes soient
relativement bien identifiés, les biogéographes cherchent toujours une
théorie cohérente pour les articuler. Parmi les problèmes les plus
fréquemment pointés du doigt, figure le problème de l'intégration des
interactions écologiques dans les modèles de distribution d'espèces.
D'un côté la littérature scientifique apporte un ensemble de preuves
soulignant le rôle prépondérant des interactions dans la structuration
des communautés locales et de l'autre, on trouve finalement assez peu
d'études révélant les empreintes laissées par les interactions dans les
données de distributions. Proposer une explication claire à la
disparition de l'effet des interactions écologiques sur la distribution
des espèces quand on augmente l'échelle spatiale, reste un défi
important que la biogéographie doit mener. Le problème majeur que pose
l'absence de réponse claire sur le rôle que jouent les interactions aux
larges échelles est que la plupart des scénarios de biodiversité partent
de l'hypothèse que ces interactions sont négligeables aux larges
échelles spatiales. Si cette hypothèse est régulièrement montrée fausse,
alors il faut réviser ces scénarios et soutenir le développement de
méthodologies incluant les relations entre les espèces.

La travail de thèse présenté ici est une réflexion sur le rôle des
interactions dans la répartition des espèces. Je pars de l'idée que la
question des interactions en biogéographie doit être menée au niveau
théorique. Les théories classiques en biogéographie relèguent souvent au
second plan les interaction écologiques. Ainsi, au chapitre \ref{chap1}
je propose une intégration des interactions écologiques dans un modèle
théorique de distribution d'espèces issue d'une des théories les plus
importantes en biogéographie: la théorie de la biogéographie des îles.
Ce travail montre comment les effets conjoints des facteurs biotiques et
abiotiques peuvent changer profondément les attendus de la théorie
classique.

En partant de cette première réflexion, je montre au chapitre
\ref{chap2} comment les interactions peuvent se répercuter dans les
données de co-occurrence. Ces données indiquent la présence ou l'absence
de plusieurs espèces sur un même ensemble de sites dispersés le long
d'un large gradient géographique. À l'aide d'un modèle probabiliste,
j'obtiens des résultats théoriques liant les données de co-occurrence et
l'information contenue dans les réseaux écologiques. Je démontre
clairement que les interactions affectent les données de co-occurrence.
Je montre également que plus le nombre d'interactions entre deux espèces
est grand, moins ces interactions sont détectables et de même si une
espèce entretient de nombreuses interactions, il sera difficile de
trouver une quelconque trace des interactions dans les données de
co-occurrence.

Au chapitre \ref{chap3} de la présente thèse, je présente une analyse de
cinq jeux de données de co-occurrence pour lesquels une description des
interactions était disponible. Cette analyse confirme qu'un certain
nombre d'hypothèses formulées au chapitre \ref{chap2}. Je montre que les
espèces qui interagissent co-occurrent différemment qui celles
n'interagissent pas. Mes résultats indiquent également que l'abondance
d'interactions est un frein à leur détection dans les données de
co-occurrence. Cependant, en intégrant la similarité des facteurs
abiotiques pour les différents sites, je montre que les signaux de
co-occurrence sont très diminués et parfois disparaissent. Mes résultats
suggèrent qu'en utilisant les données sur les facteurs abiotiques pour
inférer les probabilités de co-occurrence entre espèces, une partie du
lien entre les espèces est capturée mais qu'il existe une grande
incertitude sur la part capturée, ce qui questionne la capacité des SDMs
à prédire correctement les distributions d'espèces.

Mes résultats apportent des éléments théoriques nouveaux sur le rôle des
interactions dans les modèles de distributions et proposent une démarche
nouvelle pour étudier les données de co-occurrence~: les regarder à la
lumière des réseaux écologiques. Avant de conclure ma thèse, je propose
au chapitre \ref{chap4} une démarche prometteuse pour aller encore plus
loin dans l'intégration des interaction en biogéographie~: intégrer ces
dernières par le biais des contraintes énergétiques. Ce dernier chapitre
livre les espoirs que je fonde sur une théorie métabolique de la
biogéographie.
