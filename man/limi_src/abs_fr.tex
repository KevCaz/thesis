La biogéogrpahie est le champs de la biologie qui s'intéresse au
distribution d'espèces. De nombreux facteurs influencenel la
configuration spatiale des distribution d'espècesé Parmis ces facteurs,
il y a un particuleir, les interactions écologiques. Les espèces sont en
effet relié par des interaction qui les rendent interdépendantes. Ce
lien qui est au coeur de la biologe n,est cependant pas interprétter en
imapct sur les distributions d'espèces. C'est un enejur crucila pour une
meilleur connaissance des des détails dans les aires de réapritions mais
aussi capital pour pouvoir prédire ces distribvutions. ++ cont

Dans ma thèse, dans un premier temps je porpose une refleions sur
l'int.frations des interactions dans un modèle théorique de distribution
d'espèce issue d'une des théorie les plus importante en Biogéographie:
la Théorie de la biogéogrpahie des îles. Je montre comnent il est
possible d'envisager les effetes sjumelées des contriantes biotiques et
abiotiques sur les répatriton g.orgrpahiques des espèces.

A partir de ce première reflexion, je porpose de montrer comment les
interactions peuvent se repercuter dans les données de co-ococcurecen
des espèces. Les données d'occurrence des espàes sont une des sources
princpasles d,information pour les bioégogrpahe.s Regardez
simulatnnément des données d'occureence nous donne des données de
o-occurrence. Je montre que si les interactions laissent des traces,
elles soivent petre axminées à la lumière de l'infornation géographique.
Je montre alors que l'abondance des interactions peut être un frein à
leur detections.

ces idées sont par la suite confirmées par des données empiriques. Dans
ces donn.es je montre que celon les ssyteèmes le s lines netre les ec;es
peut petre récélé. Je montre que pour les prédateirs le nombre de pories
est impornat . de Plus la distribution jumulé des proies est un objet
sous estmé et purtant riche en onfoamtions. Je discute alrs des
médlogies utilisée de manière courante et pitantcertains défautr et en
souhaitant des apporches plusancrée à la biologique.

En nontrat que la tjéoriq est importnat e pour réexamnee je pripose une
perspective métabolique pour aller encore plus loind dans a
conmpréhension et montrantdans l'espoir que cretaines lois existent.
