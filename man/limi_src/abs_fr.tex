La biogéographie est l'étude des mécanismes qui influencent la
répartition géographique des espèces. Bien que ces mécanismes soient
relativement bien identifiés, les biogéographes cherchent toujours une
théorie cohérente pour les articuler. Parmi les problèmes les plus
fréquemment pointés du doigt, figure celui de l'intégration des
interactions écologiques dans les modèles de distribution d'espèces.
D'un côté, la littérature scientifique apporte un ensemble de preuves
soulignant le rôle prépondérant des interactions dans la structuration
des communautés locales et de l'autre, on trouve relativement peu
d'études révélant les empreintes laissées par les interactions dans les
données de distribution d'espèces. Proposer une explication claire à
cette apparente disparition del'effet des interactions écologiques reste
un défi important que la biogéographie doit mener. Le problème majeur
que pose l'absence de réponse claire sur le rôle des interactions aux
larges échelles spatiales est que la plupart des scénarios de
changements de biodiversité partent de l'hypothèse que les interactions
sont négligeables. Si cette hypothèse est régulièrement rejetée, alors
il faut réviser ces scénarios et soutenir le développement de
méthodologies incluant les relations entre les espèces.

Le travail de thèse présenté ici est une réflexion sur le rôle des
interactions dans la répartition des espèces. Je commence par traiter de
la question au niveau théorique car les théories classiques en
biogéographie relèguent souvent au second plan les interactions
écologiques. Ainsi, au chapitre \ref{chap1} je propose une intégration
des interactions écologiques dans un modèle théorique de distribution
d'espèces issue d'une des théories les plus importantes en
biogéographie: la théorie de la biogéographie des îles. Ce travail
montre comment les effets conjoints des facteurs biotiques et abiotiques
changent les attendus de la théorie classique. En m'appuyant sur ce
premier chapitre, je montre au chapitre \ref{chap2} comment les
interactions peuvent se répercuter dans les données de co-occurrence
d'espèces. Ces données indiquent la présence ou l'absence de plusieurs
espèces sur un même ensemble de sites dispersés le long d'un large
gradient géographique. À l'aide d'un modèle probabiliste, j'obtiens des
résultats théoriques liant les données de co-occurrence et l'information
contenue dans les réseaux écologiques. Je démontre clairement que les
interactions affectent les données de co-occurrence. Je montre également
que plus le nombre d'interactions entre deux espèces est grand, moins
ces interactions sont détectables. De même si une espèce entretient de
nombreuses interactions, il sera difficile de trouver une quelconque
trace des interactions dans les données de co-occurrence.

Au chapitre \ref{chap3}, je présente l'analyse de cinq jeux de données
de co-occurrence pour lesquels la description des interactions était
disponible. Cette analyse vient confirmer les hypothèses formulées au
chapitre \ref{chap2}. Je montre que les espèces qui interagissent
co-occurrent différemment de celles n'interagissant pas. Mes résultats
indiquent aussi que l'abondance d'interactions est un frein à leur
détection dans les données de co-occurrence. Cependant, en intégrant la
similarité des facteurs abiotiques pour les différents sites, je montre
que les signaux de co-occurrence s'affaiblissent pour parfois
disparaitre. Mes résultats suggèrent donc qu'en utilisant des facteurs
abiotiques pour inférer les probabilités de co-occurrence, une partie du
lien entre les espèces est capturée, mais cette part est entachée d'une
grande incertitude. Ceci vient questionner la qualité des prédiction
données par les modèles classiques de distribution d'espèces.

Mes résultats apportent des éléments théoriques nouveaux sur le rôle des
interactions écologiques dans le tracé des aires de répartition des
espèces en plus de proposer une méthode originale pour étudier les
données de co-occurrence d'espèces~: les regarder à la lumière des
réseaux écologiques. Avant de conclure ma thèse, je propose au chapitre
\ref{chap4} une démarche prometteuse pour aller encore plus loin dans
l'intégration des interactions en biogéographie~: les introduire par le
biais des contraintes énergétiques. Ce dernier chapitre livre les
espoirs que je fonde sur une théorie métabolique de la biogéographie.

\begin{quote}
Mots clés: Biogéographie, interactions biotiques, réseaux écologiques,
contraintes abiotiques, co-occurrence, théorie de la biogéographie des
îles, théorie métabolique de l'écologie.
\end{quote}
