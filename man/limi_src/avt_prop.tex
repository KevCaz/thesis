C'est animé par le désir de comprendre le fonctionnement des écosystèmes
que j'envoyai, à l'automne 2011, courriel à Nicolas Mouquet à l'automne
2011. Un courriel bien heureux car il me propulsa au Canada à la
rencontre de Dominique Gravel, fraîchement devenu professeur, et de
Nicolas Mouquet lui-même alors en visite. C'est à Rimouski que j'ai
découvert la biogéographie et que j'ai fait commencé mon stage de master
(maîtrise dit-on à Rimouski) qui a déboucha sur la thèse de doctorat
présentée dans les pages qui suivent. C'est avec Dominique Gravel,
Nicolas Mouquet et David Mouillot que j'ai découvert le problème de
l'intégration des interactions en biogéographie qui m'a profondément
intéressé et parcequ'ils m'ont mis le pied à l'étrier, je souhaite les
remercier.

Le doctorat que j'ai entrepris a été rendu possible grâce au soutien
financier de plusieurs institutions que je souhaite remercier ici. Les
trois premières années de mon doctorat ont été financées par une bourse
du ministère français de l'éducation nationale, de l'enseignement
supérieur et de la recherche que je remercie. J'ai également reçu un
soutien financier du Conseil de Recherches en Sciences Naturelles et en
Génie du Canada (CRSNG) que je remercie. Ce soutien m'a notamment permis
de faire une quatrième année de thèse très profitable. Pour mes
déplacements entre la France et le Canada, j'ai reçu un soutien du
programme de mobilité FRONTENAC destiné aux doctorants inscrits en
cotutelle de thèse franco-québécoise, je remercie tous ceux qui
participent au bon fonctionnement de ce programme. Je souhaite remercier
vivement le Centre de la Science de la Biodiversité du Québec (CSBQ) et
particulièrement l'équipe qui participe à son dynamisme, c'est-à-dire la
mise en place d'initiatives qui participent grandement à rassembler les
acteurs de la biodiversité au Québec. Je remercie aussi le CSQB pour les
bourse destinées aux étudiant qu'il met en place dont j'ai
personnellement bénéficié pour me rendre au laboratoire de Miguel Araújo
à Madrid. Je profite de mentionner le nom de ce chercheur pour le
remercier pour son accueil et pour les précieux conseils qu'il m'a
donné.

J'ai passé quatre ans à faire un doctorat en écologie et parmi les défis
les plus délicats auxquels je me suis confronté, j'en ai relevé un bien
singulier~: celui d'expliquer à des noms spécialistes mon travail.
Comment, en effet, comprendre que je travaille sur des questions
importantes relatives à la biodiversité alors que je passe mon temps à
coder derrière un ordinateur. Il faut une certaine habilité pour
expliquer les tenants et les aboutissants de mon travail à celui qui m'a
répondu que lui aussi triait ces poubelles quand je lui ai dit que je
faisais de l'écologie. L'effort de communication sur ce que je fais me
semble crucial et je souhaite prendre le temps à l'avenir pour faire cet
effort.

J'ai passé quatre ans à étudier certains aspects de la biodiversité sans
avoir tellement fait part de mes convictions citoyennes sur le sujet. Je
cherche encore à concilier le chercheur que je souhaite devenir et le
citoyen que je suis déjà, mais en attendant d'y parvenir c'est par un
message du citoyen que je suis que je souhaite ouvrir ma thèse en
empruntant les mots du \textit{Petit Prince} d'Antoine de Saint
Exupéry~:

\begin{quote}
--- Les hommes de chez toi, dit le petit prince, cultivent cinq milles
roses dans un même jardin\ldots{}. et ils n'y trouvent pas ce qu'ils
cherchent\ldots{}.\\
--- Ils ne trouvent pas, répondis-je\ldots{}\\
--- Et cependant ce qu'ils cherchent pourrait être trouvé dans une seule
rose ou un peu d'eau\ldots{}\\
--- Bien sûr, répondis-je.\\
Et le petit prince ajouta :\\
--- Mais les yeux sont aveugles. Il faut chercher avec le coeur.
\end{quote}
