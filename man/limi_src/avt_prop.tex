En envoyant un mail à Nicolas Mouquet à l'automne 2011, je n'avais pas
d'objectif particuliers bien défini, j'étais intéreesser par l'écologie
théorique, je voulais me questionner sur le fonctionnement des
écosystème. La réponse de Nicolas fut positive et je fus rapidement
propulsé au Canada au laboratoire de Dominique Gravel fraichemennt
deveneu professeur pour faire mon stage de Master 2 (maîtrise dit-on au
Québec). J'ai alors découvert un peu plus de quio il retournait et me
suis intéressé aux problèmes desinteractions sencouragé par Dominique
Gravel, Nicolas Mouquet et David Mouillot. L'aventure se poursuivie au
doctorat et et m'amena justqu'à déposer la présente thèse.

Le doctorat a été appuyé et je remercie les institutions qui rendent
cela possible. Je tiens à remercier le Misistère de l'Eduaction
nationale le FRQNT, au programme de soutien à la mogiilit. FROBTENAC qui
encoruge les co-tuetelle. Je remercie aussi le CSBQ et l'ensemble de
ceux qui s'activent pour mettre en place des initiatives très pertinente
pour soutenir la réunion annuel, ateleir R, une belle animation des
science de la biodiversité. permis aussi d'aller faire un séjour chez
MiguelAraujo que je tosn à remercuer. CRSNG (Conseil de Recherches en
Sciences Naturelles et en Génie du Canada)

J'ai passé quatre ans à faire un doctorat en écologie, parmis les défis
les plus diurs ququal je me suis confronté a été celui d'expliquer à des
noms spécialistes mon travail notamment pourquoi je passais la plupart
de mon tempsà travailer sur la biodiverstié dernière un ordinateur. Le
travail que je propose se réclame j'essaye de me contraindre dans une
argumentation aussi soloe qu epossible et basé sur des études qui ont
pour objcetcifs de comrednre le mode vivant ce qui est un challenge que
Lon doit se confonréer pour en mesurer la compléxité. Je suis beinsur un
cityoen et mes avis sur de qu'il se passent eet c'est en relisant
simplement par les mots du petis prince que j'aimerais touché avant de
laisser place à ma thèse qui se veut aussi rigoureuse que possible.

\begin{quote}
--- Les hommes de chez toi, dit le petit prince, cultivent cinq milles
roses dans un même jardin\ldots{}. et ils n'y trouvent pas ce qu'ils
cherchent\ldots{}.\\
--- Ils ne trouvent pas, répondis-je\ldots{}\\
--- Et cependant ce qu'ils cherchent pourrait être trouvé dans une seule
rose ou un peu d'eau\ldots{}\\
--- Bien sûr, répondis-je.\\
Et le petit prince ajouta :\\
--- Mais les yeux sont aveugles. Il faut chercher avec le coeur.
\end{quote}
