%----------------------------------------------------------------------%
% Liminaires de la thèse.                                              %
% UQAR septembre 2013                                                  %
% ---------------------------------------------------------------------%

% ----------------------------------------------------------------------%
% 1- Page titre.                                                        %
% ----------------------------------------------------------------------%

\Pagetitre
\cleardoublepage
% ----------------------------------------------------------------------%
% inclusions qui pourraient mériter d'être incluses dans le .cls
% (commentez si non-nécessaire)
% 1.1 - Composition du Jury.                                           %
\thispagestyle{empty}

\null
\vfill
\noindent COMPOSITION DU JURY: \\

\begin{singlespace}
  \noindent \textbf{Steven Kembel}, président du jury, Université du Québec à Montréal\\

  \noindent \textbf{Dominique Gravel}, directeur de recherche, Université du Québec à Rimouski\\

  \noindent \textbf{Nicolas Mouquet}, directeur de recherche, Université de Montpellier\\

  \noindent \textbf{Phillipe Jarne}, examinateur interne, Centre d'Écologie Fonctionnelle et Évolutive\\

  \noindent \textbf{Piero Calosi}, examinateur interne, Université du Québec à Rimouski\\

  \noindent \textbf{Nicholas J. Gotelli}, examinateur externe, University of Vermont\\
\end{singlespace}

\vspace{1.5cm}

\begin{itemize}
\item Dépôt initial le 23 septembre 2016
\item Soutenance le 13 décembre 2016
\item Dépôt final le 10 février 2017
\end{itemize}



\cleardoublepage

% % 1.2 - Avertissement biblio.
\thispagestyle{empty}

\vspace{2cm}
\begin{center}
UNIVERSITÉ DU QUÉBEC À RIMOUSKI\\
Service de la bibliothèque
\end{center}

\vspace{3cm}
\begin{center}
Avertissement
\end{center}


\vspace{1cm}

\noindent La diffusion de ce mémoire ou de cette thèse se fait dans le respect des droits de son auteur, qui a signé le formulaire {\itshape \og Autorisation de reproduire et de diffuser un rapport, un mémoire ou une thèse \fg}. 
En signant ce formulaire, l’auteur concède à l’Université du Québec à Rimouski une licence non exclusive d’utilisation et de publication de la totalité ou d’une partie importante de son travail de recherche pour des fins pédagogiques et non commerciales. 
Plus précisément, l’auteur autorise l’Université du Québec à Rimouski à reproduire, diffuser, prêter, distribuer ou vendre des copies de son travail de recherche à des fins non commerciales sur quelque support que ce soit, y compris l’Internet. 
Cette licence et cette autorisation n’entraînent pas une renonciation de la part de l’auteur à ses droits moraux ni à ses droits de propriété intellectuelle. 
Sauf entente contraire, l’auteur conserve la liberté de diffuser et de commercialiser ou non ce travail dont il possède un exemplaire.



\cleardoublepage
% % 1.3 - Dedicace.
\thispagestyle{empty}

\begin{minipage}[l]{0.45\textwidth}

\end{minipage}%
\hfill
\begin{minipage}[r]{0.5\textwidth}
\begin{quotation}
\begin{doublespace}

à JASCERB,

\end{doublespace}
\end{quotation}
\end{minipage}%

\cleardoublepage

% ----------------------------------------------------------------------%


% ----------------------------------------------------------------------%
% 2- Remerciements.                                                    %
% ----------------------------------------------------------------------%

\remerciements
\selectlanguage{frenchb}
Au royaume des idées, nous transformons les fragments de notre existence
en songes. Chaque conversation est une caisse de résonance offerte à nos
propres songes. Chaque lecture est une conversation, un échange avec
d'autres qui bien qu'étant absents, nous livrent leurs pensées. Comment
alors ne pas croire que tout le travail ici présenté n'est autre que le
récit de ma rencontre et des échanges avec tous ces autres? Un récit qui
à son tour deviendra une nouvelle source de songes pour des lecteurs que
je ne connaîtrai peut-être jamais.

L'aventure du doctorat est bien plus collective que ce qu'elle ne semble
être. Longue est la liste de ceux qui apportèrent une pierre à l'édifice
ici présenté. Des pierres les plus solides pour bâtir les fondations aux
plus petits cailloux qui donnent un rien de charme et d'originalité à
l'ensemble, nombreux sont les contributeurs au présent travail. À tous
ces autres qui parfois ignorent ce qu'ils m'ont offert, j'adresse un
chaleureux merci. Certaines rencontres laissent des empreintes plus
profondes et méritent que je leur destine des remerciements plus
personnels.

Pour m'avoir donné ma chance et offert une confiance sans faille, je
veux remercier vivement Dom, Nico et David, sans qui l'aventure n'aurait
pas été la même. Pour faire ces premiers pas sur un chemin incertain,
mieux vaut partir avec ceux qui pensent qu'on en trouvera la fin.

Un chemin qui m'amena sur deux continents, dans deux pays, à travailler
dans deux universités au sein de deux laboratoires. Le laboratoire, ce
n'est pas seulement un lieu, ce sont aussi des personnes, autant de
sources de réflexion et d'inspiration qui jamais ne tarissent. Pour de
précieux échanges à Rimouski, je remercie Hedou, Idalflex, Jacquette, Jo
Brasco, Pippin le Solver, Tim. Solarik, les \emph{behind the scenes} que
nous avons abondamment échangés m'ont été d'une précieuse aide dans les
derniers moments. Pour de passionnantes conversations à Montpellier, un
grand merci Alain, Andreï, Claire, Clara, Florian, Isa, Marie, Pierre et
Simon.

Dans les couloirs des laboratoires, il y a aussi des liens qui se nouent
autour d'intérêts convergents et qui se transforment en une amitié très
appréciée. Pour tout ce qu'ils m'ont apporté, je remercie Sonia (et
Thomas), Albouy (et Séverine et Léo et Louis) et le joyeux Legagneux (et
Aurélie et Margot et Juliette et Zélie et Romane). Dans ces mêmes
couloirs, j'en ai croisé se frottant des yeux fatigués par la lumière
des écrans. Pour mes geek préférés, ceux qui connaissent comme moi
l'appel du clavier, quand l'envie de coder devient trop forte, je désire
taper un \texttt{\textbackslash{}\textbackslash{}huge\{merci\}}. Merci
Casajus, Team Zissou, Dave et Flaul pour les morceaux de code et bien
plus: votre enthousiasme et votre insatiable curiosité.

D'un pays à l'autre, d'un laboratoire à l'autre, d'un couloir à l'autre,
d'un projet à l'autre, l'aventure n'est, bien heureusement, nullement
monochromatique et le rose de certains instants cède régulièrement la
place à un gris parfois bien sombre. Partagés ses questionnements, ses
colères, ou encore sa tristesse avec d'autres voyageurs est un soutien
plus que précieux. Marion, Matoche, Clem, pour ce soutien qui a été si
important, je vous remercie mille fois.

Pour ne pas faiblir face aux péripéties, le lieu de repos soit être
choisi avec soin. Le temps du doctorat, il est important de trouver son
havre de paix. Pour avoir été les co-locataires de cet endroit
merveilleux qu'a été la Maison des Courges, un immense merci à Camille,
Élo, Gigi, Jean, Jerem, Jerem, Lau, Léo, Marie, Palardy, TriTri.
Palette, bien sur, un remerciement tout particulier pour toi et pour
tout ce que nous partageons.

Certains alliés étaient là avant même que l'aventure ne commence et
constituèrent des repères essentiels dans un voyage quelques fois
extrêmement déboussolant. À ces étoiles du sol qui nous guident quand la
nuit vient de tomber, j'adresse un grand merci.

Ainsi, à mes amis du TerTer de Nanterre, qui me rappellent où j'ai
grandi, les joies de mon parcours en banlieue parisienne, à ceux que
j'ai connu il y a parfois plus de vingt ans, à vous Ariane, Bibi,
Cendrars, Gomar, Gronico, K-wik, Matos, Miste, Rufo, Tinico, je vous dit
un immense cimer! Un spécial cimer pour toi Rhum, mon frérot, qui n'a
jamais cessé de me surprendre.

À ma famille, ceux dont je partage parfois une partie de mon patrimoine
génétique, mais souvent tellement plus. Un profond et chaleureux merci à
Jean-Louis, Josette, Monique, Jean-Claude, Yvette, Tatoche, Tonton
François, c'est toujours un immense bonheur de vous voir.

Papi, Mami, éternels soutiens, guides indispensables de mes premier pas
à aujourd'hui, merci. Des grands-parents comme vous tous les petits
enfants en rêve\ldots{} Je vous le promets, regardez, je fais de la
science, je vais être docteur, c'est forcément vrai. Sachez que
l'aventure sans vous aurait été bien plus pénible.

Papa, père et pair, c'est quand même pas mal! Ton appui au court de ce
long voyage m'a été d'une aide au combien précieuse. Tu as été un
éclaireur génial vers qui j'ai pu me tourner quand l'horizon semblait
encore bien loin, merci papa.

ClaCla et Nico le haricot, merci pour votre générosité, votre folie et
pour le bonheur que vous respirez, je vous dois pas mal de sourire dans
les moments où j'étais au plus bas. Merci Pépette pour ta spontanéité et
ta fraicheur, je sais que j'étais absent ces derniers temps, il faudra
qu'on se rattrape!

Maman, la plus belle des mamans. Ton fils n'est pas souvent passé par la
maison ces dernières années, et oui, l'aventure m'a éloigné de mes
terres d'origine. Rassures-toi, ton fils ne t'oublie pas, il sait très
bien tout ce que tu as fait pour lui jusqu'à aujourd'hui et il t'en est
éternellement reconnaissant.


% [Cette page est facultative; l’éliminer si elle n’est pas utilisée. Les remerciements peuvent aussi être intégrés à l'avant-propos. C’est dans cette section que l’on remercie les personnes qui ont contribué au projet, les organismes ou les entreprises subventionnaires qui ont soutenu financièrement le projet.]



% ----------------------------------------------------------------------%
% 3- Avant-propos.                                                     %
% ----------------------------------------------------------------------%

\avantpropos
\selectlanguage{frenchb}
C'est animé par le désir de comprendre le fonctionnement des écosystèmes
que j'envoyai, à l'automne 2011, un courriel à Nicolas Mouquet. Un
courriel bien heureux car il me propulsa au Canada à la rencontre de
Dominique Gravel, fraîchement devenu professeur à l'UQAR, et de Nicolas
Mouquet lui-même alors en visite. C'est à Rimouski que j'ai découvert la
biogéographie et que j'ai fait commencé mon stage de master (maîtrise
dirait-on à Rimouski) qui déboucha sur la thèse de doctorat présentée
dans les pages qui suivent. C'est avec Dominique Gravel, Nicolas Mouquet
et David Mouillot que j'ai découvert le problème de l'intégration des
interactions écologiques en biogéographie qui m'a profondément
intéressé. Ces trois chercheurs m'ont mis le pied à l'étrier et je
souhaite les remercier vivement pour cela.

Le doctorat que j'ai entrepris a été rendu possible grâce au soutien
financier de plusieurs institutions que je souhaite remercier ici. Les
trois premières années de mon doctorat ont été financées par une bourse
du ministère français de l'éducation nationale, de l'enseignement
supérieur et de la recherche, qu'il en soit remercier. J'ai également
reçu un soutien financier du Conseil de Recherches en Sciences
Naturelles et en Génie du Canada (CRSNG) que je remercie. Ce soutien m'a
notamment permis de faire une quatrième année de thèse qui m'a été très
profitable. De même, je remercie le programme des chaires de recherche
du Canada qui, en plus de m'avoir offert un complément de bourse, a
soutenu la mise en place du laboratoire de Dominique Gravel au sein
duquel j'ai préparé la thèse ici présentée. Pour mes déplacements entre
la France et le Canada, j'ai reçu un soutien du programme de mobilité
FRONTENAC destiné aux doctorants inscrits en cotutelle de thèse
franco-québécoise, je remercie tous ceux qui participent au bon
fonctionnement de ce programme. J'adresse un grand merci au Centre de la
Science de la Biodiversité du Québec (CSBQ) et particulièrement à
l'équipe qui participe à son dynamisme avec la mise en place
d'initiatives qui participent fortement à rassembler les acteurs de la
biodiversité au Québec. Je remercie aussi le CSBQ pour les bourses
destinées aux étudiants qu'il met en place et dont j'ai personnellement
bénéficié pour me rendre au laboratoire de Miguel Araújo à Madrid. Je
profite de mentionner le nom de ce chercheur pour le remercier de son
accueil et pour les précieux conseils qu'il m'a prodigué.

J'ai passé quatre ans à faire un doctorat en écologie et parmi les défis
les plus délicats auxquels je me suis confronté, j'en ai relevé un bien
singulier~: celui d'expliquer, à des noms spécialistes, mon travail de
thèse. Comment, en effet, comprendre que je m'intéresse à des questions
importantes relatives à la biodiversité alors que je passe mon temps à
coder derrière un ordinateur. Il faut une certaine habilité pour
expliquer les tenants et les aboutissants de mon travail à celui qui m'a
répondu que lui aussi triait ces poubelles quand je lui ai déclaré faire
de l'écologie. L'effort de communication sur mes activités de recherche
me semble, dans le context actuel, crucial et je souhaite prendre le
temps à l'avenir pour faire cet effort.

J'ai passé quatre ans à étudier certains aspects de la biodiversité et
je n'ai pourtant pas donné beaucoup de place è l'expression de mes
convictions citoyennes sur le sujet. Je cherche encore à concilier le
chercheur que je souhaite devenir et le citoyen que je suis déjà. Dans
l'attente d'une telle conciliation, c'est par un message du citoyen que
je suis que je souhaite ouvrir ma thèse en empruntant les mots du
\textit{Petit Prince} d'Antoine de Saint Exupéry~:

\begin{quote}
--- Les hommes de chez toi, dit le petit prince, cultivent cinq milles
roses dans un même jardin\ldots{}. et ils n'y trouvent pas ce qu'ils
cherchent\ldots{}.\\
--- Ils ne trouvent pas, répondis-je\ldots{}\\
--- Et cependant ce qu'ils cherchent pourrait être trouvé dans une seule
rose ou un peu d'eau\ldots{}\\
--- Bien sûr, répondis-je.\\
Et le petit prince ajouta :\\
--- Mais les yeux sont aveugles. Il faut chercher avec le coeur.
\end{quote}



% [Cette page est facultative; l’éliminer si elle n’est pas utilisée. L’avant-propos ne doit pas être confondu avec l'introduction. Il n’est pas d’ordre scientifique alors que l’introduction l’est. Il s’agit d'un discours préliminaire qui permet notamment à l'auteur d'exposer les raisons qui l'ont amené à étudier le sujet choisi, le but qu'il veut atteindre, ainsi que les possibilités et les limites de son travail. On peut inclure les remerciements à la fin de ce texte au lieu de les présenter sur une page distincte.]



% ----------------------------------------------------------------------%
% 4- Resume/Abstract                                                           %
% ----------------------------------------------------------------------%

\resume
\begin{singlespace}
La biogéogrpahie est le champs de la biologie qui s'intéresse au
distribution d'espèces. De nombreux facteurs influencenel la
configuration spatiale des distribution d'espècesé Parmis ces facteurs,
il y a un particuleir, les interactions écologiques. Les espèces sont en
effet relié par des interaction qui les rendent interdépendantes. Ce
lien qui est au coeur de la biologe n,est cependant pas interprétter en
imapct sur les distributions d'espèces. C'est un enejur crucila pour une
meilleur connaissance des des détails dans les aires de réapritions mais
aussi capital pour pouvoir prédire ces distribvutions.

Dans ma thèse, dans un premier temps je porpose une refleions sur
l'int.frations des interactions dans un modèle théorique de distribution
d'espèce issue d'une des théorie les plus importante en Biogéographie:
la Théorie de la biogéogrpahie des îles. Je montre comnent il est
possible d'envisager les effetes sjumelées des contriantes biotiques et
abiotiques sur les répatriton g.orgrpahiques des espèces.

A partir de ce première reflexion, je porpose de montrer comment les
interactions peuvent se repercuter dans les données de co-ococcurecen
des espèces. Les données d'occurrence des espàes sont une des sources
princpasles d,information pour les bioégogrpahe.s Regardez
simulatnnément des données d'occureence nous donne des données de
o-occurrence. Je montre que si les interactions laissent des traces,
elles soivent petre axminées à la lumière de l'infornation géographique.
Je montre alors que l'abondance des interactions peut être un frein à
leur detections.

ces idées sont par la suite confirmées par des données empiriques. Dans
ces donn.es je montre que celon les ssyteèmes le s lines netre les ec;es
peut petre récélé. Je montre que pour les prédateirs le nombre de pories
est impornat . de Plus la distribution jumulé des proies est un objet
sous estmé et purtant riche en onfoamtions. Je discute alrs des
médlogies utilisée de manière courante et pitantcertains défautr et en
souhaitant des apporches plusancrée à la biologique.

En nontrat que la tjéoriq est importnat e pour réexamnee je pripose une
perspective métabolique pour aller encore plus loind dans a
conmpréhension et montrantdans l'espoir que cretaines lois existent.

  % [Le résumé en français doit présenter en 350 mots maximum pour un mémoire et en 700 mots pour une thèse : (1) le but de la recherche, (2) les sujets étudiés, (3) les hypothèses de travail et la méthode utilisée, (4) les principaux résultats et (5) les conclusions de l'étude ou de la recherche.]

\end{singlespace}
\cleardoublepage


\abstract
\begin{singlespace}
Biogeography is the study of the mechanisms and processes that control
the geographical distribution of plants and animals. Although the list
of mechanisms is clearly identified, biogeographers are still struggling
to find a consistent theory to integrate their interaction. One of the
most pressing challenges currently in the field of biogeography is the
successful integration of ecological interactions in species
distribution models. Although the scientific literature points out the
evidence of the controlling role interactions play on local community
structure, relatively few studies have demonstrated its importance over
large geographical gradients. Developing a concise, clear explanation
for this issue remains a significant challenge that biogeographers need
to answer. The main issue associated to the lack of a clear answer
concerning the role of interactions at broad spatial scales is that most
of scenarios of biodiversity changes assume that interactions can be
ignored. When tested, if this hypothesis is proven false, then a
re-consideration of species distribution models and their development
must be undertaken to include relationships among species.

In this thesis I tackle the issue of integrating the species
interactions, and how this affects species distributions. I begin this
thesis with a theoretical investigation on this topic, where classical
theories have typically ignored ecological interactions. In the first
chapter of the thesis I present the integration of interaction networks
into a theoretical model of species distribution coming from one of the
most important theory in biogeography: the theory of island
biogeography. This work shows how together the biotic and abiotic
factors can affect the expectations derived from the classical theory.
Building upon the findings in the first chapter, in the second chapter,
I show how interactions can affect co-occurrence (between species) data.
Such data contains the presence or absence of several species for a
similar set of sites dispersed along large latitudinal gradients. Using
a probabilistic model, I obtain theoretical results linking
co-occurrence data and the information included in ecological networks.
I clearly demonstrate that interactions shape co-occurrence data.
Furthermore, I show that the higher the number of links between two
species, the more difficult it is to detect their indirect interaction.
Similarly, if a species experiences many interactions, it is then
challenging to detect any sign of interactions in static co-occurrence
data for this species.

In the third chapter of the thesis, I assess five sets of co-occurrence
data, which had descriptions of their interactions available. Using this
data, I was able to confirm my hypotheses put forth in my second
chapter, by showing that species co-occur differently from
non-interacting one. These results also point out that the abundance of
interaction must preclude their detection in co-occurrence data.
However, when accounting for abiotic similarities among sites, signals
of interactions are weakened. Therefore, my results suggest that using
abiotic factors to infer co-occurrence probabilities capture a part of
the link between species and further pinpoint the uncertainty associated
to this part. As a result of these findings, the predictive power of
classical species distribution models used to date is brought into
question.

My research findings bring new theoretical elements to the forefront
when considering the influence of ecological interactions and how they
shape species geographical distributions, while also introducing an
original methodology for studying species co-occurrence: examining them
in the light of ecological networks. Before concluding, my fourth and
final chapter, I propose a promising new avenue to further investigate
integrating species interactions in biogeography. Here, I introduce
interactions in terms of energetic constraints, which will provide a
sound basis for a metabolic theory of biogeography.

\begin{quote}
KEYWORDS: Biogeography, biotic interactions, ecological networks,
abiotic constrains, co-occurrence, theory of island biogeography,
metabolic theory of ecology.
\end{quote}


  % [L'abstract doit être une traduction anglaise fidèle et grammaticalement correcte du résumé en français.]

\end{singlespace}
\cleardoublepage




% ----------------------------------------------------------------------%
% 5- Table des matières.                                               %
% ----------------------------------------------------------------------%

\tabledesmatieres



% ----------------------------------------------------------------------%
% 6- Liste des tableaux.                                               %
% ----------------------------------------------------------------------%

\listedestableaux

% ----------------------------------------------------------------------%
% 7- Table des matières.                                               %
% ----------------------------------------------------------------------%

\listedesfigures

% ----------------------------------------------------------------------%
% 8- Liste des abréviations (optionnel).                               %
% ----------------------------------------------------------------------%

\listeabrev
\begin{liste}

\item[DOI] identifiant numérique d'objet, un numéro unique pour identifier des ressources numériques comme par exemple un article scientifique (en référence terme anglais~: \textit{Digital Object Identifier})

\item[JSDM] Modèle de distribution jointe d'espèce (en référence terme anglais~: \textit{Joint Species Distribution Model})

\item[SDM] Modèle de distribution d'espèce (en référence terme anglais~: \textit{Species Distribution Model})

\item[TIB] Théorie insulaire de la biogéographie

\item[TTIB] Théorie trophique de la biogéogrpahi des île (en référence terme anglais~: \textit{Species Distribution Model})

\end{liste}



% ----------------------------------------------------------------------%
% 9- Liste des symboles (optionnel).                                   %
% ----------------------------------------------------------------------%

% \listesymboles
% \begin{liste}
% \item[SYMBOLE 1] Ceci est la définition du symbole 1.
%
% \item[SYMBOLE 2] Ceci est la définition du symbole 2.
%
% \item[SYMBOLE 3] Ceci est la définition du symbole 3.
% \end{liste}

% ----------------------------------------------------------------------%
% Fin des liminaires.                                                  %
% ----------------------------------------------------------------------%

\cleardoublepage
