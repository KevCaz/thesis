%----------------------------------------------------------------------%
% Liminaires de la thèse.                                              %
% UQAR septembre 2013                                                  %
% ---------------------------------------------------------------------%

% ----------------------------------------------------------------------%
% 1- Page titre.                                                        %
% ----------------------------------------------------------------------%

\Pagetitre
\cleardoublepage
% ----------------------------------------------------------------------%
% inclusions qui pourraient mériter d'être incluses dans le .cls
% (commentez si non-nécessaire)
% 1.1 - Composition du Jury.                                           %
\thispagestyle{empty}

\null
\vfill
\noindent COMPOSITION DU JURY: \\

\begin{singlespace}
  \noindent \textbf{Steven Kembel}, président du jury, Université du Québec à Montréal\\

  \noindent \textbf{Dominique Gravel}, directeur de recherche, Université du Québec à Rimouski\\

  \noindent \textbf{Nicolas Mouquet}, directeur de recherche, Université de Montpellier\\

  \noindent \textbf{Phillipe Jarne}, examinateur interne, Centre d'Écologie Fonctionnelle et Évolutive\\

  \noindent \textbf{Piero Calosi}, examinateur interne, Université du Québec à Rimouski\\

  \noindent \textbf{Nicholas J. Gotelli}, examinateur externe, University of Vermont\\
\end{singlespace}

\vspace{1.5cm}

\begin{itemize}
\item Dépôt initial le 23 septembre 2016
\item Soutenance le 13 décembre 2016
\item Dépôt final le 10 février 2017
\end{itemize}



\cleardoublepage

% % 1.2 - Avertissement biblio.
\thispagestyle{empty}

\vspace{2cm}
\begin{center}
UNIVERSITÉ DU QUÉBEC À RIMOUSKI\\
Service de la bibliothèque
\end{center}

\vspace{3cm}
\begin{center}
Avertissement
\end{center}


\vspace{1cm}

\noindent La diffusion de ce mémoire ou de cette thèse se fait dans le respect des droits de son auteur, qui a signé le formulaire {\itshape \og Autorisation de reproduire et de diffuser un rapport, un mémoire ou une thèse \fg}. 
En signant ce formulaire, l’auteur concède à l’Université du Québec à Rimouski une licence non exclusive d’utilisation et de publication de la totalité ou d’une partie importante de son travail de recherche pour des fins pédagogiques et non commerciales. 
Plus précisément, l’auteur autorise l’Université du Québec à Rimouski à reproduire, diffuser, prêter, distribuer ou vendre des copies de son travail de recherche à des fins non commerciales sur quelque support que ce soit, y compris l’Internet. 
Cette licence et cette autorisation n’entraînent pas une renonciation de la part de l’auteur à ses droits moraux ni à ses droits de propriété intellectuelle. 
Sauf entente contraire, l’auteur conserve la liberté de diffuser et de commercialiser ou non ce travail dont il possède un exemplaire.



\cleardoublepage
% % 1.3 - Dedicace.
\thispagestyle{empty}

\begin{minipage}[l]{0.45\textwidth}

\end{minipage}%
\hfill
\begin{minipage}[r]{0.5\textwidth}
\begin{quotation}
\begin{doublespace}

à JASCERB,

\end{doublespace}
\end{quotation}
\end{minipage}%

\cleardoublepage

% ----------------------------------------------------------------------%


% ----------------------------------------------------------------------%
% 2- Remerciements.                                                    %
% ----------------------------------------------------------------------%

\remerciements
\selectlanguage{frenchb}

Au royaume des idées, nul ne c'est ce qu'il va rencontrer. Chauque rencontre est la promesse d'un échange. Chaque conversation est une caisse de raisonnance possible pour les idées qu'on se fait de notre propre ecpérience. Chaque échange est alors une possible soure d'inspiration pour sa réflexion. De fait, je ne peux faire la liste exahustive des gens avec qui les échanges ont conduit à la présente thèse. Je me restreint à une portion de ceux avec qui les échanges ont été plus soutenu et ceux qui de longue date me soutiennent. Longeur des présents remerciements demeure ainsi raisonable, en essayant d'être personnel et aussi avec une certaine pudeur qui m'ammène à l'anonymat. Ces remerciments venant du coeur, ils expriment ma gratitude à ceux qui ont été présents dans mon parcours je les remercient pour ce qu'ils sont.
Je ne suis pas tellemnent un homme de poésie, mais pourquoi se refuser le plaisre de comtrainge. Ainsi, ai-je choisi d'écrire mes remerciments en alexandrins deux hémistiches égales rimes croisés. Les remerciements plus formels sont dans l'avant porpos.





Longue est la liste des gens que j'ai cotoyé

La liste est longue des personnes qui ont soutenu plus ou moins directement.
On vot avec des personnes dont on se nourrit plus ou moins consciemment, ils participent à ce que nous sommes et l'on grandi avec eux Marie, Gigi, Elo, Léo, Alex, Tristan, Laurence, Elissa, Camille.

Des soutiens qui sont de longues dates, je ne sais pas dans quelle mesure ils ont contribue. On a tous des enfances plus ou moins joyeuxe mais on porte en nous la trace drôle de statuts en permanent chamnegent. La mienne a été une enfancemune adolescence très heureuse avec quelquesl remous. Dans son sillage on entraine des gens qui vous supporterons très longtemos, mais ils l'ont Miste, Gros Nico, Gras Nico, Gomar, Ariane, Cendrars, K-wik, Cimer !

La famille, certains disent qu'on ne l'a choisi pas, alors, j'avouerai être très chanceux. Merci Josette, Jean-Louis, Mireille. merci.

Nico, Benjami, Guy, votre amitié et vos produit...

Le boulot c'est des rencontres de gens passionnés et passionnant avant tout tellemnt nous et nous sommes tellemnent uni par le destin particiculier de la recherche. Merci Claire, Clara, Marie, Simon, Sonia, Andrei, Florian, Rob, Isabelle, Alain, Sonia. Merci Jaquet, Idaline, Hedou, Kevin, Jo Brasco, Pipin le Solver, Tim.

Legagneux, Casajus, Vissault, plaisir de travailler avec vous encore et encore.

Directeure Nicolas, Dominique merci pour la confiance. David

Merci d'être qui tu es Palette, ça fait du bien de savoir qu'il y en a des comme toi.

Chevrinias, Chevalier, Matoche, noué par le destin  ....

Albouy de Montpellier à Rimouski! Merci à toi et Severine,

Merci les soeurs, je sais que l'frétot est pas souvent là mais bin

Merci Mother

Papi, mami, éternelle soutien, des grands-parents comme vous tous les petits enfants en rêve... Peut pas être autrement, je fais de la science.




% [Cette page est facultative; l’éliminer si elle n’est pas utilisée. Les remerciements peuvent aussi être intégrés à l'avant-propos. C’est dans cette section que l’on remercie les personnes qui ont contribué au projet, les organismes ou les entreprises subventionnaires qui ont soutenu financièrement le projet.]



% ----------------------------------------------------------------------%
% 3- Avant-propos.                                                     %
% ----------------------------------------------------------------------%

\avantpropos
\selectlanguage{frenchb}
%
% [Cette page est facultative; l’éliminer si elle n’est pas utilisée. L’avant-propos ne doit pas être confondu avec l'introduction. Il n’est pas d’ordre scientifique alors que l’introduction l’est. Il s’agit d'un discours préliminaire qui permet notamment à l'auteur d'exposer les raisons qui l'ont amené à étudier le sujet choisi, le but qu'il veut atteindre, ainsi que les possibilités et les limites de son travail. On peut inclure les remerciements à la fin de ce texte au lieu de les présenter sur une page distincte.]

Je n'avais pas d'objectif particuliers une envie de m'intéresser à la biologie et plus généra;emnt les écocsystèmes. J'ai ainsi envoyé divers mail et j'ai eu une réponse de Nocicolas Mouquet qui m'a propulsé au Canada. La j'ai dcouvert les travaux entre Nocolas Moquet et Dominique Gravel dont je ne fais qu'essayer de les prolonger. Alors décider d'aller au doctorat et voila ce que ça donne

David Mouillot.

Merci Miguel Araujo.

Merci au soutien financier, il rende possible.

Merci France, merci frontenac, merci CSBQ. rendu possible ma thèse.

Merci UQAR / Université de Montpellier.
Merci au geek.


Les articles sont en angalis mais chaques section a une partie \emp{résumé en français} dans laquelle je détaille le contexte scientique, le contexte d'écriture et de publication et je traduit l'abstract.
Pour les parties en français, la thèse en cotutelle avec le québec j'ai utilisé la règle typograhique comme disponible ici

\url{http://bdl.oqlf.gouv.qc.ca/bdl/gabarit_bdl.asp?id=2039}

% ----------------------------------------------------------------------%
% 4- Resume/Abstract                                                           %
% ----------------------------------------------------------------------%

% \resume
% \begin{singlespace}
%
%   [Le résumé en français doit présenter en 350 mots maximum pour un mémoire et en 700 mots pour une thèse : (1) le but de la recherche, (2) les sujets étudiés, (3) les hypothèses de travail et la méthode utilisée, (4) les principaux résultats et (5) les conclusions de l'étude ou de la recherche.]
%
%   \begin{quote}
%     Mots clés: [Inscrire ici 5 à 10 mots clés]
%   \end{quote}
% \end{singlespace}
% \cleardoublepage
%
% \abstract
% \begin{singlespace}
%
%   [L'abstract doit être une traduction anglaise fidèle et grammaticalement correcte du résumé en français.]
%
%   \begin{quote}
%     Keywords: [Inscrire ici 5 à 10 mots clés]
%   \end{quote}
% \end{singlespace}
% \cleardoublepage

% ----------------------------------------------------------------------%
% 5- Table des matières.                                               %
% ----------------------------------------------------------------------%

\tabledesmatieres

% ----------------------------------------------------------------------%
% 6- Liste des tableaux.                                               %
% ----------------------------------------------------------------------%

% \listedestableaux

% ----------------------------------------------------------------------%
% 7- Table des matières.                                               %
% ----------------------------------------------------------------------%

% \listedesfigures

% ----------------------------------------------------------------------%
% 8- Liste des abréviations (optionnel).                               %
% ----------------------------------------------------------------------%

\listeabrev
\begin{liste}

\item[SDM] Modèle de distribution d'espèce (en référence terme anglais : \textit{Species Distribution Model})

\item[TIB] Théorie insulaire de la biogéographie

\item[TTIB] Théorie trophique de la biogéogrpahi des île (en référence terme anglais : \textit{Species Distribution Model})

\item[RCP]  Representative Concentration Pathway
\end{liste}

% ----------------------------------------------------------------------%
% 9- Liste des symboles (optionnel).                                   %
% ----------------------------------------------------------------------%

\listesymboles
\begin{liste}
\item[SYMBOLE 1] Ceci est la définition du symbole 1.

\item[SYMBOLE 2] Ceci est la définition du symbole 2.

\item[SYMBOLE 3] Ceci est la définition du symbole 3.
\end{liste}

% ----------------------------------------------------------------------%
% Fin des liminaires.                                                  %
% ----------------------------------------------------------------------%

\cleardoublepage
