A faire.
% [C’est dans cette section qu’est mise en évidence la portée de l’étude ainsi que les liens entre les articles ou autres textes et une ouverture sur les perspectives de recherche dans le domaine concerné. On y fait état des limites de la recherche et on y propose, le cas échéant, des pistes nouvelles pour de futures recherches ou des façons de développer de nouvelles applications. La conclusion ne doit pas présenter de nouveaux résultats ni de nouvelles interprétations. Elle doit être rédigée de manière à faire ressortir la cohérence de la démarche.]
