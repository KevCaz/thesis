\section{Introduction}\label{introduction}

Disentangling the respective contribution of processes shaping species
distribution remains the central tenet of biogeography. While
biogeographers clearly envision the list of ingredients that are needed
to understand species distribution \citep{Thuiller2013}, they are
currently lacking a recipes to mix them in the right proportions toward
forecasting community assembly. Therefore, in the current context of
global changes, we likely fail to predict accurately biodiversity
responses to global changes as we keep focusing on abiotic factors,
\emph{i.e} temperature and precipitations, and overlooking biotic
interactions and short-term evolutionary responses
{[}\citet{Lavergne2010}. At the core of this issue is the need for a
renewal of theoretical foundations of the field that could be started by
a synthesis of recent enrichments of the model of the island-based
theory of biogeography \citep{Lomolino2000a, Warren2015}.

Among main theoretical challenges toward more realistic models, biotic
interactions should be integrated as a constraint for species
co-existence in meta-communities. It is a truism of community ecology
that species interact and that their persistence in communities rely on
these relationships. Consequently, ecological interactions may explain,
at least partially, the dynamics of local extinctions which in turn must
explain some properties of the geometrical shape of the ranges of
species \citep{Holt2009, Cazelles2015b} even if we have poor evidence of
such effect at large spatial scales \citep[but see][]{Gotelli2010}.
However, two of the most influential models in biogeography assume
ecological equivalence of species. Indeed, the Theory of Island
Biogeography of MacArthur and Wilson \citep[hereafter
TIB,][]{MacArthur1967} focuses on the species richness on islands
according to properties of islands and overlook the variation among
species characteristics, especially their interactions within food-webs.
Second, in his neutral theory Hubbell assumes that individuals of
different species are ecologically equivalent and predicts the
distribution of abundance without considering interactions
\citep{Hubbell1997}. These two theoretical models have been proved
relevant for certain groups of species and inadequate for others, but
none of them was indented to describe exhaustively the different
components of communities on islands. To take leap toward predictions at
the community scale, the hypothesis of ecological equivalence must be
released and ecological interactions explicitly integrated
\citep{Holt2010}.

The TIB is well-suited to explore the consequences of the integration of
ecological intearction at broad spatial scale as it includes two
fundamental processes of biogeography, namely immigration and
extinction, while being very simple and readily expandable
\citep{Losos2010, Warren2015}. Building upon the classical models,
recent studies have included interactions in the TIB
\citep{Gravel2011, Cazelles2015a}. In such approaches, the pool of
species that can colonize the island becomes a metaweb which makes
species interdepend entities with specificities (\emph{e.g.} a given
trophic level) rather than indistinguishable unities of a species
quantity. As a important consequence of such consideration, colonization
and extinction rates vary with respect to the species identity and the
local community. Five years ago, \citet{Gravel2011} propose that
predators can locally survive as long as they find at least one prey and
prevent their colonization if the island is prey-free. More generally,
\citet{Cazelles2015b} presented a Lotka-Volterra like model in which the
composition of the local community determines the extinction rates.
Although extinctions are impacted by the local community, these
approaches overlook where energy s limited, the consumption of any
populations natter and for a predator, preys are not equivalent. Here,
we propose to extent the TIB with the explicit integration of energy as
a fundamental driver of extinction in local ecological network.

Species do not escape from thermodynamics laws that have shaped the
pyramid of life we currently observe \citep{Trebilco2013}. As a dramatic
evidence, despite the complexity of evolutionary trajectories, many of
key ecological properties of species scale with body mass
\citep{Woodward2005a} founding the metabolic theory of ecology
\citep{Brown2004}. Among the major results of this theory is the scaling
of the metabolic rates \citep{Gillooly2001} than often scale with the
power function of the body mass often between 2/3 and 3/4
\citep{White2013}. Even if all the relationships are no well-understood
\citep[see the case of abundances reviewed in][ and the recent
relationship between prey and predator biomasses
\citet{Hatton2015}]{White2007}, the commonness of allometric
relationships promises to lower the complexity of ecosystems by using
body mass distribution to describe many of its properties. Allometric
relationships and energy flows are also a way to revisit models of
populations dynamics as envisioned by \citet{Yodzis1992} who considered
species as energy processors to derive a bioenergetic model of
population dynamics where energy uptake is based on allometric
relationships. Recent development have convincingly shown that
allometric is a key to analysis the properties of networks \citep[such
as stability][]{Brose2006}, the role of species within
\citep{Schneider2012} and sone authors have even proposed to infer
ecological interactions with promising results
\citep{Gravel2013, Petchey2008}.

At large spatial scales, variations of energy availability explain the
latitudinal gradient of species richness. As a first approximation,
energy availability could be estimated by the productivity of primary
producers which relies on temperature, water availability and
stoichiometry \citep{Ott2014}. Several mechanisms have been proposed to
explain the positive relationship between energy and species diversity
\citep[see][ for a review]{Evans2005} among which two involve a change
in the breadth of ecological niche of the width of the ecological space.
Recently, based on the analysis of 196 empirical food webs,
\citet{Cirtwill2015a} have shown that the link density remains constant
over a latitude gradient for most ecosystems supporting that the more
the energy the wider the niche space rather than an increase of niche
width poleward. The Ecological Limits hypothesis posit Interstently
\citet{Rabosky2015}. Turnover of species \ldots{} gives credit
fluctuation and ecological debts . Damuth showed based that the well
correlation of can be is a there is a share we propose here taht such
idea and porpose to see consequence in term of biomass
allocation\ldots{} o This hypothesis the functnnubd evolution saturated
the space and imprint such hypothesis is important and island where
turnover forced while is not here we propose to look at that !
\citet{Hurlbert2014} predicted the signal of as expected under the
constraint when energy is limiting and as an indirect effects notably
their example. ..

Ability of local population to cover their energy demand is of
fundamental importance and when they cannot they must certainly go
locally extinct. On the other aid du the nece of a popluation a
locallity caoont sustain Evidence of the relative constance of the niche
breadth idirectly support the idea of a This recent results hypthesis
partially supoprt the idea that as suggested by in the TIB TIB
equilibrium dynamic where energy constrain total diversity but also we
assume MVP to derive energy uptake pf species and constrain species
richness. We also\ldots{} energy is a good rationale for extinction. As
remembered by \citet{Gravel2011a} average per capita growth rates of all
coexisting species must be positive at low density and string enough to
overcome negative random effects. Our model : -The proposed conceptual
framework that links: - extinction dynamics - ecological networks -
energetic constrains allometry - simple model constrains - simple
calculation - more than probability of extinction - highlight turover
with degree

We show that : - dynamics - species - turnover increase with max

Our model sheds light upon and is so hopefull\ldots{} - species turnover
- energy effect - population fluctuations

\section{Model}\label{model}

The model developed by MacArthur and Wilson in their TIB is a
colonization and extinction dynamics linked to island characteristics,
predicting the species richness on the island according to island size
and its distance from the mainland \citep{MacArthur1967}. One promising
direction to extend the theory of biogeography is to include ecological
interactions into the classical model
\citep{Holt2010, Gravel2011, Cazelles2015a}. Following this avenue, we
considered explicit interactions among the pool of species based on
allometric relationships. Body masses of species are furthermore
exploited to determine the quantity of energy a species requires to
maintain a local population on the island. Once the island cannot
sustain a minimal population for all local species, then extinctions
occur. Therefore, in the model described below, we extent the TIB with
purely stochastic colonization event together with deterministic
extinction based on a energy rational.

\subsection{Primary producers and habitat
heterogeneity}\label{primary-producers-and-habitat-heterogeneity}

An island is assumed to be a patch of land covered by a maximal quantity
of primary producers constituting an amount of energy available upon
which a food web can be built. The maximal amount of energy available
for herbivores is noted \(E_0\) and varies with island area, which is
basically the assumption behind the drop of extinction rates with an
increase in the size of the island in the classical theory
\citep{MacArthur1967, Rabosky2015}. In our model, when \(E_0\)
increases, irrespective of the nature of primary producers, the energy
available to sustain herbivore populations raises. We further postulate
that the habitat heterogeneity increases with the partitioning of
\(E_0\) in a vector of different categories of primary producers. Hence,
for \(p\) different habitats, the vector \((E_1, E_2, ..., E_p)\)
represent the energy partitioning among species such that
\(\sum_i E_i=E_0\). The simplification made here is twofold: (1) the
diversity of primary producers is not taken into account, (2) the
production is constant over the time.

\subsection{Metawebs}\label{metawebs}

The regional pool of species in the TIB is basically a number of species
\(P\) reflecting the regional diversity. Here, we not only consider a
fixed number of species \(P\) but we also include trophic relationships
among species. Following \citet{Cazelles2015a}, we built regional
metawebs of \(P\) species using the niche model \citet{Williams2000}. We
furthermore posit the niche axis as the body size species of species and
also species without any links in the metaweb are assumed to be
herbivores. Primary producers are not included in the niche model, and
species without link are assumed to be herbivores. Apart from primary
producers regarded as a quantity of available energy, the model we use
exhibits strong correlations between trophic level and body mass:
herbivores are often the smallest and top predators. Although this
assumption is an oversimplification for ecosystems in general, it
remains reasonable for marine ecosystems \citep{Trebilco2013} in which
such allometric relationship have been used to infer the structure of
food webs \citep{Gravel2013}.

\subsection{Migration of species from the
metaweb}\label{migration-of-species-from-the-metaweb}

An island is made of one or more habitats made of primary producers.
Here, we focus on the migration of herbivores and species of higher
trophic levels. Following \citet{Gravel2011}, we assume that the
colonization of herbivores/predators is successful only if they find at
least one of their habitat/prey on the island. Moreover, in the model we
propose, arrivals of new species are always possible until the energetic
requirements to maintain local populations exceed the primary production
in which case energetic constraints and networks topology determine the
identity of species to go extinct. Therefore, colonization events are
assumed to be purely stochastic, extinctions are more deterministic
\citep[this difference in stochastic nature between these fundamental
processes of biogeography has been recently supported
in][]{Cirtwill2015}.

\subsection{Energetic constraints on local food
webs}\label{energetic-constraints-on-local-food-webs}

The energetic rational of the model is simple: local populations need a
certain amount of energy to maintain a minimum of population under which
the species goes extinct. Under this constrain, the species richness
locally increases until the energy production is no longer sufficient
for all populations. For a given species \(i\), energy requirements of
any individual is derived from allometric relationships proposed by the
metabolic theory of ecology \citep{Brown2004}, that is a consumption of
the form \(c_im_i^b\) where \(b\) is often set to \(.75\). We do not
integrate variance among individual of a species, \(m_i\) is therefore a
constant for a species and the energy uptake associated to a minimal
viable population (hereafter MVP) of \(n_i\) individuals becomes
\(n_im_i^b\). \citet{Shaffer1981} defined the MVP as
``\emph{{[}\ldots{}{]} the smallest isolated population having a 99\%
chance of remaining extant for 1000 years despite the foreseeable
effects of demographic, environmental and genetic stochasticity, and
natural catastrophe}'', highlighting that the smaller the population,
the higher the extinction risk which is assessed by the time to
extinction. Building upon this idea, \citet{Lande1993} showed that the
time to extinction is also affected by the mean population growth rate
underlining that species characteristics may lead to a heterogeneity in
MVP. Moreover, \citet{Savage2004} have developed a metabolic framework
within which they proved the growth rate to be proportional to
\(m_i^{-b}\). Based on these results, we explore two simple cases: 1)
MVP is equal for all species: \(n_i=n_0\) ; 2) MVP scales with the
growth rate \(n_i=n_0m_i^{-b}\). For both scenarios, species \(i\) can
survive only if the energy expenditures can be covered, \emph{i.e} if
the energy available is greater than: \(n_ic_im_i^b\).

\subsection{Energy fluxes and transfer
efficiency}\label{energy-fluxes-and-transfer-efficiency}

Although the expression of energy consumption is similar among species,
they obtain it from different sources: herbivores feed on primary
producers, whereas predators feed on a set of preys. The primary
production must therefore be split properly through the entire
community. To deal with this, we propose to convert the energy costs for
maintaining predator populations into additional populations of
herbivores to be maintained. To exemplify this idea, we start with the
simplest trophic network where a predator \(j\) feeds upon a herbivore
\(i\). The cost to maintain the MVP of \(i\) is \(n_ic_im_i^b\) and
\(n_jc_jm_j^b\) for \(j\). For the latter, we convert \(n_j\) into an
extra population of \(i\), \(n_{i,j}\) herbivore individuals dedicated
to \(j\) consumption. Furthermore, we account for the energy loss the
conversion begets by including a transfer efficiency \(\tau\). Hence,
the conversion from \(n_{j}\) to \(n_{i,j}\) is given by the following
equation:

\[ \tau n_{i,j} c_im_i^b = n_jc_jm_j^b \] \{\#eq:id1\}

which yields:

\[ n_{i,j} = \frac{n_jc_j}{\tau c_i} \left( \frac{m_j}{m_i} \right)^b \]
\{\#eq:id2\}

In our study, we postulate that the transfer efficiency is constant
across trophic levels, which is likely an oversimplification of the
reality as suggested by the sparse empirical data available
\citep{Trebilco2013, Brown2003}. We now add a new predator \(k\) feeding
on \(j\) to the insular food web. According to our reasoning, we must
covert \(n_k\) into \(n_{i,k}\). To do so, we start by converting
\(n_k\) into a population of \(j\):

\[ n_{j,k} = \frac{n_kc_k}{\tau c_j} \left( \frac{m_k}{m_j} \right)^b \]
\{\#eq:id2\}

We now turn \(n_{j,k}\) into a herbivore population:

\[ n_{i,k} = \frac{\frac{n_kc_k}{\tau c_j} \left( \frac{m_k}{m_j} \right)^bc_j}{\tau c_i} \left( \frac{m_j}{m_i} \right)^b \]
\{\#eq:id3\}

which gives:

\[ n_{i,k} = \frac{n_kc_k}{\tau^2 c_i} \left( \frac{m_k}{m_i} \right)^b \]
\{\#eq:id3b\}

In a similar fashion, for a linear trophic chain, we can demonstrate
that the additional population of herbivore \(i\) to be produced to
maintain predator \(j\) of level \(l\) is:

\[ n_{i,k} = \frac{n_jc_j}{\tau^l c_i} \left( \frac{m_j}{m_i} \right)^b \]
\{\#eq:id4\}

In many cases, a predator feeds on a set of prey rather than a single
one. In such case, the energy uptake should be shared by the different
sources. Basically, the split of energy is the realm of populations
dynamics as the population consumption depends on individuals number. To
overcome the complexity such consideration would bring, we assume that
energy costs associated with the local maintaining of a predator are
minimal. Therefore, \(n_j\) is converted into \(i\), the herbivore
linked to \(j\) for which (\citet{eq:id4}) is minimal. Basically, \(i\)
should be a large and separated from \(j\) by a low number of species.
Hence, on the island, the species richness increases as long as the
inequality below holds true:

\[ \sum_i c_in_im_i^b + \sum_j \frac{n_jc_j}{\tau^{l_j} c_i} \left( \frac{m_j}{m_{i_j}} \right)^b < E_0 \]
\{\#eq:id5\}

The first term is the energy cost for maintaining populations of
herbivores, the second term is associated to higher trophic levels:
predator \(j\) is converted into herbivore \(i_j\) from which it is
separated by \(l_j\) links. For the sake of simplicity, we make an extra
assumptions: \(c_i\) values are constant among species and set to 1.
Therefore is we assume that MVP is constant (scenario 1), inequality
\citet{eq:id5} becomes:

\[ \sum_i m_i^b + \sum_j \frac{1}{\tau^{l_j}} \left( \frac{m_j}{m_{i_j}} \right)^b< \frac{E_0}{n_0} \]
\{\#eq:id5a\}

Similarly, if we assume \(n_i=n_0m_i^{-b}\) and \(n_j=n_0m_j^{-b}\)
(scenario 2), then:

\[ \sum_i 1 + \sum_j \frac{1}{\tau^{l_j}} \left( \frac{1}{m_{i_j}} \right)^b< \frac{E_0}{n_0} \]
\{\#eq:id5b\}

The left side of this equation provides the minimal energy needed to
sustain all populations while the right side is the total amount of
energy available. Therefore, the maximal population of species \(i\)
without any extinction is given by converting the extra amount of energy
available into an additional population of species \(i\), the population
of \(i\) we get is denote \(n_{i,max}\). The range \([n_i, n_{i, max}]\)
is then the range of possible fluctuations of species \(i\) without any
new extinction event.

\subsection{Extinctions}\label{extinctions}

When the local primary production cannot sustain the establishment of a
new immigrant, \emph{i.e.} when inequality \citet{eq:id5} is no longer
verified, its arrival is either impossible or lead to extinction of
other species. In the latter situation, we must determine the identity
of species to go extinct. This is a challenge that remains we will not
take up here especially given its complexity highlighted by recent
theoretical studies \citep{Saterberg2013, Zhao2016}. To overcome this
difficulty, we use two scenarios: 1) \emph{random extinction}: any
species can go extinct and 2) \emph{costs-based extinctions}: the
probability of extinction is proportional to the energetic costs of the
species. Once a extinction happened we ensure that all species remained
link to at least on herbivore, otherwise unlinked species go extinct.
Furthermore extinctions occur until \citet{eq:id5} is satisfied.

Although the assumptions we have made here are likely unrealistic, they
all intend to examine the model under the special case of an optimal
energy allocation on island. As long as we focus on the qualitative
consequences in term of community dynamics, these assumption remain
acceptable. As an important remark, in our model the energy cost
associated to one predator is not an inherent properties as it is
determined by the identity of preys available on the island.

\subsection{Simulations}\label{simulations}

\section{Results}\label{results}
