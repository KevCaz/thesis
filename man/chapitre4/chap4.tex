\chapter{Towards a metabolic theory of biogeography}
\label{chap4}

\subsection{Résumé en français du troisième article}

Cet article est à la fois une présenatation d'une xtension durect du modèle de TTTIb et d'un cas particulier de mon premie chapitre mais aussi
le preneir pas vers un espoir motiver par plusierus chsoes. En 2015 j'ai comencà à m'interessé au problème energie mais pa assez et puis j'ai commencé
des reflexiosn pour lever le nombre d'espèce dans mon modèle mais aussi à essayer d'aller vers qulqueauchose qui est plus réalistes.
Quand on pense à une pyramide du viavant uil y a la base des producitoeru qui de l'energie inogranique jusqu'au top prédateurs sont capables de faire de l'énergie.
COmme souligné par le pepier de gravel et al. il n'y a le simple fat qu'un prédateur mais aussi il y a une différenece de triantemnt dansun top prédateur et un herbivore dans la rpartitoin de l'énergie. J'ai essayeé de me vconfrointer à ces porblèmes mais j'ai du commencé par comprendre l'.tat du svaoir et les enjeux. Il m'apparait que 2 théories int.ressantes plus ou moins mécanistqieu amais pas tant de pronlèeme. Mais comment comprendre que certain relation soit visibles à lare échelle est difficile à comprendre ou sont la preuve que certaine hypothèse nitamen la saturation..
Ce modèle offre des perspectives nouvelles et un point d'ancrage concret sur plein de belles choses.

\subsection{Publication envisagée}

Ce article traite de deux ascpect les perspectives qu'offre un approche et propose un proeneir pas pour sclaer ver sles ppulations depuis le haut.
Le modèle qui y est présenté est simple mais innovant et offre possibilité pour explorer des hypothèses.
Le chapitre est en cours de dévelpppement. L'avancement est indiqué dans le rapport du troisième compte-rendu de comité de thèse.
Il est la base d'un porjet que je viens de proposer pour le post doc.
Miguel Araujo et Loic Pelissier. cool
\section{Introduction}\label{introduction}

Disentangling the respective contribution of processes shaping species
distribution remains the central tenet of biogeography. While
biogeographers clearly envision the list of ingredients that are needed
to understand species distribution \citep{Thuiller2013}, they are
currently lacking a recipes to mix them in the right proportions toward
forecasting community assembly. Therefore, in the current context of
global changes, we likely fail to predict accurately biodiversity
responses to global changes as we keep focusing on abiotic factors,
\emph{i.e} temperature and precipitations, and overlooking biotic
interactions and short-term evolutionary responses \citep{Lavergne2010}.
At the core of this issue is the need for a renewal of theoretical
foundations of the field that could be started by a synthesis of recent
enrichments of the model of the island-based theory of biogeography
\citep{Lomolino2000a, Warren2015}.

Among major theoretical challenges toward more realistic models, biotic
interactions should be integrated as a constraint for species
co-existence in meta-communities. It is a truism of community ecology
that species interact and that their persistence in communities rely on
these relationships. Consequently, ecological interactions may explain,
at least partially, the dynamics of local extinctions which in turn must
explain some properties of the geometrical shape of the ranges of
species \citep{Holt2009, Cazelles2015a} even if we have poor evidence of
such effect at large spatial scales \citep[but see][]{Gotelli2010}.
However, two of the most influential models in biogeography assume
ecological equivalence of species. Indeed, the Theory of Island
Biogeography of MacArthur and Wilson \citep[hereafter
TIB,][]{MacArthur1967} focuses on the species richness on islands
according to properties of islands and overlook the variation among
species characteristics, especially their interactions within food-webs.
Second, in his neutral theory, Hubbell assumes that individuals of
different species are ecologically equivalent and predicts the
distribution of abundance without considering interactions
\citep{Hubbell1997}. These two theoretical models have been proved
relevant for certain groups of species and inadequate for others, but
none of them was indented to describe exhaustively the different
components of communities on islands. To take a stride toward
predictions at the community scale, the hypothesis of ecological
equivalence must be released and ecological interactions explicitly
integrated \citep{Holt2010, Gravel2011}.

The TIB is well-suited to explore the consequences of the integration of
ecological interaction at broad spatial scale as it includes two
fundamental processes of biogeography, namely immigration and
extinction, while being very simple and readily expandable
\citep{Losos2010, Warren2015}. Building upon the classical models,
recent studies have included interactions in the TIB
\citep{Gravel2011, Cazelles2015a}. In such approaches, the regional
becomes a metaweb which makes species interdepend entities with
specificities (\emph{e.g.} a given trophic level) rather than
indistinguishable unities of a species quantity. As a important
consequence of such consideration, colonization and extinction rates
vary with respect to the species identity and the local community. Five
years ago, \citet{Gravel2011} propose the Trophic TIB (hereinafter TTIB)
that predators locally survive as long as they find at least one prey
and prevent their colonization if for prey-free islands. More generally,
\citet{Cazelles2015a} presented a Lotka-Volterra like model in which the
composition of the local community determines the extinction rates.
Despite the increase of the realism of the mode, we must acknowledge
that adding new ecological processes in the TIB affects its simplicity
whereas the quality of its predictions has rarely be proven better
\citep[see][]{Cirtwill2015}. Extending the TIB while preserving its
elegancy is therefore a challenging and technical issue. As a promising
avenue to find answers, we propose to reformulate the TTIB in terms of
energy constrains.

Species do not escape from thermodynamics laws that have shaped the
pyramid of life we currently observe \citep{Trebilco2013}. As a dramatic
evidence, despite the complexity of evolutionary trajectories, many of
key ecological properties of species scale with body mass
\citep{Woodward2005a} founding the metabolic theory of ecology
\citep{Brown2004}. Among the major results of this theory is the scaling
of the metabolic rates \citep{Gillooly2001} than often scale with the
power function of the body mass often between 2/3 and 3/4
\citep{White2013}. Even if all the relationships are no well-understood
\citep[see the case of abundances reviewed in][ and the recent
relationship between prey and predator biomasses
\citet{Hatton2015}]{White2007}, the commonness of allometric
relationships promises to lower the complexity of ecosystems by using
body mass distribution to describe many of its properties. Allometric
relationships and energy flows are also a way to revisit models of
populations dynamics as envisioned by \citet{Yodzis1992} who considered
species as energy processors to derive a bioenergetic model of
population dynamics where energy uptake is based on allometric
relationships. Recent developments have convincingly shown that
allometric is a key to analysis the properties of networks \citep[such
as stability][]{Brose2006}, the role of species within
\citep{Schneider2012} and sone authors have even proposed to infer
ecological interactions with promising results
\citep{Gravel2013, Petchey2008}.

At large spatial scales, the diminution of solar energy availability
from lower to higher latitudes explain one of the most obvious pattern
in biogeography : the latitudinal gradient of species richness
\citep{Rhode1992, Stevens1989, Evans2005}. The productivity of primary
producers is a good predictor for species richness
\citep{Evans2005, Storch2005} and is strongly correlated with energy and
water ability, making the climate a strong descriptor of species
richness over large spatial extents \citep{Hawkins2003}. Several
mechanisms have been proposed to explain the positive relationship
between energy and species diversity \citep[see][ for a
review]{Evans2005} among which two involve a change in the breadth of
ecological niche of the width of the ecological space. Recently, based
on the analysis of 196 empirical food webs, \citet{Cirtwill2015a} have
shown that the link density remains constant over a latitude gradient
for most ecosystems supporting that the more the energy the wider the
niche space rather than an increase of niche width poleward. Answers
must therefore be found in the light of energetic constraints and the
structure of food webs. In 1983, \citet{Wright1983} developed the
species-energy theory and replaced the ``area'' with ``available
energy'' to derive a meaningful Species Energy Relationship (SER).
Although area and energy may bring more information taken together
\citep{Storch2005}, the rational behind allows the derivation of species
abundance and occurrence probability based on energetic constraint
\citep{Wright1983}.

Here, we propose to rebuild the model of the TTIB following the vision
of \citet{Wright1983}. To do so, we build a theoretical model where
islands are patches of primary producers determining the amount of
energy available, upon which local network may be build by successive
colonization form the regional metaweb. Body masses of species are used
to derive the need to sustain a minimal population on the island. As
species from different trophic level are considered, we use a transfer
efficiency to integrate discrepancies of energetic costs caused by the
difference of energy sources. Based on the model we were able to derive
a SER based on the dynamics of colonization and extinction of
communities

\section{Model}\label{model}

The model developed by MacArthur and Wilson in their TIB is a
colonization and extinction dynamics linked to island characteristics,
predicting the species richness on the island according to island size
and its distance from the mainland \citep{MacArthur1967}. One promising
direction to extend the theory of biogeography is to include ecological
interactions into the classical model
\citep{Holt2010, Gravel2011, Cazelles2015a}. Following this avenue, we
considered explicit interactions among the pool of species based on
allometric relationships. Body masses of species are furthermore
exploited to determine the quantity of energy a species requires to
maintain a local population on the island. Once the island cannot
sustain a minimal population for all local species, then extinctions
occur. Therefore, in the model described below, we extent the TIB with
purely stochastic colonization event together with deterministic
extinction based on a energy rational.

\subsection{Primary producers and habitat
heterogeneity}\label{primary-producers-and-habitat-heterogeneity}

An island is assumed to be a patch of land covered by a maximal quantity
of primary producers constituting an amount of energy available upon
which a food web can be built. The maximal amount of energy available
for herbivores is noted \(E_0\) and varies with island area, which is
basically the assumption behind the drop of extinction rates with an
increase in the size of the island in the classical theory
\citep{MacArthur1967, Rabosky2015}. In our model, when \(E_0\)
increases, irrespective of the nature of primary producers, the energy
available to sustain herbivore populations raises. The simplification
made here is twofold: (1) the diversity of primary producers is not
taken into account, (2) the production is constant over the time.

\subsection{Metawebs}\label{metawebs}

The regional pool of species in the TIB is basically a number of species
\(P\) reflecting the regional diversity. Here, we not only consider a
fixed number of species \(P\) but we also include trophic relationships
among species. Following \citet{Cazelles2015a}, we built regional
metawebs of \(P\) species using the niche model \citet{Williams2000}. We
furthermore posit the niche axis as the body size species of species and
also species without any links in the metaweb are assumed to be
herbivores. Primary producers are not included in the niche model, and
species without link are assumed to be herbivores. Apart from primary
producers regarded as a quantity of available energy, the model we use
exhibits strong correlations between trophic level and body mass:
herbivores are often the smallest and top predators. Although this
assumption is an oversimplification for ecosystems in general, it
remains reasonable for marine ecosystems \citep{Trebilco2013} in which
such allometric relationship have been used to infer the structure of
food webs \citep{Gravel2013}.

\subsection{Migration of species from the
metaweb}\label{migration-of-species-from-the-metaweb}

An island is made of one or more habitats made of primary producers.
Here, we focus on the migration of herbivores and species of higher
trophic levels. Following \citet{Gravel2011}, we assume that the
colonization of herbivores/predators is successful only if they find at
least one of their habitat/prey on the island. Moreover, in the model we
propose, arrivals of new species are always possible until the energetic
requirements to maintain local populations exceed the primary production
in which case energetic constraints and networks topology determine the
identity of species to go extinct. Therefore, colonization events are
assumed to be purely stochastic, extinctions are more deterministic
\citep[this difference in stochastic nature between these fundamental
processes of biogeography has been recently supported
in][]{Cirtwill2015}.

\subsection{Energetic constraints on local food
webs}\label{energetic-constraints-on-local-food-webs}

The energetic rational of the model is simple: local populations need a
certain amount of energy to maintain a minimum of population under which
the species goes extinct. Under this constrain, the species richness
locally increases until the energy production is no longer sufficient
for all populations. For a given species \(i\), energy requirements of
any individual is derived from allometric relationships proposed by the
metabolic theory of ecology \citep{Brown2004}, that is a consumption of
the form \(c_im_i^b\) where \(b\) is often set to \(.75\). We do not
integrate variance among individual of a species, \(m_i\) is therefore a
constant for a species and the energy uptake associated to a minimal
viable population (hereafter MVP) of \(n_i\) individuals becomes
\(n_im_i^b\). \citet{Shaffer1981} defined the MVP as
``\emph{{[}\ldots{}{]} the smallest isolated population having a 99\%
chance of remaining extant for 1000 years despite the foreseeable
effects of demographic, environmental and genetic stochasticity, and
natural catastrophe}'', highlighting that the smaller the population,
the higher the extinction risk which is assessed by the time to
extinction. Building upon this idea, \citet{Lande1993} showed that the
time to extinction is also affected by the mean population growth rate
underlining that species characteristics may lead to a heterogeneity in
MVP. Moreover, \citet{Savage2004} have developed a metabolic framework
within which they proved the growth rate to be proportional to
\(m_i^{-b}\). Based on these results, we explore two simple cases: 1)
MVP is equal for all species: \(n_i=n_0\) ; 2) MVP scales with the
growth rate \(n_i=n_0m_i^{-b}\). For both scenarios, species \(i\) can
survive only if the energy expenditures can be covered, \emph{i.e} if
the energy available is greater than: \(n_ic_im_i^b\).

\subsection{Energy fluxes and transfer
efficiency}\label{energy-fluxes-and-transfer-efficiency}

Although the expression of energy consumption is similar among species,
they obtain it from different sources: herbivores feed on primary
producers, whereas predators feed on a set of preys. The primary
production must therefore be split properly through the entire
community. To deal with this, we propose to convert the energy costs for
maintaining predator populations into additional populations of
herbivores to be maintained. To exemplify this idea, we start with the
simplest trophic network where a predator \(j\) feeds upon a herbivore
\(i\). The cost to maintain the MVP of \(i\) is \(n_ic_im_i^b\) and
\(n_jc_jm_j^b\) for \(j\). For the latter, we convert \(n_j\) into an
extra population of \(i\), \(n_{i,j}\) herbivore individuals dedicated
to \(j\) consumption. Furthermore, we account for the energy loss the
conversion begets by including a transfer efficiency \(\tau\). Hence,
the conversion from \(n_{j}\) to \(n_{i,j}\) is given by the following
equation:

\begin{equation} \tau n_{i,j} c_im_i^b = n_jc_jm_j^b \label{eq:id1}\end{equation}

which yields:

\begin{equation} n_{i,j} = \frac{n_jc_j}{\tau c_i} \left( \frac{m_j}{m_i} \right)^b \label{eq:id2}\end{equation}

In our study, we postulate that the transfer efficiency is constant
across trophic levels, which is likely an oversimplification of the
reality as suggested by the sparse empirical data available
\citep{Trebilco2013, Brown2003}. We now add a new predator \(k\) feeding
on \(j\) to the insular food web. According to our reasoning, we must
covert \(n_k\) into \(n_{i,k}\). To do so, we start by converting
\(n_k\) into a population of \(j\):

\begin{equation} n_{j,k} = \frac{n_kc_k}{\tau c_j} \left( \frac{m_k}{m_j} \right)^b \label{eq:id2}\end{equation}

We now turn \(n_{j,k}\) into a herbivore population:

\begin{equation} n_{i,k} = \frac{\frac{n_kc_k}{\tau c_j} \left( \frac{m_k}{m_j} \right)^bc_j}{\tau c_i} \left( \frac{m_j}{m_i} \right)^b \label{eq:id3}\end{equation}

which gives:

\begin{equation} n_{i,k} = \frac{n_kc_k}{\tau^2 c_i} \left( \frac{m_k}{m_i} \right)^b \label{eq:id3b}\end{equation}

In a similar fashion, for a linear trophic chain, we can demonstrate
that the additional population of herbivore \(i\) to be produced to
maintain predator \(j\) of level \(l\) is:

\begin{equation} n_{i,k} = \frac{n_jc_j}{\tau^l c_i} \left( \frac{m_j}{m_i} \right)^b \label{eq:id4}\end{equation}

In many cases, a predator feeds on a set of prey rather than a single
one. In such case, the energy uptake should be shared by the different
sources. Basically, the split of energy is the realm of populations
dynamics as the population consumption depends on individuals number. To
overcome the complexity such consideration would bring, we assume that
energy costs associated with the local maintaining of a predator are
minimal. Therefore, \(n_j\) is converted into \(i\), the herbivore
linked to \(j\) for which (\ref{eq:id4}) is minimal. Basically, \(i\)
should be a large and separated from \(j\) by a low number of species.
Hence, on the island, the species richness increases as long as the
inequality below holds true:

\begin{equation} \sum_i c_in_im_i^b + \sum_j \frac{n_jc_j}{\tau^{l_j} c_i} \left( \frac{m_j}{m_{i_j}} \right)^b < E_0 \label{eq:id5}\end{equation}

The first term is the energy cost for maintaining populations of
herbivores, the second term is associated to higher trophic levels:
predator \(j\) is converted into herbivore \(i_j\) from which it is
separated by \(l_j\) links. For the sake of simplicity, we make an extra
assumptions: \(c_i\) values are constant among species and set to 1.
Therefore is we assume that MVP is constant (scenario 1), inequality
\ref{eq:id5} becomes:

\begin{equation} \sum_i m_i^b + \sum_j \frac{1}{\tau^{l_j}} \left( \frac{m_j}{m_{i_j}} \right)^b< \frac{E_0}{n_0} \label{eq:id5a}\end{equation}

Similarly, if we assume \(n_i=n_0m_i^{-b}\) and \(n_j=n_0m_j^{-b}\)
(scenario 2), then:

\begin{equation} \sum_i 1 + \sum_j \frac{1}{\tau^{l_j}} \left( \frac{1}{m_{i_j}} \right)^b< \frac{E_0}{n_0} \label{eq:id5b}\end{equation}

The left side of this equation provides the minimal energy needed to
sustain all populations while the right side is the total amount of
energy available. Therefore, the maximal population of species \(i\)
without any extinction is given by converting the extra amount of energy
available into an additional population of species \(i\), the population
of \(i\) we get is denote \(n_{i,max}\). The range \([n_i, n_{i, max}]\)
is then the range of possible fluctuations of species \(i\) without any
new extinction event.

\subsection{Extinctions}\label{extinctions}

When the local primary production cannot sustain the establishment of a
new immigrant, \emph{i.e.} when inequality \ref{eq:id5} is no longer
verified, its arrival is either impossible or lead to extinction of
other species. In the latter situation, we must determine the identity
of species to go extinct. This is a challenge that remains we will not
take up here especially given its complexity highlighted by recent
theoretical studies \citep{Saterberg2013, Zhao2016}. To overcome this
difficulty, we use two scenarios: 1) \emph{random extinction}: any
species can go extinct and 2) \emph{costs-based extinctions}: the
probability of extinction is proportional to the energetic costs of the
species. Once a extinction happened we ensure that all species remained
link to at least on herbivore, otherwise unlinked species go extinct.
Furthermore extinctions occur until \ref{eq:id5} is satisfied.

Although the assumptions we have made here are likely unrealistic, they
all intend to examine the model under the special case of an optimal
energy allocation on island. As long as we focus on the qualitative
consequences in term of community dynamics, these assumption remain
acceptable. As an important remark, in our model, the energy cost
associated to one predator is not an inherent properties as it is
determined by the identity of preys available on the island.

\subsection{Simulations}\label{simulations}

Along a gradient of energy ranging in \(\frac{E_0}{n_0}\) from \(.1\) to
\(10^8\) we simulated the model described above. We start with an empty
island which and at any time step, any species of the metaweb has a
probability \(c=0.001\) of colonizing the island. If a colonization
occurs then it is successful if the specie is a herbivore or the species
is a predator. If the colonization is a success then we derive whether
the energy available on the island allow a MVP of the new species to be
maintained locally. If it is indeed possible then the species becomes a
part of the insular network, otherwise extinction process is triggered.
For all time step we record the species on the island. We perform
125,000 iterations and discard 25,000 burn-in iterations and the 100,000
remaining to do our analyses. For all the 91 values of the energy
gradient we used 100 replicates, meaning 100 different networks
generated using the niche model with a connectance of \(0.05\) and we
force the network to generate ten herbivores (non connected species).

\begin{longtable}[]{@{}lll@{}}
\caption{Hypotheses associated to the four scenarios.}\tabularnewline
\toprule
Scenario & MVP & extinction\tabularnewline
\midrule
\endfirsthead
\toprule
Scenario & MVP & extinction\tabularnewline
\midrule
\endhead
1 & \(n_0\) & random\tabularnewline
2 & \(n_0\) & costs-based\tabularnewline
3 & \(n_0m^{-.75}\) & random\tabularnewline
4 & \(n_0m^{-.75}\) & costs-based\tabularnewline
\bottomrule
\end{longtable}

\subsection{Results and discussions}\label{results-and-discussions}

p1 Porvide a SRER and difference among scenario not a impcat but in
population it dies. Interestinglt the variance attached to cost based
are lower. This highlih that constrainlig less than expected in approach
where energetic constrains are taken into account that TTIB, and more
determined thancolonisation \citep{Cirtwill2015}.

Linearisation this also linear \citep{Wright1983} is linear and this is
when energy liits a superior if this reulst were further confirmed it
awould give credits

Cost based and logica extinction

Illustrate that generalists species are first to colonize and and
specialist second this is a constant among our sceario. This is actually
during the classical biogegraphic stufy by Simerlof revealing once
reanalysed by \citep{Piechnik2008}. This is also an importnat when
looking on the other way when fragmentation favourize generalist
\citep{Simberloff1969b}

Link turover role in ecology trophic level and context depeends.

stpe forward The model described here is a step forward the integration
of more and ecology, As envision by wright there is a link with
population. However here mechnis are bottom-up usefull to described the
netedness of ranges \citep{Cirtwill2015} Here we provide simply a first
reus and the density mass is implicitley assue. BTop-down very important
We don include variation among mass\citep{Terborgh1971} the analysis
must allow to undersand \citep{Hatton2015}; productvity

Possibilities of the model We further postulate that the habitat
heterogeneity increases with the partitioning of \(E_0\) in a vector of
different categories of primary producers. Hence, for \(p\) different
habitats, the vector \((E_1, E_2, ..., E_p)\) represent the energy
partitioning among species such that \(\sum_i E_i=E_0\).

The Ecological Limits hypothesis posit Interstently \citet{Rabosky2015}.
Turnover of species \ldots{} gives credit fluctuation and ecological
debts . Damuth showed based that the well correlation of can be is a
there is a share we propose here taht such idea and porpose to see
consequence in term of biomass allocation\ldots{} o This hypothesis the
functnnubd evolution saturated the space and imprint such hypothesis is
important and island where turnover forced while is not here we propose
to look at that ! \citet{Hurlbert2014} predicted the signal of as
expected under the constraint when energy is limiting and as an indirect
effects notably their example. .. Theses question have been explored in
a evolutionnary perpsective
\citep{Damuth2007, Hurlbert2014, Rabosky2015}. In his approach Dauth
demonstrate that an equal sharing of energy can be obtained obtain some
intuive model whose ccpetion may be accepted in the lght of the Red
Queen hypothesis and therefore explain the relationship he reveails more
than two decade before \citet{Damuth1981}. Recentlty, Hulbert 2014 have
demonstrated that energetic constraining popultaion speciation rates and
left som impront on the three of life they evaluate on a gradient froma
A;aska to Baja California of rockfish and demonstare Rather we look at
the network it produces by a similar approach.

\subsubsection{Food web stability}\label{food-web-stability}

The only constraints is the feasibility Althought the incredible
stabiity and its importantce we disregrad this properties. Firste we
think that this would enhance our prodeicoom and provides more detailed
about the structure of the snetweork however it would chage physical
control

Storage effect

we focus on flow and so interaction is proportionnel to the flow not
fron the amount which make population inexcitable. As 90\% of oxygen
consumption is associated with ATP production \citet{White2013}. As in
\citet{Damuth2007} it serves ti calculate density at equiibrium but also
it also means that population dynamics and energy approach are easily
related through biomass. Note that in Damuth appro \(a_i\) are actually
subject to stocahstic approach. We are then in a zero-sum dynamics.
Metabolic rates of socomation.

\[\text{MR}_{i}=i_0M^{.75}e^{-E_a/kT}\]

See the review \citet{Post2002} It has been hypothesized that imperfect
energetic transfers limits trophic also constraint by population dynamic
the longer the chain the less stable \citet{Pimm1978} a evolutionary
concern \citet{hastings1979} contrast between degree of three
relationship with size In a defense for stability increasing by
diversity Mac Arthur hase envisionned a conceptual link between
population and increase food \citet{MacArthur1955}. There are classical
evidence for such flcutuation.
