\chapter{Vers une théorie métabolique de la biogéographie}
\label{chap4}


\subsection{Résumé en français du quatrième article}

Dans ce chapitre, je présente une formulation énergétique du modèle
de la TTIB \citep{Gravel2011}. Pour cela, je transforme les îles en
quantités de producteurs primaires. Les espèces sont liées entre elles
par des relations trophiques et le réseau est décrit pour l'ensemble des interactions,
il s'agit donc d'un metaweb. Les espèces de niveau supérieurs peuvent coloniser
une île tant que toutes les espèces alors présentes peuvent y maintenir une
population minimale locale. Je propose un calcul simple mais astucieux pour
intégrer les différences de consommation entre les différents niveaux trophiques.
Le calcul est basé sur la biomasse des espèces mais aussi sur le transfert énergétique,
qui représente la quantité d'énergie qui passant d'un niveau trophique à l'autre.
Lorsqu'une espèce arrive sur l'île mais que l'ensemble des espèces ne peuvent
plus toutes soutenir une population minimale localement, alors un processus
d'extinction est enclenché. À partir de cette nouvelle formulation de la
TTIB, j'obtiens une relation espèce-énergie (SER) pour laquelle la probabilité
d'occurrence des espèces en fonction de leur masse corporelle et
leur statut trophique est connue. Je montre alors l'importance du statut trophique
pour comprendre la probabilité d'occurrence et montre alors comment une relation
énergie-longeur de la chaine trophique se met en place.

Cette approche est un pas significative vers une intégration plus aboutie des
mécanismes écologiques dans la TIB, notamment pour les
interactions biotiques. Je prolonge le travail de la TTIB et
je fais un premier rapprochement avec la théorie métabolique de l'écologie.
Bien que le modèle proposé soit relativement simple, il est une pierre solide pour
aller plus loin dans la théorie. De plus, au regard de l'abondante littérature
relative aux relations allométriques, il me semble essentiel de comprendre
comment les contraintes énergétiques, exercées à l'échelle de l'individu,
se propagent à l’échelle des populations puis aux échelles biogéographiques.



\subsection{Publication envisagée}

Présentement, l'article offre des perspectives intéressantes sur une
théorie métabolique en biogéographie mais aussi des pistes de réflexion
pour s'affranchir simplement de certaines contraintes de modélisation
qui pourraient hautement complexifier le modèles (par exemple, je ne considère
pas explicitement de dynamiques de population). Je vois deux opportunités
associées au présent travail. Premièrement, orienter davantage mon travail
sur les aspects techniques et mathématiques du modèle et essayer,
par exemple, de dériver une solution approchée du modèle.
Cela m'orienterait vers une publication plus technique. Deuxièmement, aller p
lus loin dans l'exploration des possibilités offertes par le modèle, insister
sur les perspectives, notamment sur l'impact de la structure sur la relation
énergie-espèces. Cela m'orienterait plutôt vers
un article de perspective pour un journal d’écologie plus généraliste.

Pour cet article, ma réflexion a été alimentée par discussion avec Dominique Gravel,
Miguel Araújo, Loïc Pelissier que je remercie. Je me suis occupé de concevoir le modèle,
d'implémenter le modèle et de sortir les résultats et de l'analyser.
J'ai écrit la première version du modèle et j'ai bénéficié d'apport significatif
de Dominique Gravel qui a fait une relecture complète du manuscrit. Je remercie
également chaleureusement Kevin Solarik pour le temps qu'il m'a donné en
réalisant de nombreuses relectures et des suggestions très pertinentes.
Cet article est également la base de réflexion du travail que
j’aimerais mener dans un projet post-doctoral.


\subsection{Traduction du résumé en anglais}

L’énergie façonne la biodiversité; l’abondance de l’
énergie solaire et la disponibilité en eau décrivent bien les écosystèmes
terrestres de même que la température de surface de la mer et la disponibilité
en nutriments écrivent les écosystèmes marins. À l'échelle de la communauté,
les réseaux trophiques et leurs structures déterminent les échanges d'énergie
des producteurs primaires aux top prédateurs. En dépit de sa grande importance,
l'introduction de l'énergie dans les théories classiques en biogéographie a été
négligée, ces théories considèrent les espèces équivalentes (c'est le cas de la
Théorie de la Biogéographie des îles), ou alors se concentrent sur le concept de niche
écologique. Pour rectifier cela, nous développons une théorie pour inclure
les contraintes énergétiques pour décrire la répartition de la biodiversité.
Comme illustration des perspectives offertes par cette approche, nous étudions
la richesse spécifique et a structure des réseau le long d'un gradient
d'énergie. Avec l'augmentation de l'énergie disponible, nous obtenons une succession claire
des espèces selon leur statut trophique~: des plus bas niveaux trophiques sur
les îles pauvres en énergie jusqu'au plus haut niveaux sur les îles où l'énergie
est abondante. C'est une piste prometteuse pour aller au-delà de la prédiction quantitative
des pertes d’espèces lorsque les habitats sont fragmentés et prédire le rôle qu'elles
ont au sein du réseau écologique. Nous pensons que l'utilisation de ce modèle
sera très utile pour une intégration de l'écologie des communauté en biogéographie,
tout en permettant aussi une meilleure paramétrisation des modèles de distribution
d'espèces.



\emph{Les sections suivantes sont celles de l'article envisagé.}
