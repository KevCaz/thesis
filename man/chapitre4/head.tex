\chapter{Towards a metabolic theory of biogeography}
\label{chap4}


\subsection{Résumé en français du troisième article}

Dpeuis longtems cotrinate énergétqies
En 2015 j'ai comencà à m'interessé au problème energie mais pa assez et puis j'ai commencé

Cet article présente d'une extension du modèle de la TTIB \citep{Gravel2011}
que je proposed de reformuler en terme de contrainte énergétiques.


des reflexiosn pour lever le nombre d'espèce dans mon modèle mais aussi à essayer d'aller vers qulqueauchose qui est plus réalistes.
Quand on pense à une pyramide du viavant uil y a la base des producitoeru qui de l'energie inogranique jusqu'au top prédateurs sont capables de faire de l'énergie.
COmme souligné par le pepier de gravel et al. il n'y a le simple fat qu'un prédateur mais aussi il y a une différenece de triantemnt dansun top prédateur et un herbivore dans la rpartitoin de l'énergie. J'ai essayeé de me vconfrointer à ces porblèmes mais j'ai du commencé par comprendre l'.tat du svaoir et les enjeux. Il m'apparait que 2 théories int.ressantes plus ou moins mécanistqieu amais pas tant de pronlèeme. Mais comment comprendre que certain relation soit visibles à lare échelle est difficile à comprendre ou sont la preuve que certaine hypothèse nitamen la saturation..
Ce modèle offre des perspectives nouvelles et un point d'ancrage concret sur plein de belles choses.



\subsection{Publication envisagée}

En l'état plutot théorique et lève un certains nombre d'obstacle en ce sens
mais aussi aller plus loin avec des données en post doc et aussi plu loind ans population
proposée pour post doc...

Ce article traite de deux ascpect les perspectives qu'offre un approche et propose un proeneir pas pour sclaer ver sles ppulations depuis le haut.
Le modèle qui y est présenté est simple mais innovant et offre possibilité pour explorer des hypothèses.
Le chapitre est en cours de dévelpppement. L'avancement est indiqué dans le rapport du troisième compte-rendu de comité de thèse.
Il est la base d'un porjet que je viens de proposer pour le post doc.
Miguel Araujo et Loic Pelissier. cool

Un grand merci à pour m'avoir donnée des suggestion por améliorer la clareté du texte. Solraik..
