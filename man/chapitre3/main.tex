\section{Context}\label{context}

Biogeographers have long been fascinated by the picture of species
distributions and questioned how it could have been made, i.e.~searching
for the processes shaping biodiversity on Earth
\citep{Engelbrecht2007, Tittensor2010}. Starting from the clear
relationship between abiotic variables and the physiological constraints
of organisms, large-scale studies have been conducted in a
pattern-driven perspective making Biogeography a realm of correlations
\citep{Gaston2000}. Such an approach has provided many valuable
knowledge along with the development of efficient statistical tools.
However, in the context of global changes, many researchers claim for
strengthening the theoretical foundations of the field towards a
biogeography mechanism-driven providing reliable biodiversity forecasts
\citep{Lomolino2000, Thuiller2013, Violle2014}.

The importance of biotic constraints on species distribution are one of
the many concerns regarding this request \citep{Godsoe2012, Araujo2014}.
In order to test whether interactions influence species distributions,
the simplest avenue is to investigate the species co-distribution in
light of their ecological relationships. Such investigation started with
Diamond's original study stating that species interacting by competition
should avoid each other in space, leading to a `checkerboard'
distribution \citep{Diamond1975}. This idea was rapidly criticized for
the lack of an adequate null hypothesis \citep{Connor1979, Gilpin1982}.
Nevertheless, the resulting debate captured the attention of
biogeographers as it must unravel whether co-occurrence data are more
than the sum of occurrence information of a set of species
\citep{Clark2014}. The answer to this question as direct and major
consequences: a negative one would support the use of classical species
distribution models (hereafter SDMs, \citet{Elith2009}) that postulate
the independence among species whereas a positive one would give credit
to methods taking co-occurrence information as a proxy for ecological
interactions \citep{Morales-Castilla2015} and would support the
development of methods including network into species distribution
models \citep{Ovaskainen2010, Pollock2014}.

Recent theoretical developments have proposed mechanisms explaining how
ecological interactions must affect the fundamental niche (see Box 1)
and how they could impact occurrence data \citep{Holt2009, Araujo2014}.
However, ranges of species are very often inferred from the realized
niche which includes the impact of abiotic and biotic factors.
Therefore, finding evidences of interaction signals may prove difficult
which would explain the scarcity of studies reporting such effect (but
see \citet{Gotelli2010}). Fortunately, the co-occurrence theory in
interaction networks has been formalized and suggests that the
repercussion of interactions in co-occurrence data depends on the
structure of the network \citep{Cazelles2016}. Notably, the higher the
degree of a species, \emph{i.e} the number of species which with it
interacts, the trickier the detection of a signal of co-occurrence.
Finding good correspondence between network topology and co-occurrence
signal in empirical data would support the idea that interactions affect
geographic ranges, even if for many pairs of interacting species no
significant co-occurrences are found.

Here, we examined occurrence data in the light of these recent
theoretical developments for five different datasets for which
interactions are observed or assessed to determine whether ecological
interactions impact the distribution of species. We report that the
analysis of co-occurrence failed to clearly reveal a difference between
pairs of interacting species and pairs of not-interacting species.
However our results suggest that the degree of species influences our
ability to detect significant association making co-occurrence
information more than a collection of co-occurrence only for species
with a limited number of link. Moreover, we discover a clear
relationship between the co-occurrence strength of a species and the
cumulated occupancy of the entire set of species with which a species
interact. Interestingly, we point out that the relation vanishes when we
used classical SDMs which questions the capacity of these approaches in
capturing adequately species interactions. Our results reveal the
existence of conditions under which interaction may or may not be
detected in static co-occurrence data. This lead us to think that
interaction in biogeography is not only a matter of scale but also a
matter of relevant biological unit for which occurrence must be
predicted.

\section{Material and Methods}\label{material-and-methods}

\subsection{Datasets}\label{datasets}

We analyzed five datasets spawning a large range of environmental
conditions (see Fig S1 and SI Text), a large diversity of organisms and
covering all fundamental type of interactions (see \citet{tbl:id}). Four
of them came with observed interactions: the Willow Leaf Network (WLN),
the Pitcher Plants Network (PPN), the Caribbean Hummingbirds Network
(CHN), the French Breeding Birds Survey (FBBS). For these network, we
derived metawebs, \emph{i.e}, the matrix recording all interactions, and
computed three metrics: (1) the connectance of metawebs (proportion of
links), the degree of species (number of links for one species) and the
shortest-path between all pairs of species (the minimum number of link
from one species to the other, see SI Text). For the North American
Trees (NAT) datasets, we derive a distance based on functional traits
(see table S1 and Fig S2). For the FBBS dataset, we also derived
different trait-based distance (see table S2). For all datasets, we kept
only species that were present at least on 1\% of the total number of
sites (see SI Text).

\subsection{Measures of co-occurrence}\label{measures-of-co-occurrence}

For each pair of species, we determined the number of observed
co-occurrence \(O_{i,j}\) and we calculated the expected co-occurrence
values \(E_{i,j}\) and its standard deviation \(SD_{i,j}\) to compute
the following Z-score \(O_{i,j}-E_{i,j}/SD_{i,j}\) \citep{Gilpin1982}.
For this standardized metric of co-occurrence, positive (negative)
values indicates more (less) co-occurrence than randomly expected.
Expectations were derived using three different methods. First, we
assumed that all sites were equivalent, meaning we occulted the
potential influence of abiotic conditions. The distribution of
co-occurrence for a limited number of sites have been already studied
elsewhere \citep{Gilpin1982, Veech2013}, therefore, we used an
hypergeometric distribution (see SI Text for further details). For the
two other expected values, we used two different classical SDMs, namely,
Generalized Linear Model (hereafter GLM) and Radom Forest (hereafter RF)
in order to assign a probability of being presence in a given site for
all species (see SI Text for more details and Fig S3 for the assessment
of models' performances). With these two last scenario, we actually
hypothesize that species may often co-occur simply because they have
similar abiotic requirements which is basically what SDMs do.

\section{Results}\label{results}

For two out of four datasets for which interactions were known, we
obtained a clear difference between interacting and not-interacting
species (\ref{fig:synth} panels A to D). Therefore, when integrating all
pairs of species we did not obtain a clear evidence that direct
interaction species co-occur differently from not-interacting one. For
the WLN, distinguishing herbivore-willow interactions from
herbivore-parasitoids revealed that the strength of co-occurrence was
stronger for the former than the latter ones (\ref{fig:shtpth} A-B).
Interestingly, we noticed that the higher the mean degree of species in
the dataset, the more difficult the detection of a signal of
interactions in co-occurrence was (\ref{fig:shtpth} A-D).

For the NAT and FBBS datasets for which we inferred a distance based on
functional traits, we found that co-occurrence where higher for pairs of
similar species (\ref{fig:synth} panels E and F). As similarity could be
taken as a proxy for competition strength \citep{Morales-Castilla2015},
this result suggests that competition is poorly detectable at large
scale which is theoretically supported \citep{Araujo2014}. Therefore,
co-occurrences of similar species are likely driven by the similarity of
their abiotic requirements. The results for the FBBS dataset were
identical irrespective the type of traits examined (Fig S4). This
hypothesis was further supported by the decrease of the Z-score with the
distance for both datasets (fig S6 A and D).

For all datasets, we report that the distribution is concentrated but
the Wilcoxon rank test we used (see SI Text) lead us to the distribution
is symmetric about 0 taking environmental context into account shifts
the distribution toward 0 and diminishes the variance associated
(\ref{fig:synth}). This trend is more pronounced for RF than for SDM.
Hence, assuming that sites are not identical for species due to the
abiotic context reduces the signal of co-occurrence which could even
disappeared. In the PPN dataset, we found that the signal is even
reversed (\ref{fig:synth}) but the quality of the SDM approaches used
were low (fig S3 B).

When examining the Z-scores against the shortest path between preys
(hosts) versus predators (pollinators), we report that the higher the
number of links the less distinguishable observed co-occurrence are from
random ones. For shortest paths higher than one, a large part of
co-occurrence are non significant (\ref{fig:shtpth} A-D) as predicted by
the theory, but the observed decay is steeper \citep{Cazelles2016}. We
conclude that species separated by more than one link may be treated as
independent species. The decay is similar when all pairs of species were
taken into account (see Fig S5). Moreover when we calculated the mean
degree in of predators (pollinators), we demonstrate the signal of
co-occurrence for low shortest path is stronger for specialists
(\ref{fig:shtpth} A) than for generalists (\ref{fig:shtpth} C)
suggesting that the abundance of interactions may prevent us from
detecting them in static occurrence data.

Finally, we analyze the mean Z-score of a predator (pollinator),
\emph{i.e} Z-scores averaged over all the set of its preys (host
plants), against the total number of site covered by its preys (hots)
(\ref{fig:degocc}). We report that when abiotic context is not taken
into account, their is a clear negative relationship explaining up to
.69 of the variance (\ref{fig:degocc} panels A, D, G and J). Therefore,
when a predator feeds on a set of preys that jointly cover a large part
of the geographic range studied, the impact of species interactions is
undetectable, but when the joint repartition of the prey is restricted,
the imprint of interactions remains appreciable. Additionally, the
associated linear regression outperformed the one using the degree of
the species that has been envisioned by the theory (Fig S7 panels A, D,
G and J). Also, we show this relation asymmetric: the decay is less
convincing when the the mean Z-scores of the preys are plotted against
the cumulated range of their predators (fig S8 panels A, D, G and J).
Hence the imprints of interactions in static occurrence are appreciated
once relevant pair of specials species are student. When species are
highly linked with other species and when these species have ranges that
do not completely overlap, we cannot make clearly co-occurrence to
interactions. This suggest that the range of the set of species should
be examined rather that individual range of prey. Interestingly, we
found that using the presence of the whole set of prey as predictor to
assign the presence of species outperformed GLMs (see Fig S9) but not
RFs.

All these last results are no longer true once abiotic constrains are
taken into account, the clear relationship obtained is either weakened
or even reversed (\ref{fig:degocc} B,C,E,F,H,I,K,L) meaning the signals
of co-occurrence for specialists are no longer different from the one of
generalists. If we consider that this relationship is biologically
meaningful as biotic constraints may be softer for generalists than for
specialists we must then question to what extent inferring species
distribution fail in capturing occurrence properties exhibit by species
assemblage.

\subsection{Discussion}\label{discussion}

Our study demonstrate how the information of ecological networks can
shed light upon static occurrence data over broad spatial scales.
Although we noted that we cannot assert in general that interacting
species co-occur differently from non-interacting one, we highlight the
possibilities offer by asking alternative questions such as the impact
of the abundance of interactions and the relative positions of two
species within the network on their co-occurrence. Our results
demonstrate that there is an imprint of ecological interactions on
occurrence data but also that there are specific conditions under which
they can be detected. Hence, we show that the signal of co-occurrence is
blurred by the abundance of interactions. This has been envisioned by a
recent theoretical work \citep{Cazelles2016} and stress the usefulness
of such studies to propose ways to revisit co-occurrence data with
additional information. Moreover, we also prove relevant the examination
of a the range of a given predator against the range of the set of its
preys rather that individual prey. This indicates that the role of
ecological interactions may not only be a matter of spatial scale
\citep{McGill2010} but also a matter of consistent biological unit to be
predicted at over large spatial scales.

Classical SDMs approaches are based on the assumption that species are
independent and focus one abiotic variables \citep{Jeschke2008} to
propose scenarios of tomorrow's biodiversity. Emerging approaches
propose to release this strong hypothesis by predicting a set of species
as assemblage among which correlations are accounted for
\citep{Pollock2014, Warton2015b}. The development of more powerful
statistical tools is fundamental, however it can only provides a partial
solution to improve range forecasting. A substantial part of the
solution lies in the understanding of the role of ecological and
evolutionary mechanisms shaping species distributions
\citep{Thuiller2013}. Here, we underline the need for not hypothesizing
\emph{a priori} the independence among species. Rather, the assumption
must be proved to apply, otherwise a relevant assemblage must be
modeled. Here, we suggest that for generalist species, the assumption of
independence is reasonable while for specialists the relative position
with the other species within the network should help deciding which set
of species are to be modeled. We claim for the integration of ecological
information at the core of SDMs and we suggest that the amount of
information to integrate should be a reasonable. Therefore, once
identified, such information should greatly strengthen our predictions
for a fair additional complexity into our SDMs.

Our results lead us to conclude that the abundance of interaction
renders the assumption of independence valid which therefore simplifies
the predictions. Interestingly, it has been recently highlighted that
some allometric features of networks are better predicted when the
species richness increases \citep{Berlow2009}. More than forty years
ago, MacArthur similarly noted the following paradox ``How can a more
complex community be easier to understand? A possible answer might be
that the complex community has strong interactions among species so that
the lives of the separate species are less independent than in a simple
community. Where there is greater interdependence, patterns may be more
conspicuous.'' \citep[p.199]{macarthur1972geographical}. In our study,
the strong interactions among species ease the forecasting of
biodiversity. However, under the ongoing mass extinction, a myriad of
links vanishes while new ones emerge with the changes in the composition
of local communities. Therefore, even if under current conditions the
assumption of independence may be valid, under dramatic modifications of
ecosystem as currently observed, it may often prove false. As a step
forward, new biodiversity scenarios must not solely map the future
ranges of individual species but the entire community including the
consequences of potential extinction in term of community structure.

\newpage

\subsection{Box 1}\label{box-1}

The fundamental niche is here described as the occurrence probability
under the assumptions that (1) biotic factors are not limiting occupancy
and (2) that dispersion is unconstrained. In this case, only abiotic
factors (such as water availability, temperature variability and edaphic
variables) limits survival and/or reproduction success, and then the
occurrence probability. Consequently, predators occupancy is computed
assuming that preys are abundant enough all along the environmental
gradient. Similarly, the fundamental niche of any prey is not influenced
neither by predators nor by competitors.

For a three species network made of one predator and its two preys, we
derive the three fundamental niches \(f_i\) (\ref{fig:box1} A).
Regarding the predator (species 3), we assume its prey are equivalent
and that the presence of at least one prey is sufficient to release all
the biological constraints:

\[f_3(w)=P(X_3=1|X_2+X_1>0, G=w)\]

where \(G\) denotes the environmental gradient and \(X_i\) is the random
variable associated to the presence of species \(i\). Similarly, \(f_1\)
and \(f_2\) are obtained assuming that 3 is absent :

\[f_2(w)=P(X_2=1|X_3=0, G=w)\]

and:

\[f_1(w)=P(X_1=1|X_3=0, G=w)\]

Once projected on a map, the fundamental niche unravels the potential
distribution of a species \citep{Kearney2004}. The expected distribution
can be compared to real observations and could reveal whether dispersal
limits and ecological interactions are prevalent in the occupancy
dynamic of studied species. The realized niche (\ref{fig:box1} B)
includes these factors.\\
In our simplified example, fundamental and realized niches of preys are
identical. The realized niche of the predator, \(r_3\), is controlled by
the joint realized niches of its preys:

\[r_3(w)=f_3(w)\left(1-(1-r_1(w))(1-r_2(w))\right)\]

The above expression may often be more complicated due to the size and
the structure of the network. For instance, we do not consider the
apparent competition between 1 and 2 although it must affect the
distribution of all species. Integrating the impact of many interactions
may be possible using occurrence probabilities of species assemblages
rather than single species \citep{Cazelles2015b}. Integrating network
information to shed light upon species distribution is also crucial to
understand what kind of co-occurrence is biologically relevant. Consider
as an example the co-occurrence between species 1 and 3: the
co-occurrence may be strong if we restrict the analysis to the suitable
conditions for species 1 but it must be weak if the entire environmental
gradient is sampled. However, if we examine the co-occurrence between 3
and the assemblage made of species 1 plus 2, the co-occurrence may
always be strong. Although this is meaningful in a biological point of
view, co-occurrence studies often remain focus on pairs of species.

\newpage

\subsection{Tables}\label{tables}

\begin{longtable}[]{@{}lrrrrrrr@{}}
\caption{Data sets analyzed in this article.
\label{tbl:data}}\tabularnewline
\toprule
\begin{minipage}[b]{0.15\columnwidth}\raggedright\strut
Type\strut
\end{minipage} & \begin{minipage}[b]{0.07\columnwidth}\raggedleft\strut
No. of sites\strut
\end{minipage} & \begin{minipage}[b]{0.07\columnwidth}\raggedleft\strut
No. of species\strut
\end{minipage} & \begin{minipage}[b]{0.11\columnwidth}\raggedleft\strut
Interaction type\strut
\end{minipage} & \begin{minipage}[b]{0.05\columnwidth}\raggedleft\strut
Observed\strut
\end{minipage} & \begin{minipage}[b]{0.04\columnwidth}\raggedleft\strut
Traits\strut
\end{minipage} & \begin{minipage}[b]{0.06\columnwidth}\raggedleft\strut
Connectance\strut
\end{minipage} & \begin{minipage}[b]{0.22\columnwidth}\raggedleft\strut
References\strut
\end{minipage}\tabularnewline
\midrule
\endfirsthead
\toprule
\begin{minipage}[b]{0.15\columnwidth}\raggedright\strut
Type\strut
\end{minipage} & \begin{minipage}[b]{0.07\columnwidth}\raggedleft\strut
No. of sites\strut
\end{minipage} & \begin{minipage}[b]{0.07\columnwidth}\raggedleft\strut
No. of species\strut
\end{minipage} & \begin{minipage}[b]{0.11\columnwidth}\raggedleft\strut
Interaction type\strut
\end{minipage} & \begin{minipage}[b]{0.05\columnwidth}\raggedleft\strut
Observed\strut
\end{minipage} & \begin{minipage}[b]{0.04\columnwidth}\raggedleft\strut
Traits\strut
\end{minipage} & \begin{minipage}[b]{0.06\columnwidth}\raggedleft\strut
Connectance\strut
\end{minipage} & \begin{minipage}[b]{0.22\columnwidth}\raggedleft\strut
References\strut
\end{minipage}\tabularnewline
\midrule
\endhead
\begin{minipage}[t]{0.15\columnwidth}\raggedright\strut
Willow Leaf Network\strut
\end{minipage} & \begin{minipage}[t]{0.07\columnwidth}\raggedleft\strut
374\strut
\end{minipage} & \begin{minipage}[t]{0.07\columnwidth}\raggedleft\strut
156\strut
\end{minipage} & \begin{minipage}[t]{0.11\columnwidth}\raggedleft\strut
Trophic / Parasitism\strut
\end{minipage} & \begin{minipage}[t]{0.05\columnwidth}\raggedleft\strut
yes\strut
\end{minipage} & \begin{minipage}[t]{0.04\columnwidth}\raggedleft\strut
no\strut
\end{minipage} & \begin{minipage}[t]{0.06\columnwidth}\raggedleft\strut
0.042\strut
\end{minipage} & \begin{minipage}[t]{0.22\columnwidth}\raggedleft\strut
unpublished\strut
\end{minipage}\tabularnewline
\begin{minipage}[t]{0.15\columnwidth}\raggedright\strut
Pitcher Plants Network\strut
\end{minipage} & \begin{minipage}[t]{0.07\columnwidth}\raggedleft\strut
39x20\strut
\end{minipage} & \begin{minipage}[t]{0.07\columnwidth}\raggedleft\strut
53\strut
\end{minipage} & \begin{minipage}[t]{0.11\columnwidth}\raggedleft\strut
Trophic\strut
\end{minipage} & \begin{minipage}[t]{0.05\columnwidth}\raggedleft\strut
yes\strut
\end{minipage} & \begin{minipage}[t]{0.04\columnwidth}\raggedleft\strut
no\strut
\end{minipage} & \begin{minipage}[t]{0.06\columnwidth}\raggedleft\strut
0.44\strut
\end{minipage} & \begin{minipage}[t]{0.22\columnwidth}\raggedleft\strut
\citet{Baiser_2011}\strut
\end{minipage}\tabularnewline
\begin{minipage}[t]{0.15\columnwidth}\raggedright\strut
Caribbean Hummingbirds Network\strut
\end{minipage} & \begin{minipage}[t]{0.07\columnwidth}\raggedleft\strut
32\strut
\end{minipage} & \begin{minipage}[t]{0.07\columnwidth}\raggedleft\strut
62\strut
\end{minipage} & \begin{minipage}[t]{0.11\columnwidth}\raggedleft\strut
Mutualist\strut
\end{minipage} & \begin{minipage}[t]{0.05\columnwidth}\raggedleft\strut
yes\strut
\end{minipage} & \begin{minipage}[t]{0.04\columnwidth}\raggedleft\strut
no\strut
\end{minipage} & \begin{minipage}[t]{0.06\columnwidth}\raggedleft\strut
0.011\strut
\end{minipage} & \begin{minipage}[t]{0.22\columnwidth}\raggedleft\strut
\citet{Mart_n_Gonz_lez_2015}, \citet{Sonne_2016},
\citet{Lack_1973}\strut
\end{minipage}\tabularnewline
\begin{minipage}[t]{0.15\columnwidth}\raggedright\strut
North American Trees\strut
\end{minipage} & \begin{minipage}[t]{0.07\columnwidth}\raggedleft\strut
128891\strut
\end{minipage} & \begin{minipage}[t]{0.07\columnwidth}\raggedleft\strut
31\strut
\end{minipage} & \begin{minipage}[t]{0.11\columnwidth}\raggedleft\strut
Competition\strut
\end{minipage} & \begin{minipage}[t]{0.05\columnwidth}\raggedleft\strut
yes\strut
\end{minipage} & \begin{minipage}[t]{0.04\columnwidth}\raggedleft\strut
no\strut
\end{minipage} & \begin{minipage}[t]{0.06\columnwidth}\raggedleft\strut
-\strut
\end{minipage} & \begin{minipage}[t]{0.22\columnwidth}\raggedleft\strut
unpublished\strut
\end{minipage}\tabularnewline
\begin{minipage}[t]{0.15\columnwidth}\raggedright\strut
French Breeding Birds Survey\strut
\end{minipage} & \begin{minipage}[t]{0.07\columnwidth}\raggedleft\strut
2354\strut
\end{minipage} & \begin{minipage}[t]{0.07\columnwidth}\raggedleft\strut
179\strut
\end{minipage} & \begin{minipage}[t]{0.11\columnwidth}\raggedleft\strut
Trophic\strut
\end{minipage} & \begin{minipage}[t]{0.05\columnwidth}\raggedleft\strut
yes\strut
\end{minipage} & \begin{minipage}[t]{0.04\columnwidth}\raggedleft\strut
no\strut
\end{minipage} & \begin{minipage}[t]{0.06\columnwidth}\raggedleft\strut
0.018\strut
\end{minipage} & \begin{minipage}[t]{0.22\columnwidth}\raggedleft\strut
\citet{Gauzere2015}\strut
\end{minipage}\tabularnewline
\bottomrule
\end{longtable}

\newpage

\section{Figures}\label{figures}

\begin{figure}[htbp]
\centering
\includegraphics{../fig/figConcept.pdf}
\caption{\textbf{Probabilistic description of fundamental and realized
niches} For a three species network all the occurrence probabilities are
derived along an environmental gradient assuming that A interactions are
not limiting the distribution and B that species 3 needs at least of one
of its preys, \emph{i.e.} species 1 or 2. Horizontal dotted lines stand
for the occurrence probabilities reached at an environmental
optimum.\label{fig:box1}}
\end{figure}

\newpage

\begin{figure}[htbp]
\centering
\includegraphics{../fig/figIntVsNoint.pdf}
\caption{\textbf{Co-occurrence of interacting versus not-interacting
pairs of species} Figures under each groups of boxplots indicate the
number of pairs to which the Z-score distributions refer. The light grey
rectangle corresponds to the 95\% confidence interval for the standard
normal distribution which gives insight into the proportion of pairs of
species significantly different from 0. The comparison made in panels A
to D is based on direct interactions observed. For panels E and F,
similar species are defined as the species for which the trait-based
distance is less than or equal to the lower decile of this distance
distribution. Note that outliers are not displayed. P values were
computed using the Wilcoxon rank sum test, to compare interacting versus
not-interacting Z-score distribution calculated for the three different
methods (black symbols) and to show whether the distribution is
symmetric about 0 (light grey symbols).\label{fig:synth}}
\end{figure}

\newpage

\begin{figure}[htbp]
\centering
\includegraphics{../fig/figOrder.pdf}
\caption{\textbf{Co-occurrence signal decays when the shortest path
between a pair of species decay } The Z-score distribution are plotted
against the shortest path for A willows-herbivores interactions, B
herbivores-parasitoids interactions, C birds-plants interactions and D
the pitcher plants network. First figures under each grouped boxplots
indicate the shortest path associated while the figures below provide
the number of pair to which the distribution refers. Note that we used
the same y-axis for panels A and B as they regard two different kind of
interaction of the same dataset.\label{fig:shtpth}}
\end{figure}

\newpage

\begin{figure}[htbp]
\centering
\includegraphics{../fig/figdegocc.pdf}
\caption{\textbf{Co-occurrence significance decreases as the cumulated
occupancy increases} For a given species, Z-scores are averaged over the
all set species it interacts with and plotted against the joint
distribution of the same set of species. We do so for the herbivores in
the willows leafs network (panels A to C), the parasitoids in the willow
leafs network (panels D to F), the hummingbirds in the Caribbean
hummingbirds datasets (panels G to I) and all species in the pitcher
plants network that consume other species (panels J to L). The x-axis is
expressed as a log proportion of the total number of sites. Black
symbols are mean Z-scores significantly different from 0 (see SI Text).
In each panel, the dotted line represents the linear regression
\(y~ax+b\) for which the \(R^2\) is provided. The size of circles
reflects the degree of species for which the Z-score was calculated, the
relation size-degree for each row is given in the middle panel. For the
hummingbirds dataset (panels G to I), the triangle represent the values
obtained for the former distribution of a species already analyzed (see
SI text).\label{fig:degocc}}
\end{figure}

\newpage
