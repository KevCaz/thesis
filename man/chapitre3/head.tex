\chapter{Quelles traces les interactions biotiques laissent-elles dans les données de cooccurrence des espèces?}
\label{chap3}

\section{Résumé en français du troisième article}

Dans ce troisième article, je me suis intéressé aux données de cooccurrence.
Ces données, très importantes en biogéographie, sont des enregistrements
de présences et d'absences pour un certain nombre d'espèces
le long d'un gradient géographique (par exemple, un gradient latitudinal).
On dispose alors d’un grand nombre de sites sur lesquels les conditions abiotiques
varient et pour lesquels nous disposons également de la connaissance
d’une partie du contexte biotique (les présences de différentes espèces).
En plus de ces informations, les jeux de données présentés sont munis
d’observations relatives aux informations réelles avec lesquelles il est possible
de construire des réseaux d'interaction. J'ai donc analysé la cooccurrence à la
lumière des propriétés des réseaux trophiques, mutualistes ou de compétition
avec des espèces plus ou moins généralistes ce qui m’a permis de tester les
hypothèses faites au chapitre \ref{chap2}.


Nos résultats montrent que les interactions ont un effet sur la cooccurrence
mais que la détection de cet effet est délicate quand 1) les espèces sont éloignées
dans le réseaux, les interactions indirectes sont donc difficiles à mettre en
évidence et 2) quand les interactions sont nombreuses, il est plus
difficile de trouver des signaux de cooccurrence \footnote{Il s'agit de
déviations par rapport à des attendus sous l'hypothèse que les interactions
n'ont pas d'influence} pour les espèces généralistes. De plus, en intégrant
la similarité des facteurs abiotiques pour les différents sites, je montre que
les signaux de cooccurrence s’affaiblissent et parfois disparaissent. Mes
résultats suggèrent donc qu’en utilisant des facteurs abiotiques pour
inférer les probabilités de cooccurrence, une partie du lien entre les espèces
est capturée, mais cette part est entachée d’une grande incertitude. Ceci vient
questionner la qualité des prédictions données par les modèles classiques de
distribution d'espèces actuellement utilisés. Nos résultats propose un nouvel
éclairage sur un débat classique de l'écologie concernant la détection
des interactions à partir des aires de distribution. Nous montrons que la
configuration du réseau est aussi importante et qu’il ne faut pas trop
rapidement conclure que les espèces sont indépendantes, ce qui est souvent une
hypothèse de départ dans les projections que nous faisons pour anticiper
les conséquences des activités humaines sur les changements de biodiversité.



\subsection{Publication envisagée}

Le travail ici présenté a été rendu possible grâce à une collaboration entre différents
chercheurs menée avec Dominique Gravel. Les jeux de données que j'ai analysé
sont, en effet, difficiles à trouver et particulièrement intéressantes.
Je remercie tous les chercheurs qui m'ont accordé leur confiance
pour analyser les données et toutes les personnes qui ont contribué à la collecte
des données. Pour les données sur les colibris et les plantes des Caraïbes,
je remercie Bo Dalsgaard, Louise J. Lehmann, Ana M. Martín González,
Andrea Baquero, Allan Timmermann. Pour le jeu de données relatif aux communautés
associées aux feuilles de saules, je remercie Tomas Roslin. Pour les données
sur les communautés microbiennes des Sarracénies (\emph{Sarracenia purpurea}),
je remercie Benjamin Baiser. Pour les données du Suivi Temporel des Oiseaux
Communs (STOC), je remercie Wilfried Thuiller et pour les données des arbres,
je remercie Dominique Gravel et Steve Vissault.

L'article qui suit est le fruit de discussions avec Dominique Gravel sur le type
d'analyse à mener pour donner suite au chapitre \ref{chap2}. Je me suis occupé
de l'ensemble des analyses, des figures, de la rédaction de la première version
relue par Dominique Gravel. J'ai également bénéficié d'une précieuse et
attentive relecture de David Beauchesne que je remerci chaleureusement.
Due à la qualité des jeux de données, à l'originalité de nos analyses et
nos résultats, je nourris l'espoir que mes résultats intéresseront
une revue généraliste. C'est pour cela que le chapitre est dans un format court avec une
longue section "Supplementary information" (SI, \ref{chap3si}). Se confronter
à la rédaction en format court fut aussi un exercice très enrichissant bien que délicat.




\subsection{Traduction du résumé en anglais}

Un problème majeur en biogéographie et en modélisation de la biodiversité
est de savoir si la cooccurrence des espèces contient une trace des
interactions. L'hypothèse conventionnelle affirme que les interactions
négatives et positives conduiraient respectivement à des répulsions et
des attractions. La rareté des jeux de données combinant des distributions
à larges échelles avec des données d'interactions a longtemps empêché
les biogéographes d'obtenir des idées claires sur ce problème. Pour répondre
à cette question, nous avons utilisé cinq jeux de données sur un large
gradient de conditions climatiques incluant des informations sur les interactions écologiques,
représentant un total de 793 espèces pour 354015 observations de cooccurrence.
Nous avons comparé la cooccurrence des pairs d'espèces interagissant avec
celle de paire d'espèces n'interagissant pas en intégrant ou non les
variables abiotiques. Nous avons trouvé un effet des interactions sur la
distribution des espèces de cooccurrence brute mais aucune preuve d'un tel
effet lorsque les variables abiotiques étaient intégrées. De plus, nous avons
trouvé que certaines propriétés du réseau d'interactions pouvaient masquer l'impact
des interactions sur la cooccurrence. Nous avons ainsi montré que plus le nombre
d'interactions qu'une espèce entretenait était grand, plus notre capacité
à détecter le signal dans les données d'interactions statiques était faible.
De plus, nous démontrons clairement que le signal de cooccurrence entre un
prédateur et ses proies disparaît quand la proportion de sites couverts par
l'ensemble de proies augmente. Dans un contexte où les écosystèmes sont
fortement perturbés par l'activité humaine, nos résultats insistent
sur le besoin d'intégrer les processus écologiques dans les
modes de distribution d'espèces pour mieux prédire la biodiversité de demain.






\emph{Les sections qui suivent sont celles de l'article envisagé.}


\newpage
