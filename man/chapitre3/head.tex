\chapter{Do interacting species co-occur differently from not-interacting species?}


\section{Résumé en français du troisième article}

Dans ce troisième article, je me suis me confronter aux données d'occurrence qui sont une données essentielle en biogéograhie pour répondre avec ces données à la question centrale de ma thèse~:Afin de répondre èà la quetsion centrale de ma thèse sur le rôle des interactions dans les conséquencens
Le but de l'article est de tester les prédictions du chapitre précédents, pour évaluer la pertinence d'utiliser l'information
contenue dans les réseaux écologiques pour expliquer les co-occurrences.
Bien que la théori ait été dévoloppé, les données requisent osnt particulièrement rares car elles demandesnt des distribution d'une même communauté sur un large gradient. Dominique Gravel a cependent réussi à réunir de tels jeux de données. Notamment grâce à la partcipiation de Thomas Roselin, Bo Dalsgaard, Wilfred Thuillier, Benjamin Baiser qui nous ont données accès à quatre jeux de données dont les efforts pour les constitués ont été conséquents et je suis redevable à tous ceux qui ont permis la réalisation.

J'ai anlyser la co-occurrence à la lumnière du réseaux pour certain jeux de donnes les inetraction spour un jeux de données sur les arbres nous avons sonctruit un arbre de trait fincitonnel et suput.é uteacrtion. Nos résulats jettent une lumière forteintéressante nous montrons que quand les espèces sont éloignos dans  que les inetractions ont très porbablement un effet sur la co-occureence mais que la détection de cet effet est très vite rendu impossible quand 1) les espèces sont éloignés dans le réseaux et 2) quand les interactions sont nombreuses. Les interaction indirectes n'ont donc pas d'effet mais aussi le nombre affecte notre capcité de sdetection. Il ne fat donc pas conslue et aussi pensé que ce qui est vrai dans un mondre riche en esèces ne l'est pas quand nous alterrons la biovidersité.



\subsection{Publication envisagée}

Due à la nouveauté de nos résultats et la qualité des données j'envsage, J,ai rédigé je serais premier auteur, Dominique Gravel à supervisé dernier et tous les auteures ayanit apportés des données et aprticipé à la rédaction serint includ dans la liste des auteurs. Nous souhaiterins publié dans un et nuso envsoageon de sounettre dans les prochaines semaines à \emph{Proceedings of the National Academy of Sciences} ou le récent journal \emph{Nature Ecology \& Evolution}.



\emph{Les sections qui suivent sont celles de l'article envisagé.}


\newpage
