\chapter{La co-occurrence d'une paire d'espèce intéragissant est-elle différente de la co-occurrecnce d'une paire n'interagisssant pas?}
\label{chap3}

\section{Résumé en français du troisième article}

Dans ce troisième article, je me suis me intéresser aux données de co-occurrence.
Ces données, très importantes en biogéographie, sont des enregistrements
de présences et d'absences pour un certain nombre d'espèces
le long d'un gradient géographique (par exemple latitudinal).
On dispose alors d’un grand nombre de sites sur lesquels les conditions abiotiques
varient et pour lesquels nous disposons également de la connaissance
d’une partie du contexte biotiques (les présences de différentes espèces).
En plus de ces informations, les jeux de données présentés sont munis
d’observations relatives aux information réelles avec lesquelles il es possible
de construire des réseaux d'interaction. J'ai donc analysé la co-occurrence à la
lumière des propriétés des réseaux de type trophiques, mutualistes ou de compétition
avec des espèces plus ou moins généralistes ce qui m’a permis de tester les
hypothèses faîtes au chapitre \ref{chap2}


Les résultats montrent que les interactions ont un effet sur la co-occurrence
mais que la détection de cet effet est délicate quand 1) les espèces sont éloignées
dans le réseaux : les interaction indirectes sont finalement presque
indétectables et 2) quand les interactions sont nombreuses : il est plus
difficile de trouver des signaux de co-occurrence pour les espèces généralistes.
De plus, en intégrant la similarité des facteurs abiotiques
pour les différents sites, je montre que les signaux de co-occurrence
s’affaiblissent et parfois disparaissent. Mes résultats suggèrent donc qu’en
utilisant des facteurs abiotiques pour inférer les probabilités de co-occurrence,
une partie du lien entre les espèces est capturée, mais cette part est entachée
d’une grande incertitude. Ceci vient questionner la qualité des prédictions
données par les modèles classiques de distribution d'espèces actuellement utilisés.
Nos résultats jettent une lumière forte intéressante sur un débat classique en
écologie concernant la détection des interaction à partir des aires de
distribution. Nous montrons que la configuration du réseau est aussi
importante et qu’il ne faut pas trop rapidement conclure que les espèces sont
indépendantes, ce qui est souvent à la base des projections que nous faisons
pour anticiper les changements de biodiversité.




\subsection{Publication envisagée}


L'article a été rendu possible grâce à une collaboration entre différents
chercheurs menéee avec Gravel. Les jeux de données que j'ai analysé
sont en effet rares et particulièrement riche en informations.
Je remercie tous les chercheurs qui m'ont accordé leur confiance
pour analyser les données et tous ceux qui ont contribué à la collect des données
analysé. Pour les données sur les collibris et les plantes des Carraibes,
je remecie Bo Dalsgaard, Louise J. Lehmann, Ana M. Martín González,
Andrea Baquero, Allan Timmermann. Pour les données sur les communautés des feuilles de Saules,
je remercie Tomas Roslin.Pour les données sur les communautés microbiennes des
Sarracénies (\emph{Sarracenia purpurea}), je remercie Benjamin Baiser.
Pour les données du Suivi Temporel des Oiseaux Communs (STOC), je remercie Wilfried Thuiller
et pour les données des arbres, je remercie Dominique Gravel et Steve Vissault.

L'article qui suit est le fruit de discussions avec Dominique Grave sur le type
d'analyse à mener. Je me suis occupé de l'ensemble des analyses, des figures,
de la rédaction de la première version relue par Dominique Gravel. J'ai
bénéficié d'une précieurse relecture attentive de David Beauchesne que je
merci particulièremnt. Due à l'originalité et la nouveauté de nos résultats
produits sur des jeux de données, je nourris l'espoir que mes résulats intéresseront
une revue généraliste. C'estpour cela que le chapitre est formatté court avec une
longue section "Supplementary information" (SI, \ref{chap3si}). Ce fut aussi
un exercie très enrichissant bien que délicat. 


\emph{Les sections qui suivent sont celles de l'article envisagé.}


\newpage
