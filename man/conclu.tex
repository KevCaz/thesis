\section*{Interactions écologiques et distribution des
espèces}\label{interactions-uxe9cologiques-et-distribution-des-espuxe8ces}
\addcontentsline{toc}{section}{Interactions écologiques et distribution
des espèces}

\subsection*{Des modèles théoriques à
développer}\label{des-moduxe8les-thuxe9oriques-uxe0-duxe9velopper}
\addcontentsline{toc}{subsection}{Des modèles théoriques à développer}

Dans son appel pour un renouvellement de la théorie de la biogéographie
des îles, Mark Lomolino soulignait le besoin d'intégrer davantage de
mécanismes écologiques et évolutifs autour des trois processus
fondamentaux de la biogéographie~: colonisation, extinction et évolution
\citep{Lomolino2000}. Au chapitre \ref{chap1}, je me suis confronté à ce
problème en proposant une démarche visant à incorporer les interactions
et les contraintes abiotiques dans la TIB. J'ai proposé d'utiliser un
cadre mathématique général (détaillé à l'annexe \ref{annII}) reliant ces
facteurs aux probabilités d'extinction et de colonisation. Cette
approche est simple et prometteuse, je la considère aujourd'hui comme
une extension de la TIB en ce sens que la théorie classique devient un
cas particulier pour lequel les interactions et les contraintes
abiotiques n'ont pas d'influence sur les taux de colonisation et
d'extinction.

Le modèle proposé au chapitre \ref{chap1} souligne le potentiel d'une
approche communauté-centrée dans laquelle le raisonnement repose sur des
probabilités de transition entre différents assemblages
\citep{Cazelles2016a} plutôt que sur le traitement indépendant de la
présence de plusieurs espèces. Je pense que cette idée était aussi en
germe dès 1972 dans le livre de MacArthur, mais sans avoir été
mathématiquement complètement formalisée
\citep{macarthur1972geographical}. Partir des assemblages pour
comprendre la distribution des espèces posent néanmoins un problème
majeur~: le nombre de communautés à envisager devient rapidement très
élevé (pour \(n\) espèces, ce sont \(2^n\) assemblages possibles), ce
qui limite la mise en application du modèle développé au chapitre
\ref{chap1} dans sa formulation actuelle. Cela étant dit, il est
possible que des moyens émergent pour en réduire la complexité et qu'il
soit progressivement transformé en une méthode d'inférence statistique
efficace.

En me confrontant à l'incorporation des interactions écologiques dans la
TIB (chapitre \ref{chap1}), je me suis aperçu à quel point il était
délicat de construire des modèles simples et élégants expliquant un
grand nombre de faits. En conséquence, je ne suis pas étonné que la TIB
soit toujours abondamment utilisée comme point de départ de nombreuses
études \citep{Warren2015} et cela en dépit de ces défauts reconnus dans
la réédition de 2001 de \emph{The Theory of Island Biogeography} par
Wilson lui même~: « The flaws of the book lie in its oversimplification
and incompleteness, which are endemic to most efforts at theory and
synthesis. «

L'objet \emph{interactions} n'est pas simple à manipuler~; à l'échelle
de la communauté, les interactions ne peuvent pas être traitées
isolement~: elles forment des réseaux. Un champ entier de la
mathématique est dédié à l'étude de tels objets qui sont appelés
graphes. La théorie qui traite de ces objets est utilisée dans
différents domaines pour appréhender les réseaux, qu'ils soient sociaux
ou neuronaux. Les neurosciences d'ailleurs, en cherchant à expliquer la
cognition à partir des réseaux neuronaux, pointent, comme l'écologie,
les difficultés à comprendre les systèmes caractérisés par
l'interdépendance de leurs unités \citep{Park2013}. Avec de tels objets
à modéliser, il n'est pas étonnant de voir des questions importantes de
l'écologie traitées par des mathématiques assez pointues
\citep{Allesina2012a, Rohr2014}.

En parallèle des questionnements très précis, il me semble également
important que des réflexions soient menées pour aller vers des approches
simplifiées et davantage intégratives. En forçant un peu le trait, en
écologie nous avons d'un côté des modèles qui avec très peu de
populations considérées engendrent des dynamiques complexes voir
chaotiques \citep[dont l'existence est validée
expérimentalement][]{Costantino1997b, Fussmann2000} et de l'autre des
modèles comme celui de la TIB qui, avec une équation différentielle
simple, propose une vision profonde de la biogéographie
\citep{MacArthur1967}. Je pense qu'il est tout aussi pertinent d'essayer
de partir de l'échelle la plus large pour aller vers des échelles plus
petites que de mener la démarche inverse. Il est par ailleurs possible
que les objets finaux à prédire, c'est-à-dire d'un côté l'abondance
relative de populations en interaction et de l'autre la composition
spécifique sur des larges échelles spatiotemporelles, ne puissent être
prédits de la même façon, comme le suggère \citet{Lawton1999}, ce qui
serait une forme de \emph{rupture de symétrie} \citep[au sens
de][]{Anderson1972}. Quoi qu'il en soit, c'est bien en essayant
d'utiliser une approche simplifiée mais plus intégrative que j'ai réussi
à mieux cerner le fait qu'elles pouvaient être les traces laissées par
les interactions écologiques sur les distributions d'espèces.

\subsection*{Des théories pour mieux appréhender les données de
co-occurrence}\label{des-thuxe9ories-pour-mieux-appruxe9hender-les-donnuxe9es-de-co-occurrence}
\addcontentsline{toc}{subsection}{Des théories pour mieux appréhender
les données de co-occurrence}

Le chapitre \ref{chap2} bien que théorique, est une avancé importante en
direction de l'application de mes réflexions à des données empiriques.
Il y est en effet question de données de co-occurrence et de réseaux
écologiques. Les données d'occurrence constituent une source de
réflexion importante pour les biogéographes sur lesquelles se concentre
l'effort de développement méthodologique du domaine
\citep{Elith2006, Phillips2006, Pollock2014}. Les données de
co-occurrence sont issues de la considération simultanée de données
d'occurence de plusieurs espèces sur un ensemble de sites dispersé le
long d'un gradient géographique. Exploiter ces données permet, par
exemple, d'envisager la structure des assemblages de demain
\citep{Albouy2012}. En proposant une réflexion de l'impact des
interactions écologiques sur les données de co-occurrence, j'ai essayé
d'améliorer la compréhension de la nature de l'information que pouvaient
contenir les données de co-occurrence. Ce travail est crucial pour
orienter le développement des outils sur lesquels sont construits les
scénarios de changement de la biodiversité \citep{Godsoe2015}. De
manière générale, il s'agit de comprendre le lien qu'il existe entre la
distribution d'une espèce et sa niche hutchinsonienne
\citep{Pulliam2000, Godsoe2010a}. En utilisant un modèle basée sur des
probabilités simples et la version trophique de la TIB
\citep{Gravel2011}, j'ai découvert des attentes théoriques précises sur
les données de co-occurrence et j'ai montré que l'empreinte des
interactions écologiques sur les données de co-occurrence n'est pas
appréciable, notamment lorsque les interactions sont nombreuses.

L'article présenté au chapitre \ref{chap3} propose de tester la théorie
du chapitre \ref{chap2}. En analysant des données de co-occurence pour
des systèmes dont les interactions étaient documentées, j'ai montré que
celles-ci laissent des traces visibles dans les données statiques de
co-occurrence. La détection de signaux de co-occurrence imputables aux
liens écologiques liant les espèces n'est cependant possible que sous
certaines conditions, à savoir lorsque les espèces interagissent
directement, soit quand le nombre d'interactions est limité. De manière
cohérente, la distribution d'un prédateur spécialiste est très corrélée
avec celle de sa proie alors qu'un prédateur généraliste voit sa
distribution partiellement corrélée avec un grand nombre de
distributions, celle de ces proies, ce qui rend difficile la présence
d'un signal clair dans la co-occurrence d'un généraliste avec une de ces
proies. Un signal peut néanmoins exister lorsque l'on examine la
corrélation de la distribution de ce prédateur et la répartition
géographique jointe de l'ensemble de ces proies.

En travaillant sur les co-occurrences avec des données de distribution
d'espèces en interaction, j'ai aussi pointé du doigt un problème
important de l'application des SDMs au regard des réseaux écologiques.
La co-occurrence forte de deux espèces est souvent interprétée comme le
témoin de la similarité de leurs besoins physiologiques, ce qui justifie
d'utiliser des projections à l'échelle de l'espèce pour prédire des
communautés \citep{Rehfeldt2006, Albouy2012}. Cela dit, en partant de ce
principe et en optant pour l'utilisation d'un espace explicatif fait
exclusivement de variables abiotiques, le risque est très fort
d'attribuer la co-occurrence aux variables abiotiques alors que la cause
profonde de cette co-occurrence pourrait être l'interaction entre les
espèces. Il en est de même en inférant la distribution d'une espèce à
partir de la distribution des autres espèces du réseaux~: en nous
concentrons sur les variables biotiques, nous occulterions le signal
abiotique. Ce problème de choix des variables est illustré au chapitre
\ref{chap3} dans lequel j'ai montré que l'utilisation de SDMs
affaiblissait considérablement le signal observé sur les données de
co-occurrence brutes (pour lequel les facteurs abiotiques ne sont pas
intégrés). Au lieu de conclure que la co-occurrence est contrainte par
les variables pédoclimatiques, j'ai cherché la cohérence de cette
affaiblissement et constaté que même les associations les plus fortes
(pour les prédateurs spécialistes et leurs proies) disparaissaient et
qu'un modèle simple basé sur la présence de proies était plus performant
que certains SDMs. Ainsi, l'association entre deux espèces interagissant
n'est pas nécessairement capturée de manière adéquate par les SDMs, ce
qui jette le doute sur la qualité des prédictions basées sur les SDMs.
Pour remédier à cela, une fusion méthodique des deux types
d'informations est nécessaire \citep{Meier2010, Thuiller2013}. Dans les
cas précis d'un prédateur et de ses proies\footnote{cela est aussi
  valable pour un pollinisateur et les plantes qu'il pollinise, ou
  encore pour un parasite et ses hôtes.}, il y a un lien évident entre
les distributions: le prédateur est nécessairement limité par la
distribution conjointe de ces proies \citep{Holt2009, Shenbrot2007}. La
reconnaissance des réalités des interactions biotiques auxquelles les
espèces sont soumises doit être au coeur d'un renouvellement des
approches pour prédire des espèces en réseaux \citep{Godsoe2015}.

Tout au long des chapitres \ref{chap2} et \ref{chap3}, j'ai souligné
l'intérêt des développements théoriques pour mieux comprendre les
données empiriques. En partant initialement de la question \emph{est-ce
que les espèces qui interagissent co-occurrent différemment que celles
qui n'interagissent pas}, j'ai compris qu'il n'y avait pas de réponse
tranchée, mais plutôt une réponse qui dépendait de la nature du réseau.
Ce résultat sera, je pense, très utile pour amener une lumière nouvelle
sur le débat qui anime la communauté des biogéographes autour de la
question du rôle des interactions dans la distribution des espèces aux
larges échelles spatiales \citep{Araujo2014, Godsoe2015}. Bien que dans
les dernières années des avancements méthodologiques significatifs ont
été fait avec l'essor des JSDMs
\citep{Ovaskainen2010, Pollock2014, Warton2015b}, il faut prolonger
l'effort pour intégrer davantage d'informations biologiques dans les
SDMs car l'intégration systématique des co-occurrences n'est peut être
pas suffisante. Je n'ai été capable d'interpréter de manière cohérente
les données de co-occurrence qu'avec l'apport d'une source d'information
extérieure~: celles des réseaux, grâce à laquelle j'ai pu montré que
l'importance des interactions n'était pas seulement une question
d'échelle spatiale \citep{Araujo2014, Belmaker2015}, mais aussi un
problème de topologie du réseaux d'interaction du système étudié.

\section*{Vers une biogéographie
prédictive?}\label{vers-une-bioguxe9ographie-pruxe9dictive}
\addcontentsline{toc}{section}{Vers une biogéographie prédictive?}

\subsection*{Les défis à relever dans un monde en
changement}\label{les-duxe9fis-uxe0-relever-dans-un-monde-en-changement}
\addcontentsline{toc}{subsection}{Les défis à relever dans un monde en
changement}

Érosion de la biodiversité, extinctions de masse, perte de services
écosystémiques, la liste est longue des bouleversements biologiques de
notre époque. Le cinquième rapport d'évaluation du Groupe d'experts
Intergouvernemental sur l'Evolution du Climat (le GIEC\footnote{Tous les
  documents émis par le GIEC sont disponibles en ligne sur le site du
  GIEC \url{https://ipcc.ch} et certaines initiatives rassemblent en des
  documents synthétiques leurs conclusions, voir par exemple
  \url{http://leclimatchange.fr}.{]}}) souligne qu'il y a très peu de
doute sur le lien entre les activités humaines et ces changements. Pour
y faire face, les gouvernements d'un grand nombre de pays doivent agir
de concert et, ces dernières années nous donnent quelques raisons
d'espérer. En janvier 2013 se tenait la première réunion de la
Plate-forme Intergouvernementale scientifique et politique sur la
Biodiversité et les Services Écosystémiques (l'IPBES \footnote{L'IPBES a
  été construit sur le même modèle que le GIEC en ayant pourtant un
  mandat plus large\&nbsp: en plus de faire une synthèse des
  connaissances en vue de guider les décideurs politiques, elle a une
  action dans la production de nouvelles connaissances
  \citep{Brooks2014}.}) qui se veut être un acteur privilégié de la
sauvegarde de la biodiversité à l'échelle mondiale \citep{Diaz2015a}. En
décembre 2015, l'accord de Paris sur le climat signé formellement par
175 pays lors de la 21\textsuperscript{ème} conférence des parties
(COP21), et ratifié par la Chine et les États-Unis au début du mois de
septembre 2016, vise des réductions considérables de gaz à effet de
serre à l'échelle mondiale. Ajouté à ces initiatives internationales,
des initiatives nationales attestent de la prise de conscience des
enjeux de la biodiversité. Ainsi, en France, projet de loi pour la
reconquête de la biodiversité, de la nature et des paysages a été adopté
le 20 juillet 2016 et prévoit la mise en place de l'agence française
pour la biodiversité. En dépit des défis écologiques à relever, il faut
également reconnaître les problèmes liés à l'état d'avancements des
connaissances en écologie. C'est, en effet, un obstacle considérable~:
comment la communauté scientifique peut-elle envoyer un message fort et
cohérent aux décideurs politiques quand elle est traversée par des
débats non résolus? Pour rebondir sur un exemple donné en introduction~:
si l'érable à sucre ne peut migrer assez vite pour suivre les conditions
favorables, nous pouvons artificiellement déplacer des arbres pour
pallier ce problème, c'est ce qu'on appelle la migration assistée. Mais
comment anticiper les conséquences de l'utilisation massive d'une telle
pratique? Un immense effort de synthèse sur les différentes positions du
débat scientifique devrait alors être mené afin d'intégrer les
connaissances dans le domaine pour mettre en place un cadre législatif
cohérent entourant cette pratique, un défi considérable
\citep{McLachlan2007}.

La crise d'extinction d'espèces que nous traversons actuellement
\citep{Thomas2004} est un moment de restructuration intense des
communautés. Ce phénomène est particulièrement visible dans les aires
urbaines où les communautés natives sont remplacées par des communautés
adaptées aux milieux anthropiques. Depuis la seconde partie du
XIX\textsuperscript{ème} siècle, la ville de New York a ainsi perdu 578
espèces natives de plantes vasculaires et gagné 411 espèces non-natives
\citep{DeCandido2004}. Ces communautés urbaines sont, par ailleurs, plus
homogènes à travers les États-Unis que les communautés endémiques
\citep{McKinney2006}. De manière générale, les espèces spécialistes
subissent un recul plus marqué que celui des espèces généralistes, ce
qui conduit à une homogénéisation fonctionnelle qui s'ajoute à une
homogénéisation taxinomique \citep{Clavel2011}. Face à ces
restructurations, deux attitudes opposées existent~: d'un côté la
restauration des communautés qui postule l'existence d'une intégrité
écologique des systèmes naturels \citep{Suding2015} et de l'autre la
primauté du bon fonctionnement des systèmes écologiques quelle qu'en
soit la composition, ce que résume la formule \emph{Don't juge a species
on their origin}, titre d'un article de Mark Davis qui développe cette
idée \citep{Davis2011}.

Alors, doit-on lutter contre le Frelon asiatique? Au-delà d'un
positionnement éthique, il y a une forme de spéculation dans la réponse
car il faut répondre à une autre interrogation~: quelle serait les
conséquences de l'inaction? Si nous laissons la population de frelon
grandir, peut-être qu'une partie de l'entomofaune européenne
disparaîtrait. Mais peut-être aussi que les abeilles domestiques
auraient des stratégies de défense plus efficaces, ce qui stabiliserait
la population du frelon. Les théories actuelles ne nous permettent pas
de savoir avec exactitude lequel des scénarios est le plus probable, et
nous constatons que dans les systèmes vivants, la surprise est
généralement la règle. Dans une étude très récente sur le Diamant
mandarin (\emph{Taeniopygia guttata}), un oiseau commun du centre de
l'Australie, Mylene M. Mariette et Katherine L. Buchanan, ont montré ont
qu'au dessus de 26°C le mâle, seul adulte dans le nid, produit un chant
particulier destiné à ces oeufs induisant une réduction significative de
la taille des futurs oiseaux adultes dont la fertilité est augmentée
\citep{Mariette2016}. Face à cette étude, on mesure les difficultés pour
intégrer l'ensemble des réponses possibles des espèces face à une
perturbation. Leur connaissance exhaustive semble pourtant nécessaire
pour prédire de manière adéquate la réponse des écosystèmes aux
changements actuels. Les systèmes biologiques sont d'une haute
complexité et nos connaissances sont très incomplètes. Nous devons
reconnaître nos carences théoriques mais rester ambitieux pour
construire une écologie prédictive \citep{Mouquet2015}.

\subsection*{Des règles en
écologie?}\label{des-ruxe8gles-en-uxe9cologie}
\addcontentsline{toc}{subsection}{Des règles en écologie?}

Dans un article publié en 1999, John H. Lawton pose la question
suivante~: y a-t-il des lois générales en écologie? Lawton y relève que
les systèmes écologiques sont contraints par des lois fondamentales,
incluant notamment les principes de la thermodynamique et l'évolution.
Cet auteur place au niveau de ces lois fondamentales les interactions
écologiques qu'il présente comme une \emph{observation} \citep[selon ces
mots,][]{Lawton1999}. En dehors de ces lois, Lawton argumente qu'il n'y
a pas de règle universelle, mais plutôt des règles applicables à un
nombre restreint systèmes écologiques. Cette idée est résumée par la
formule suivante (p.178)~: « The universal laws do not allow us to
predict the existence of kangaroos ». Dans son article, Lawton dresse un
portrait d'une écologie dominée par la contingence et suggère d'aller
chercher des règles dans la macroécologie. En dépit de la pertinence de
ses développements, Lawton oublie, à mon avis, de répondre à une
question centrale~: quels sont les objets des prédictions~?
Cherchons-nous vraiment à prédire l'existence des kangourous? Peut-être
que l'objet de la prédiction est plutôt la probabilité de l'émergence,
en Australie, dans un contexte biotique et abiotique donné, d'un
marsupial herbivore d'une taille donnée. Il me semble que présenté
ainsi, l'objectif de prédiction fiable semble bien plus réalisable que
celui présenté par Lawton. La contingence des systèmes biologiques nous
indique qu'il faut un formalisme qui envisage des possibilités et non
pas la supposition d'existence de règles dans ces systèmes. Ainsi, en
suivant le physicien James C. Maxwell, le second principe de la
thermodynamique nous indique que lorsque l'on met en présence de l'eau
chaude et de l'eau froide initialement séparées, obtenir de l'eau tiède
après un certain temps est simplement un évènement très probable.

Les communautés (au sens d'ensemble structuré d'espèce) ont
vraisemblablement des fonctionnements et des propriétés qui leur sont
propres, néanmoins, il y a aussi des rôles écologiques similaires entre
différents réseaux. Les top-prédateurs par exemple~: ces espèces des
niveaux trophiques les plus élevés structurent profondément les
communautés, leur extinction ont de fortes influences sur l'ensemble de
la communautés
\citep{Terborgh2001, Sinclair2003a, Myers2007, Ripple2014}. En plus de
relever les similarités, il faut essayer de comprendre les différences
de fonctionnement au regard de la différence en termes de composition
spécifique. De même que la loi universelle de la gravitation dépend de
la masse des deux corps en interaction et de la distance qui les
séparent, les lois de l'écologie pourraient se décliner aisément en
function de la composition spécifique ou même de la combinaison des
caractéristiques de ces espèces (de traits fonctionnels). C'est
peut-être, à la suite de MacArthur et Wilson \citep{MacArthur1967},dans
la réduction des espèces à quelques propriétés que des règles émergeront
\citep{McGill2006, Poisot2015}. Une des premières propriétés est tout
simplement la masse des espèces dont dépendent un grande nombre de
propriétés écologiques \citep{Woodward2005a}. Le lien avec les lois de
la thermodynamique est évident~: les êtres vivants, pour se maintenir
(et donc respecter les principes thermodynamiques) ont des besoins
énergétiques qui augmentent avec la taille. La taille impose donc des
contraintes essentielles pour construire des approches cohérentes (ce
que j'ai essayé de faire au chapitre \ref{chap4}) sans lesquelles la
recherche de règle éventuelle serait délicate. Il est difficile
d'argumenter que l'écologie a des règles universelles, mais il est
encore plus délicat de démontrer qu'il n'y en a pas (et la démonstration
formelle de ce dernier principe serait une révolution). Aussi,
constatant l'absence actuelle de lois universelles en écologie, je
dirais que nous avons besoin de trouver les chemins par lesquelles elles
pourraient éventuellement être révéler, de même que pour trouver des
traces d'interactions écologiques dans des données statiques de
co-occurrence, c'est finalement la méthode qu'il m'a d'abord fallu
trouver.

\subsection*{Une théorie intégrative de la
biogéographie}\label{une-thuxe9orie-intuxe9grative-de-la-bioguxe9ographie}
\addcontentsline{toc}{subsection}{Une théorie intégrative de la
biogéographie}

Malgré les avancements théoriques des dernières années en biogéographie,
l'appel lancé par Lomolino en 2000 reste d'actualité
\citep{Lomolino2000}. L'effort d'intégration ordonné de concepts clés
issus de différents champs de l'écologie dans une théorie intégrative de
la biogéographie doit être poursuivit \citep{Thuiller2013}. Avec
l'éclairage que je propose sur la relation entre la distribution des
espèces et les interactions écologiques, j'espère que mes travaux
participeront à avancer vers une telle théorie. Comme je l'ai indiqué
tout au long de ma thèse, ce n'est pas tant la liste des mécanismes qui
pose problème mais bien leur articulation cohérente sans laquelle on ne
peut connaître leur importance relative, ce qui freine l'émergence des
méthodes plus cohérentes pour prédire les distribution d'espèces.
Néanmoins, comme le relevaient Miguel B. Araújo, et Carsten Rahbek,
pendant que la recherche progresse, il faut essayer de tirer le meilleur
parti de l'état actuel des connaissances \citep{Araujo2006}~:

\begin{quote}
We will never be able to predict the future with accuracy, but we need a
strategy for using existing knowledge and bioclimatic modeling to
improve understanding of the likely effects of future climate on
biodiversity.
\end{quote}

Les écosystèmes sont des objets complexes, des réseaux dynamiques
d'objets tout aussi complexes. Prédire la composition précise d'un
écosystème donné semble un objectif encore très éloigné, mais estimer le
temps de survie d'une espèce dans un contexte biotique et abiotique
semble être plus à notre portée. Pour y arriver, pour avoir une vision
cohérente de la dynamique des écosystèmes sur des échelles de temps plus
longue, on ne peut négliger ni les processus écologiques ni les
processus évolutifs~: nous devons construire des approches intégratives.
Pour cela, nous devons chercher et trouver le liant entre tous ces
processus et examiner soigneusement les contraintes auxquelles les
espèces n'échappent pas. Les contraintes énergétiques (voir chapitre
\ref{chap4}) en sont un bon exemple. Même l'évolution, toute aussi
pourvoyeuse de contingence qu'elle soit, apporte son lot de
contraintes~: elle met en jeu une variation qui engendre un différentiel
démographique dans un contexte biotique et abiotique donné. Comme le
rappelait Wallace au troisième chapitre de son livre \emph{Island Life}:
« In the first place we must remember that new species can only be
formed when and where there is room for them. »
