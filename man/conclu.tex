\section*{Intéractions écologiques et distribution des
espèces}\label{intuxe9ractions-uxe9cologiques-et-distribution-des-espuxe8ces}
\addcontentsline{toc}{section}{Intéractions écologiques et distribution
des espèces}

\subsection*{Des modèles théoriques à
développer}\label{des-moduxe8les-thuxe9oriques-uxe0-duxe9velopper}
\addcontentsline{toc}{subsection}{Des modèles théoriques à développer}

Dans son appel pour un renouvellement de la théorie de la biogéographie
des îles, Mark Lomolino soulignait le besoin d'intégrer davantage de
processus écologiques et évolutifs autour des trois processus
fondamentaux de la biogéographie : colonisation, extinction et évolution
\citep{Lomolino2000}. Au chapitre \ref{chap1}, je me suis confronté
frontalement à ce problème en proposant une démarche pour incorporer les
interactions et les contraintes abiotiques dans la TIB. J'ai proposé
d'utiliser un cadre mathématique général reliant ces facteurs aux
probabilités d'extinction et de colonisation. Il me semble que cette
approche est simple et prometteuse, je la considère aujourd'hui comme
une extension de la TIB en ce sens que la théorie classique devient une
cas particulier : les interactions et les contraintes abiotiques n'ont
pas d'influence sur les taux de colonisation et d'extinction.

Le modèle proposé au chapitre \ref{chap1} suggère aussi le potentiel de
réfléchir en terme de probabilité d'assemblage \citep{Cazelles2015a}
plutôt que de considérer les espèces individuellement. A la lecture du
livre de Robert MacArthur publié en 1972, j'ai ressenti que cette idée
était présente mais pas formulalisée \citep{macarthur1972geographical}.
Partir des assemblages pour comprendre les présence d'espèces posent un
porblème technique très concret: le nombre de communautés à envisager
devient rapidement très important (pour \(n\) espèces, ce sont \(2^n\)
assemblages possibles), ce qui limite la mise en application du modèle
développé au chapitre \ref{chap1} dans sa formulation actuelle. Cela
étant dit, il est possible que des moyens émergent pour réduire en
compléxité et qu'il soit progressivement transformé en une méthode
d'inférence statique efficace.

En me confrontant à l'incorporation des intéractions écologiques dans la
TIB (chapitre \ref{chap1}), je me suis aperçu à quel point il est
délicat de construire des modèles simples, élégants expliquant à un
grand nombre de faits. En conséquences, je ne suis pas étonné que la TIB
soit toujours abondamment utilisée comme point de départ de nombreuses
études \citep{Warren2015} et cela en dépit de ces défauts reconnus dans
la ré-édition de 2001 de \emph{The Theory of Island Biogeography} par
Edward 0. Wilson lui même :

\begin{quote}
The flaws of the book lie in its oversimplification and incompleteness,
which are endemic to most efforts at theory and synthesis.
\end{quote}

L'objet \emph{interactions} n'est pas simple à manipuler : à l'échelle
de la communauté, les interactions ne peuvent pas être traitées
isolémment, elles forment des réseaux. Il y a un champ de la
mathématique entier dédié à l'étude de tels objets appelé graphes et la
théorie qui traite de ces objets est utilisée pour apprhender des
réseaux de toutes sortes qu'ils soient sociaux ou neuronnaux. D'autres
champs de la biologie utilisent ces objets, les neurosciences par
exemple, et pointent également les difficultés à comprendre les systèmes
caractérisés par l'interdépendence de ses unités. L'écologie des réseaux
bénéficient très directement des travaux de mathématiciens et de
physiciens dont elles retirent des outils performants et de plus en plus
pointus, ce qui ajoute à la difficulté du sujet une complexité
technique.

En parallèle des questionnements très précis que soulèvent différents
champs de l'écologie, il me semble également important que des
réflexions soient menées pour aller vers des modèles plus intégratifs.
Une part importante de l'effort doit être dédiée à des approches
simplifiées et davantage intégratives. En forçant un peu le trait, en
écologie nous avons d'un côté des modèles qui avec très peu de
populations considérées engendrent des dynamiques complexes voir
chaotiques \citep[dont l'existence est validée
expérimentalement][]{Costantino1997b, Fussmann2000} et de l'autre des
modèles comme celui de la TIB qui, avec une équation différentielle
simple, propose une vision profonde de la biogéographie
\citep{MacArthur1967}. Je pense qu'il qu'il est tout aussi pertient
d'essayer de partir de l'échelle la plus large pour aller vers des
échelles plus petites que de mener la démarche inverse. Il est par
ailleurs tout aussi possible que les deux objets finaux à prédire
c'est-à-dire l'abondance relative de populations en interaction et la
composition spécifique sur des larges échelles spatio-temporelles ne
puissent être prédits de la même façon \citep[ce qui serait une forme de
\emph{rupture de symétrie}][]{Anderson1972}. Quoi qu'il en soit, c'est
bien en essayant d'utilser la première approche que j'ai mieux cerné
quelles pouvaient être les traces laissées par les interactions
écologiques sur les distributons d'espèces.

\subsection*{Des théories pour mieux apréhender les données de
co-occurrence}\label{des-thuxe9ories-pour-mieux-apruxe9hender-les-donnuxe9es-de-co-occurrence}
\addcontentsline{toc}{subsection}{Des théories pour mieux apréhender les
données de co-occurrence}

Le chapitre \ref{chap2} bien que théorique, est un pas important en
direction de l'application de mes reflexions à des données empiriques.
Il y est en effet question de données de co-occurrence et de réseaux
écologiques. Les données d'occurrence constituent une source de
réflexion importante pour les biogéographes sur lequel se concentre
l'effort de développement méthodologique du domaine
\citep{Elith2006, Phillips2006, Pollock2014}. Les données de
co-occurrence sont issues de la considération simultannée de données
d'occurence de plusieurs espèces sur un même gradient biogéographique.
Exploiter ces données permet, par exemple, d'envisager la structure des
assemblages de demain \citep{Albouy2012}. En proposant une réflexion de
l'impact des interactions écologiques sur les données de co-occurrence,
j'ai essayé d'améliorer la compréhension de la nature de l'information
que pouvaient contenir les données de co-occurrence. Ce travail de
compréhension du lien qu'il existe entre les processus écologiques et
les données de distributions analysées est crucial pour orienter le
développement des outils sur lesquels sont construits les scénarios de
changement de la biodiversité. De manière générale, il s'agit de
comprendre du lien qu'il existe entre la distribution d'une espèce et sa
niche hutchinsonienne \citep{Pulliam2000, Godsoe2010a}. En utilisant un
modèle de probabilité simple et la version trophique de la TIB
\citep{Gravel2011}, j'ai découvert des attentes théoriques précises sur
les données de co-occurrence et j'ai montré que l'empreinte des
interactions écologiques sur les données de co-occurrence n'est pas
appréciable notament lorsque les intéractions sont nombreuses.

L'article présenté au chapitre \ref{chap3} proposent de tester la
théorie du chapitre \ref{chap2}. En analysant des données de co-occurece
pour des systèmes dont les interactions étaient documentées, j'ai montré
que celles-ci laissent des traces visibles dans les données statiques de
co-occurrence. La détection de signaux de co-occurrence imputables aux
lien écologiques liant les espèces est cepandant possible que sous
certaines conditions: lorsque les espèce interagissent directement,
lorsque le nombre d'interactions est limité. De manière cohérente, la
distribution d'un prédateur spécialiste est très corrélée avec celle de
sa proie alors qu'un prédateur généraliste voit sa distribution
partiellement corrélée avec un grand nombre de distributions, celle de
ces proies, ce qui rend difficile la présence d'un signal clair dans la
co-occurrence d'un généraliste avec une de ces proies. Un signal peut
néanmoins exister lorsque l'on éxamine la corrélation de la distribution
de ce prédateur et la répartition géographique jointe de l'ensemble de
ces proies.

En travaillant sur les co-occurrences avec des données de distribution
d'espèce en interaction, j'ai aussi pointé du doight un problème
important de l'application des SDMs au regard des réseaux écologiques.
La co-occurrence forte de deux espèces est souvent interprétée comme le
temoin de la similarité de leurs besoins physiologiques, ce qui justifie
d'utiliser des projections à l'échelle de l'espèce pour prédire des
communautés \citep{Rehfeldt2006, Albouy2012}. Cela dit, en partant de ce
principe là, lorsque l'on prend pour espace explicatif seulement les
variables abiotiques, l'occurrence des espèces seulement des variables
climatiques, il est vraismblable que nous capturions une part de
l'impact des interactions dans la distributon sans pour autant le voir.
Nous avons montré au chapitre \ref{chap3} que l'utilisation de SDMs pour
obtenir des co-occurrences intégrant les contraintes abiotiques
affaiblissait considérablement le signal observé sur les données de
co-occurrence brutes. L'interprétation immédiate consiste à dire que la
co-occurrence est conrtainte par le variables pédo-climatiques
abiotiques. Néanmoins, le fait que même les associations les plus fortes
(pour les prédateurs spécialistes et leur proie) disparaissent et qu'un
modèle simple basé sur la présence de proies soit plus performant que
certains SDM, semble indiqué qu'une portion de l'effet des interactions
et comem nous ne sommes pas en mesure de connaître précisemment cette
part, il se peut que l'association soit pas très bien reflétée dans les
prédicitons basée sur les SDMs. C'est bien la fusion méthodiques des
deux informations qui doit permettre d'aller vers des approches
systématiques \citep{Meier2010}. Dans les cas précis d'un prédateur et
ses proies\footnote{cela est aussi valable pour un pollinisateur et les
  plantes qu'ils pollinisent ou encore pour un parasite et ses hotes.},
il y a un lien évident entre les distributions: le prédateur est
nécessairement limité par la distribution conjointe de ces proies
\citep{Holt2009, Shenbrot2007}. Ainsi, la reconnaissance de cette
réalité doit être au coeur d'un renouvellemnent des apporches pour
prédire des espèce en réseaux.

Du chapitre \ref{chap2} au chapitre \ref{chap3}, j'ai souligné l'intérêt
des développemnets théoriques pour mieux comprendre des données
empiriques. En partant initialement de la question \emph{est-ce que les
espèces qui interagissent co-occurent différemment que celle qui
n'intéragissent pas}, j'ai compris qu'il n'y avait pas de réponse
tranchée, mais plutôt une réponse qui dépendait de la nature du réseau.
Ce résultat sera, je pense, très utile pour amener une lumière nouvelle
sur le débat qui anime la communauté des biogéographes autour de la
question du rôle des interactions dans la distribution aux larges
échelles spatiale. En méditant sur ce chapitre, j'ai également bien
compris comment le choix d'un espace explicatif donné pouvait amener à
des conclusion qui demandait une alternative. Bien que dans les
dernières années avec l'essort des JSDMs il y a une attention
particulière, il faut apporter davantage de biologie pour bien comprndre
les données que nous traitons et notamment rapidement lever l'hypothèse
d'indépendance \citep{Elith2006}. Je suis convaincu que le problème des
intéractions n'estpas seulement question d'un problème d'échelle
spatiales\citep{Araujo2014, Belmaker2015}, mais c'est aussi une question
qui concerne la nature du système étudié. Mes résultats sont seulement
un premier indice fort en ce sens et soulignent l'intérêt d'étudier le
système pour conclure la nature des facteurs qui sont à prendre en
compte. Pour aller plus loin dans ma réflexion, il faudrait, je pense,
que nous parvenions à une caracctérisation des systèmes pour lesquels
les interactions sont ou ne sont pas important afin que l'on puisse
avoir des règles efficaces pour savoir quelles types d'approches est
pertienent pour quel type de système. C'est une étape importante et
longue pour aller vers des prédictions robustes qui sont très
aujourd'hui plus que nécessaires. En particulier je pense que
L'intégration systématiques des co-occurrence à travers les JSDMs tels
qu'ils sont présentés aujoud'hui ne permetrra pas toujours de comprendre
ce qu'il se passe \citep{Ovaskainen2010, Pollock2014, Warton2015b}.

\section*{Vers une biogéographie
prédictive?}\label{vers-une-bioguxe9ographie-pruxe9dictive}
\addcontentsline{toc}{section}{Vers une biogéographie prédictive?}

\subsection*{Les défis à relever dans un monde en
changement}\label{les-duxe9fis-uxe0-relever-dans-un-monde-en-changement}
\addcontentsline{toc}{subsection}{Les défis à relever dans un monde en
changement}

Érosion de la biodiversité, extinctions de masses, perte de service
écosystémiques, la liste est longue des boulversements biologiques de
notre époque. Le cinquième rapport d'évaluation du Groupe d'experts
Intergouvernemental sur l'Evolution du Climat (le GIEC\footnote{Tous les
  documents émis par le GIEC sont dispobiles en ligne sur le site du
  GIEC \url{https://ipcc.ch} et certaines initiatives rassemblent en des
  documents synthétiques leurs conclsuions, voir par exemple
  \url{http://leclimatchange.fr}.{]}}) souligne qu'il n'y a très peu de
doute sur le lien entre les activités humaines et ces changements. Pour
y faire face, les gouvernements d'un maximum de pays doivent agir de
concert. Les événements des derniers mois sont un source d'espoir avec
l'accord de Paris obtenu lors de la 21\textsuperscript{ème} conférence
des parties (COP21) et que la Chine et les États-Unis ont signé au début
du mois de spetembre 2010, en marge du sommet du G20 à Hangzhou (est de
la Chine). En 2013, la première réunion de la Plate-forme
intergouvernementale scientifique et politique sur la biodiversité et
les services écosystémiques (l'IPBES \footnote{L'IPBES a été construit
  sur le même modèle que le GIEC en ayant pourtant un mandat plus large
  car en plus de faire une sythèse des connaissances en vu de guider les
  décideurs politiques, elle a aussi d'autre rôle dont celui de le
  générer des connaissances \citep{Brooks2014}.}) avait également
apporté de nombreux espoirs dans la protection de la biodiversité è
l'échelle moniale espoirs \citep{Diaz2015a}. De plus, dans différents
pays naissent des initiatives pour faire face aux enjeux de la
biodiversité, en France par exemple, la loi française sur la
bioviversité prévoit la mise en place d'une agence française pour la
biodiversité dans les prochains mois. En dépit de la reconnaisance des
défis posés par la biodiversité, il faut également reconnaître les
problèmes de l'états d'avamcements des cpnnaissance en écologie. C'est
un obstace considérable : comment la communauté scientifique peut-elle
envoyer un message fort et cohérent aux décideurs politiques quand elle
est traversée par des débats non résolus? Pour rebondir sur un exemble
donnée en introduction, si l'érable à sucre ne peut migrer assez vite
pour suivre les conditions favorables, nous pouvons artificiellemnet
déplacé des populations d'érables pour pallier ce problème, c'est ce
qu'on appelle de migration assistée. Mais comment savoir les
conséquences de l'utilisation massive d'une telle pratique? Il faut
alors que les différentes positions du débat scientifique soit intégrer
dans la construcion du cadre législatf entourant cette pratique, ce qui
est un autre défi considérable \citep{McLachlan2007}.

La crise d'extinction majeur que nous traversons \citep{Thomas2004} est
un momnet de restructuration intense des communautés. Ce phénomène est
particulièremnet visible dans les aires urbaines où les communatés
natives ont remplacées par des communautés adaptés au milieu
anthropisée. La ville NewYork a ainsi perdu 578 espèces natives de
plantes vasculaires et gagné 411 non-natives \citep{McKinney2006}; ce
changement de communautée a aussi été signalé comme étant d'une
homogénéité supérieure aux communautés endémiques \citep{McKinney2006}.
De manière plus générale, on assiste à un déclin des espèces spécialites
plus intense que celui des espèces généralistes, ce qui conduit à une
honogénéisation fonctionelle qui s'ajoute à une homogénisation
taxonomique \citep{Clavel2011}. La restructuration des communautés peut
amener à deux positions opposés : d'un côté l'idée de réstaurer les
commuanité de pronner une intégrité écologique des systèmes naturels
\citep{Suding2015} et de l'autre une forme d'examen de la communautés
perturbée et de privilégié le bon fontionnement du sytème sans regarder
l'endémisme de que résume la formule \emph{Don't juge a species on their
origin} qui est le titre d'un artcile de Mark Davis qui développe cette
idée \citep{Davis2011}.

Doit-on lutter contre le frelon asiatique? Quelle serait les
conséquences de l'inaction? Une partie de l'entomofaune européenne
pourrait disparaître mais peut être que les abeilles domestiques
deviendraient plus efficace et finirais par stabiilser la population du
frelon. Comment faire lorsque la surprise est la règle? Dans une étude
très récente sur le Diamant mandarin (\emph{Taeniopygia guttata}), un
oiseau commun du centre de l'Autralie, Mylene M. Mariette et Katherine
L. Buchanan, ont montré ont qu'au dessus de 26°C le mâle seul avec dans
le nid produit un chant particulier à ces oeufs induisant ainis un
changement particulier, réduisant la taille des oiseaux adulte et
augmentant la fertilité \citep{Mariette2016}. Face à cette suprise de
taille, comment penser être en mesure de connaître l'ensembles des
mécanismes dont regorgent les systèmes vivants et dont la connaissance
semble nécessaire pour prédire la réposne des ecosystèmes? Sans même
parler d'évolutions, les systèmes biologiques sont d'une haute
compléxité qu'on ne connait que très partiellement, il faut être
reconnaître l'état de nos connaissances et accepter nos difficultés pour
faire face à l'ensemble des challenges qui demandent de prédictions
fiables et surtout de grands efforts théoriques \citep{Mouquet2015}.

\subsection*{Des règles en écologie et
évolution?}\label{des-ruxe8gles-en-uxe9cologie-et-uxe9volution}
\addcontentsline{toc}{subsection}{Des règles en écologie et évolution?}

En 1999, John H. Lawton posait la question suivante : Y a-t-il des lois
générales en écologie?. Il relevait que les systèmes écologiques sont
contraints par des lois physiques fndamentales comme les principes de la
thermodynamqiues auxquels s'ajoute l'évolution et aussi l'observation
(selon ces mots) que les espèces interacgossent entre elle et avec leur
milieu \citep{Lawton1999}. Cet auteur affirme qu'il n'y a pas de lois
universelles propres à l'écologiques mais plutôt plusieurs règles assez
peu générale et que finalement les lois fondamentales ne permettent pas
de tout expliquer, cet idée est résumé par la formule suivante (p.178) :
« The universal laws do not allow us to predict the existence of
kangaroos; ». Je pense que les idées de Lawton posent un certain nombre
de bonne question mais pose aussi la question de quels peuvent être les
objets de prédicitions. Cherchons vraiment à prédire l'existence des
kangorous? Peut-être que l'objet de la prédiction est plutôt la
probabilité de l'emergence en Australie dans un context biotique et
abiotique donné d'un marsupial herbivore d'une taille donnée. Il me
semble que présemté ainsi, l'objectif est bien plus atteignale que celui
prsenté par Lawton. La contingence des systèmes biologiques indique
seulement qu'il faut construire des règles qui envisage des possibles et
non qu'il n'y a pas de règle. Ainsi, je rappellerai que l'explication
statistiques du second principe nous dit que si dans un même récipienton
met de l'eau chaude et de l'eau froide initialement séparées par un
obstace, lorsuqe l'obscatale est enlevé, avoir de l'eau tiède est
simplement plus probable.

De même il me semble qu'il est trivial de dire que les communautés sont
toutes différenste te que es espèces différents mais il y a des
similarités et des différences qui peuvent être intégré dans un cadre
conspteuel pertinent. On peut avoir des objets mais les regroupé est une
première étape importnate quelle type de loi comme on le faot en
mathémayiqye. Cette catégorisation estune constante dans les travaux en
écologie, elle peut être de différente nature mais souvent foncitonnelle
nous parlons aisin de poducteur primaires, de proie, de prédateur, de
généraliste, de spécialiste\ldots{} Peut=-être que certaine loi
s'applique à certaines espèces et pas d'autres à ecratones combinaison
et pas d'aitres. Ce n'est nécessairemnt des lio différemtes se sont
peut-être des lois à décliner et que ces loins dépend des objets en
question. Cette idée d'avoir des lois bas. sur un set de
chacrastéristque est au coeur de En 2006, McGill proposait de rebâtir
l'écologie des communautés des traits fonctionelle, ces traits qui
mesurent différentes propriétés des espèces \citep{McGill2006}. Aisin au
lieu de se référer à une catéégorisation de l'esèce par son no taxonomie
un ensemble plus objectof sur la bases desquelles des rgles sont à
trouver notamment sur les stratégies de modélisation des ranges. Et
mieux en composition su des prédiction sur les set de triats sont
possibles.

Le problème qu'ajoute l'évolution est la variation de ces sets de traits
en il faut en saisir les raison. Néanmoins, l'évolution n'apporte pas
que des contingences mais bien aussi une nécessité, dans le contexte
dans lequel l'espèce a émergé, elle a réussi à ce perpété et donc
trouver des ressources, eu un succès rpeporucteur. Je pense qu'il aut
davantage insister sur le cadre dans lequel notamment l'insertion aus
sien d'un réseau quidoit être une contraintes forte, et comprendre
comment les variations peuvent apporter la place. Ainsi, pouvons nous
fire des hypothèses sur les produits de l'évolution? Pouvons nous dire
qu'ils ont considérablement optimisé l'utiilsaion de et que la richesse
que nous connaissions est optiale \citep{Rabosky2015}. De même les
allométires 'allométrie aussi \citep{Schneider2012} n'y as t-il pas une
fonctionnement qassumer que énergir?. Recemment Ian Hatton et collègues
ont montré une relation bien aprticulière :il y a une lien entre la
biomasse de différents niveaux trophiques mais la relation n'est pas
qualconque en 3/4 comme une relation allométriqe.é L'espoir mais la
publication de Ian Hatton êut faire douter de l'absence de l'absecmed de
règel. Comment croire qu'il n'y a pas des principes d'ordre enerétique
la-dessous. Convergence\ldots{} La présence importante des relations
allométriques est en fait un espoir de règle mais a envidagé au nibeau
des résauex. \citep{Hatton2015}.

\subsection*{Vers une théorie en intégrative de la
biogéographie}\label{vers-une-thuxe9orie-en-intuxe9grative-de-la-bioguxe9ographie}
\addcontentsline{toc}{subsection}{Vers une théorie en intégrative de la
biogéographie}

L'effort théorique en biogéographie doit se faire autour être
intégration ordonnée de concepts clés issus de différents champs de
l'écologie \citep{Thuiller2013} est une clef essentielle pour aller vers
des prédctions de Ainsi, alors que les conditions climatiques et plus
généralement la géographie physique sont classiquement évoquées pour
expliquer la répartition des espèces \cite{Kearney2004}, les
interactions entre espèces sont quant à elles souvent occultées. De
même, bien que les processus évolutifs soient souvent évoqués comme
déterminants majeurs de la diversité des espèces \cite{Rosindell2011},
leurs effets à court terme sont souvent ignorés \cite{Parmesan2006} dans
les scénarios décrivant la biodiversité de demain \cite{Lavergne2010}.
La difficulté principale est alors de produire des modèles (théoriques
en première instance) qui intègrent l'ensemble des processus et les
relations qu'ils entretient \cite{Thuiller2013} tout en gardant une
relative simplicité. Une théorie intégrative en biogéographie pourrait
être le meilleur point d'ancrage pour construire de nouvelles approches
appliquées. Avec une telle théorie en main, nous pourrions aller vers
l'enjeux majeurs de ces dernières années en biogéographie : relâcher les
hypothèses que les modèles classiques de répartitions des espèces
d'aujourd'hui utilisent (notamment en occultant les interactions) pour
prédire la biodiversité de demain \cite{Guisan2011}.

Dans ma thèse j'ai montré que yMa des choses à décourvir et redécouvirr
et des espoirs. trouver les règles dans des systèmes de nombreuse entoté
hautement complexes individuellemnt et qui interagissent en soit et
dynamqiues dans le temps est l'espèces c'est un défi d'une immensité.
Mais il y a des interpdépendances des réalité, des loins
thermodynqmique, organisé consomation énergie pour l'être fatilité
l'espèce à une histoire elle est là, elle est insérée dans le rseaux.
Biensur il y un certain nombre de chose comment ne pas oender que le
lègue de la TIB n'est aps quelque chose mais et l'emsmeble des théories
est souvent resreint à un chmape à une catégorei et comme moi j'ai
montrés que des système oour lesquels les interaction sosnt plus ou
moins importnates, je pense qu,'il y a un un promier travails ed
evatégoraisation.

\begin{quote}
We will never be able to predict the future with accuracy, but we need a
strategy for using existing knowledge and bioclimatic modeling to
improve understanding of the likely effects of future climate on
biodiversity. \citep{Araujo2006}.
\end{quote}

Des modèles repenser plus biologique et aussi comprdnre queles type s
d'apporche pour quels systèems Cette terminalogie soulève bien des
différences majeurs mais de manière paradoxale les SDMs dont j'ai
souvent parlé dans mon travil de thèse semble être valables pour toutes
les espèces. Biens entendu dans les faits les chercheurs connaissent le
plus souvent les différences des grands groupes et les approches les
plus appropriés pour tel ou tel groupe. Néanmoins quand on ne reconnait
pas dans une forme de systématisation ces différences. Ainsi, si par
exemple, la plupart des SDMs sont efficaces pour traiter des arbres mais
plutôt problématiques pour traiter des oiseaux, il me semble qu'il faut
expliquer pouquoi et ne pas essayer d'affirmer que les interactions sont
importantes ou pas basé sur un ensemble aprticulier d'exemple bien
choisi. En disant cela je pense qu'il serait souhaitable d'avoir des
arguments théorique solide pour dire quel ou quel type d'espèce il faut
prendre en compte tel ou tel facteur pour bien conpendre. Cette idée
peut être batie sur les traits finctionnels.

Il est plus facile de s'appuyer que sur des correlatons d'autant plus
que si des correlatons fortes il existent une esplication peut alors
voir le jour. avoir des erreurs quantifiables mieux dessiner ce qui est
suremnet plus déterminsite de ceux qui l'ai moins Ca ne fat pas une
théorie on peut esp.reer que c'est une bonne apporximation. Mias
cooller\\
orter ma contribution dans les prochaines année.s

De même peut être que des hypothèse eambietieurse, dans des que le tems
à cerie à aller vers des systèmes énergétique Aller vers des contriantes
énergétiques mais il est dur qu'on trouvera des règles fiables sur un
système qui bien que régit par des règles physique assez nien comprise
est un moteur de stochasticé..Oui de l'energie sur le temos et l'espace.
Chapitre 3 Wallace

\begin{quote}
In the first place we must remember that new species can only be formed
when and where there is room for them.(Wallace :56)
\end{quote}
