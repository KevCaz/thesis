\section*{Intéractions écologiques et distribution des
espèces}\label{intuxe9ractions-uxe9cologiques-et-distribution-des-espuxe8ces}
\addcontentsline{toc}{section}{Intéractions écologiques et distribution
des espèces}

\subsection*{Des modèles théoriques à
développer}\label{des-moduxe8les-thuxe9oriques-uxe0-duxe9velopper}
\addcontentsline{toc}{subsection}{Des modèles théoriques à développer}

Dans son appel pour un renouvellement de la théorie de la biogéographie
des îles, Mark Lomolino soulignait le besoin d'intégrer davantage de
processus écologiques et évolutifs autour des trois processus
fondamentaux de la biogéographie : colonisation, extinction et évolution
\citep{Lomolino2000}. Au chapitre \ref{chap1}, je me suis confronté
frontalement à ce problème en proposant une démarche pour incorporer les
interactions et les contraintes abiotique dans la TIB. J'ai proposé
d'utiliser une approche générale reliant ces facteur aux probabilité
d'extinction et de colonisation. Il me semble que cette approche est
simple et prometteuse, elle est finalement une extension de la TIB qui
devient un cas particulier pour lesquelles les interactions et les
contraintes abiotiques n'ont pas d'influence.

Le modèle proposé au chapitre \ref{chap1} suggère aussi le potentiel de
réfléchir en terme de probabilité d'assemblage \citep{Cazelles2015}. A
la lecture de du livre de 1972 de Robert MacArthur, j'ai ressenti que
l'idée était présente mais pas formuler de manière explicite
\citep{macarthur1972geographical}. Bien que je pense que ce type
d'approche soit important, il y une limite en terme de nombre de
communautés à envisager (pour \(n\) espèces, ce sont \(2^n\) espèces)
qui restraint l'application de telle approche. Cela limite l'application
directe du modlèle à des données. Cela étant dit, il est possible que
des moyens émergent pour réduire la compléxité.

En me confrontant à l'incorporation des intéractions écologiques dans la
TIB (chapitre \ref{chap1}), je me suis aperçu à quel point il est hardu
de construire des modèles simples, élégant et qui expliquent à un grand
nombre de faits. En conséquences, je ne suis pas étonné que la TIB soit
toujours utilisée pour un grand nombre d'étude comme point de départ
malgré ces défaut reconnu dans la ré-édition de 2001 de \emph{The Theory
of Island Biogeography} par Edward 0. Wilson lui même :

\begin{quote}
``The flaws of the book lie in its oversimplification and
incompleteness, which are endemic to most efforts at theory and
synthesis.''
\end{quote}

La nature même de l'objet \emph{interactions} n'est pas quelque chose de
simple, à l'échelle de la communauté, les interactions ne peuvent pas
être traitées isolémment, elles forment des réseaux. Il y a un champ de
la mathématique entier dédié à l'étude de tels objets : la théorie des
graphes. L'écologie n'est pas la seule à utiliser ces objets, d'autres
disciplines, comme les neurosciences, pointent également la difficulté
des systèmes caractérisé par l'interdépendence de ses unités. La théorie
des réseaux à amener des mathématiciens et des physiciens à nourir à
l'écologie de leurs outils auxqels les écologues se familiarisent,
doublant la complexités des systèmes d'une compléxité technique.

En parallèle des questionnements pointus que soulève différents champs
de l'écologie, il me semble également important que des réflexions
soient menées pour aller vers des modèles plus intégratifs. Une part
importante de l'effort doit être dédiée à des approches plus simplifiées
mais davantage intégratives. Si je force le trait, d'un côté avec très
peu de population, les dynamiques engendrées peuvent être très complexe
voir chaotique (ce qui est appuyé par des expérience
\citep{Costantino1997b, Fussmann2000}) et de l'autre avec une équation
différentielle très simple on peut donner une vision de la biogéographie
\citep{MacArthur1967}. Il me semble qu'il est tout pertient d'essayer de
partir de l'échelle la plus large et ôir aller vers des échelles plus
petite que la démarche inverse. Il est par ailleurs tout aussi possible
que les deux objets finaux à prédire : abondance relative de populations
en interaction, et composition en espèces ne puissent être prédit de la
même façon \citep[ce qui serait une forme de \emph{rupture de
symétrie}][]{Anderson1972}. Quoi qu'il en soit, c'est en essayant
d'utilser la première approche que j'ai mieux cerné quelles pouvaient
être les traces laissées par les interactions écologiques sur les
distributons d'espèces.

\subsection*{Des théories pour mieux apréhender les données de
co-occurrence}\label{des-thuxe9ories-pour-mieux-apruxe9hender-les-donnuxe9es-de-co-occurrence}
\addcontentsline{toc}{subsection}{Des théories pour mieux apréhender les
données de co-occurrence}

Le chapitre \ref{chap2} bien que théorique, est un pas significatif vers
des apporches plus appiquée. Il est question de données de co-occurrece.
Les données d'occurrence (ou de présence et d'absence) sont les plus
utilisées en biogéographie et qui font l'objet de développement
méthodologique \citep{Elith2006, Phillips2006}. L'information donnée par
la co-occurrence est finalemnt une information donnée par un ensemle de
données d'occurrence qui permet, par exemple d'envisager la structure
des assemblages de demain \citep{Albouy2012}. En proposant une réflexion
de l'impact des interactions écologiques sur ces réseaux, j'ai donc
essayer de mieux comprendre ce que ces données pouvaient contenir
\citep[ce qui est un travail capital comme le souligne][]{Godsoe2010a}.
En repartant sur un modèle de probabilité simple doublé de l'utilisation
de la version trophique de la TIB \citep{Gravel2011} comme support, j'ai
montré comment la théorie permetait de jeter une lumière nouvelle pour
regarder les données de co-occurrence.

Le message central de ma thèse, livré au chapitre \ref{chap3}, s'appuie
sur cette théorie et la confirme. En regardant des données de
co-occurece pour des systèmes pour lesquelles les interactions étaient
connues, j'ai montré que les interactions laissaient des traces visibles
dans les données statiques de co-occurrence sous certaines conditions.
La détection de ce signal n'est, en effet possible que lorsque les
espèce interagissent drectement et lorsuqe le nombre d'interaction n'est
pas trop impornat. De manière cohérente, prédire la distribution d'un
prédateur spécialiste est difficile sans comprender ou sera sa proie.
Par contre pour des généraliste ou pour des paires d'espèces qui ne sont
pas en interactions directes il semble que leur co-occurrence ne pussent
pas être distinguée de rencontres aléatoires. Du chapitre \ref{chap2} au
chapitre \ref{chap3}, j'ai souligné l'intérêt des développemnets
théoriques pour mieux comprendre des données empiriques.

En partant initialement de la question ``Est-ce que les espèces qui
interagissent co-occurent différemment que celle qui n'intéragissent
pas'', j'ai compris qu'il n'y avait pas de réponse tranchée, mais plutôt
une réponse qui dépendait de la nature du réseau. Ce résultat sera, je
pense, très utile pour amener une lumière nouvelle sur le débat qui
enime la communauté des biogéographes, celui de savoir si oui ou non les
interactions sont importantes à larges échelles. Je suis convaincu qu'in
n'est pas seulement question d'un problème d'échelles spatiales
\citep{Araujo2014, Belmaker2015}, mais c'est aussi une question qui
concerne la nature du système étudié. mes résultats indiquent qu'il faut
étudier le système pour conclure la nature des facteurs qui sont à
prendre en compte. Pour aller plus loin dans ma réflexion il faudrait,
je pense que nous arvenions à une caracctérisation des systèmes pour
lesquels les interactions sont ou ne sont pas important afin que l'on
puisse avoir des règles efficace pour savoir quelles types d'approches
est pertienent pour quel type de système. C'est une étape importante et
longue pour aller vers des prédictions robustes qui sont très
aujourd'hui plus que nécessaires.

\section*{Vers une écologie
prédictive?}\label{vers-une-uxe9cologie-pruxe9dictive}
\addcontentsline{toc}{section}{Vers une écologie prédictive?}

\subsection*{Les défis à relever dans un monde en
changement}\label{les-duxe9fis-uxe0-relever-dans-un-monde-en-changement}
\addcontentsline{toc}{subsection}{Les défis à relever dans un monde en
changement}

Érosion de la biodiversité, extinctions de masses, perte de service
écosystémiques, les activités anthropiques ont fortement bouleversé les
écosystèmes. On peut espérer que La facilité des données de
co-occurrence la d.marche des migrations en cours prédictions parfois
exactes parfois juste la migration northwrad rééxaminer esr semble
indiqué qu'il n'y a pas de mgration plus vers le nors. mais besoin de
plus sur 'homogénéité ++ mais avec les espèces invasive le signal est
fortemnt briollé aussi ! Je pense qu'on est a un tournant de la biogoe
vers un chamgemnt de paradigme communaité centré qui ne nit pas les
travaux précédant mais les suit.

Nous assostons à une recomposition des communautés. Lorque l``on parle
de sixième extinction c'est que nous avons des taux record d'extinction
\citep{Thomas2004}. Biensur cela pose des grandes questions sur comme
savoir quel partie du vivant est davatage touché \citep{Thuiller2011},
mais d'un point de vue on est dans une ériode de focntionnenement
particulier qu'on peut voir comme ue grande expérience mais aussi comme
m moment où des théories solides seraient le sbienvenue. Dans son
arctile `Don't juge a species on their origin' Mark Davis prend à revers
un sertain nombre d'idée recu et souligne qye les effects des invedeurs
peuvent être positives \citet{Davis2011}.

Et si on faisait rien pour le frelon asiatique ? Peut être que qu'une
partie de l'entomofaune distparaitrait, peut être que les abeilles
domestiques deviendraient pus efficace et finirais par stabiiser sa
populatiom. Dans tous les cas, au moins au point de vue
\citep{Villemant2011}. Récemment une suprenante étude sur le Diamant
mandarin (\emph{Taeniopygia guttata}), un oiseau commun du centre de
l'autralie, qu'au dessus dessu de 26°C un champ particulier du mâle pour
allerqui induit une différence à des oiseau plus petit et à une meilel
ferticilité \citep{Mariette2016}. Comolexit. des systèmes biologoqes à
prendre et cMest surprise sont fialemnt plutôt la règle et l'excetion et
donc une modestie dans la tâche de modélisé la biodiversité mondiale
\citep{Mouquet2015}.

\subsection*{Des règles en écologie et
évolution?}\label{des-ruxe8gles-en-uxe9cologie-et-uxe9volution}
\addcontentsline{toc}{subsection}{Des règles en écologie et évolution?}

Il est plus facile de s'ppuyer que sur des correlatons d'autant plus que
si des correlatons fortes il existent une esplication peut alors voir le
jour.

Biensur il y un certain nombre de chose comment ne pas oender que le
lègue de la TIB n'est aps quelque chose mais et l'emsmeble des théories
est souvent resreint à un chmape à une catégorei et comme moi j'ai
montrés que des système oour lesquels les interaction sosnt plus ou
moins importnates, je pense qu,'il y a un un promier travails ed
evatégoraisation.

De manière tout a fait probante, l'étude de la nature a été un travaille
de groupement por essayer de classé les êtres vivants par des criètère
plus ou moins cohérents. La classification que nous connaissons
maintentantse base sur les lien de parenté entre les êtres vivants. En
plus de cette catégorisation lobale, nous regroupons les animaux de
manière fonctionnelle en écologique et nous parlons aisin de poducteur
primaires, de proie, de prédateur, de généraliste, de
spécialiste\ldots{} Cette terminalogie soulève bien des différences
majeurs mais de manière paradoxale les SDMs dont j'ai souvent parlé dans
mon travil de thèse semble être valables pour toutes les espèces. Biens
entendu dans les faits les chercheurs connaissent le plus souvent les
différences des grands groupes et les approches les plus appropriés pour
tel ou tel groupe. Néanmoins quand on ne reconnait pas dans une forme de
systématisation ces différences. Ainsi, si par exemple, la plupart des
SDMs sont efficaces pour traiter des arbres mais plutôt problématiques
pour traiter des oiseaux, il me semble qu'il faut expliquer pouquoi et
ne pas essayer d'affirmer que les interactions sont importantes ou pas
basé sur un ensemble aprticulier d'exemple bien choisi. En disant cela
je pense qu'il serait souhaitable d'avoir des arguments théorique solide
pour dire quel ou quel type d'espèce il faut prendre en compte tel ou
tel facteur pour bien conpendre. Cette idée peut être batie sur les
traits finctionnels. En 2006, McGill proposait de rebâtir l'écologie des
communautés des traits fonctionelle, ces traits qui mesurent différentes
propriétés des espèces \citep{McGill2006}. Aisin au lieu de se référer à
une catéégorisation de l'esèce par son no taxonomie un ensemble plus
objectof sur la bases desquelles des rgles sont à trouver notamment sur
les stratégies de modélisation des ranges. Et mieux en composition su
des prédiction sur les set de triats sont possibles.

De même peut être que des hypothèse eambietieurse, dans des que le tems
à cerie à aller vers des systèmes énergétique Aller vers des contriantes
énergétiques mais il est dur qu'on trouvera des règles fiables sur un
système qui bien que régit par des règles physique assez nien comprise
est un moteur de stochasticé..

Quelles hypothèse pouvons nous faire sur les produits de évolution? Si
on peut supposer qu'il y a des compétition ou la règle est le changemnt
cette même propriété a-t-elle des propriétés sur le long terme. Peut-on
affirmer que les produits de l'évoluton dans un enviroemnt stable amène
à des entités qui optimise l'tilisation de l'énergie. Si oui, que dire
des produites de l'évoltion dans avec variation. Si on peut faire des
hypothèses comment les tester. Dans l'article de

Si l'évolution est imprévisible si au dela d'un certain temps on ne peut
presque rien dire\ldots{} Si la chance de des abeilles européennes
changeait comment prédire cela changement de comportement mais que nous
sommes dand l'incapacité de le prédire que pensé du status de l'écologie
et de l'évoluton en tant que science. Si la composant historique domine
le royaume de la biologie devons-nous nous dsatisfaire de le décire.
L'espoir mais la publication de Ian Hatton êut faire douter de l'absence
de l'absecmed de règel. Comment croire qu'il n'y a pas des principes
d'ordre enerétique la-dessous. Convergence\ldots{}

2014, Hurlbert et Stegen propose une série d'hypothèse pour mettre en
évidence l'impact de l'énergy sur l'évolution la troisième hypothèse est
temps suffisant pour équilibre. Une telle hypothèse une forme de
maximistaion de la production de la biomasse et l'utilisation qui est
peut être. Peut-être que les différents mécanismes en jeu dans les
processu évolutifs amène probablement à une forme de
stationarité\ldots{}

avoir des erreurs quantifiables mieux dessiner ce qui est suremnet plus
déterminsite de ceux qui l'ai moins

\subsection*{Vers une théorie en intégrative de la
biogéographie}\label{vers-une-thuxe9orie-en-intuxe9grative-de-la-bioguxe9ographie}
\addcontentsline{toc}{subsection}{Vers une théorie en intégrative de la
biogéographie}

En s'appuantddur un champ bien dessiner il afiat être précis en
biogéogrpaheo

En me confrontant à l'incorporation des intéractions écologiques dans la
TIB (chapitre \ref{chap1}), je me suis aperçu à quelle point il est
difficil de construire des modèles simples, élégant et qui expliquent à
un grand nombre de fait. Je ne suis pas étonné que la TIB soit toujours
utilisée pour un grand nombre d'étude comme point de départ malgré ces
défaut reconnu dans la ré-édition de 2001 de \emph{The Theory of Island
Biogeography} par Edward 0. Wilson lui même :

\begin{quote}
``The flaws of the book lie in its oversimplification and
incompleteness, which are endemic to most efforts at theory and
synthesis.''
\end{quote}

L'effort théorique en biogéographie est importnat et 'intégration
ordonnée de concepts clés issus de différents champs de l'écologie
\cite{Thuiller2013} est une clef essentielle pour aller vers des
prédctions de Ainsi, alors que les conditions climatiques et plus
généralement la géographie physique sont classiquement évoquées pour
expliquer la répartition des espèces \cite{Kearney2004}, les
interactions entre espèces sont quant à elles souvent occultées. De
même, bien que les processus évolutifs soient souvent évoqués comme
déterminants majeurs de la diversité des espèces \cite{Rosindell2011},
leurs effets à court terme sont souvent ignorés \cite{Parmesan2006} dans
les scénarios décrivant la biodiversité de demain \cite{Lavergne2010}.
La difficulté principale est alors de produire des modèles (théoriques
en première instance) qui intègrent l'ensemble des processus et les
relations qu'ils entretient \cite{Thuiller2013} tout en gardant une
relative simplicité. Une théorie intégrative en biogéographie pourrait
être le meilleur point d'ancrage pour construire de nouvelles approches
appliquées. Avec une telle théorie en main, nous pourrions aller vers
l'enjeux majeurs de ces dernières années en biogéographie : relâcher les
hypothèses que les modèles classiques de répartitions des espèces
d'aujourd'hui utilisent (notamment en occultant les interactions) pour
prédire la biodiversité de demain \cite{Guisan2011}.

Comme un prmier pas plus loin que mes travax le chapitre \ref{chap4}
vien apporter un pas vers le développent dd'un théorie métabllolqieude
la vers laquelle e veux aporter ma contribution dans les prochaines
année.s

\textless{}!-- Dans une théorie intégrative de la biogéographie, les
traits fonctionnels peuvent être un pivot très intéressant pour
rassembler les différents concepts que nous avons développés dans les
paragraphes précédents. Les traits peuvent tout d'abord être mis en
relation avec le milieu abiotique. Le taux métabolique ou encore la
sensibilité à la sécheresse sont des indices performant pour décrire la
survie dans un milieu donné \cite{Kearney2004,Engelbrecht2007} que l'on
peut capturer sous forme de traits. Kearney \textit{et al.} 2010 propose
une approche prometteuse dans laquelle, l'environnement physique, la
disponibilité des ressources et la dynamique énergétique sont reliées
par les traits fonctionnelles le tout aboutissant à un modèle de
distribution très mécanistes. La structure d'un réseaux peut également
être dérivée à partir de l'espace des traits. Dans leur méthode proposée
cette année, Gravel \textit{et al.} infèrent les paramètres du modèle de
niche de Williams et Martinez \cite{Williams2000} à partir des relations
de masse du corps entre proie et prédateurs \cite{Gravel2013}. Ils sont
alors en mesure de dériver un réseau global pour un ensemble d'espèce
donné. Enfin, en tant qu'expression phénotypique, les traits
fonctionnels sont soumis aux processus évolutifs. Sur les temps longs,
l'expression de l'évolution résulte en la modification progressive des
traits qui se répercute sur l'ensemble des propriétés qui en découle.
Ainsi la considération d'une modification des traits est une approche
simple et réaliste pour introduire les processus évolutifs et leurs
conséquences \cite{Guill2008,Loeuille2005}.
