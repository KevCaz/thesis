\section*{Revister les données de co-occurrences avec la
donnée}\label{revister-les-donnuxe9es-de-co-occurrences-avec-la-donnuxe9e}
\addcontentsline{toc}{section}{Revister les données de co-occurrences
avec la donnée}

\subsection*{Revister les données de co-occurrences avec la
donnée}\label{revister-les-donnuxe9es-de-co-occurrences-avec-la-donnuxe9e-1}
\addcontentsline{toc}{subsection}{Revister les données de co-occurrences
avec la donnée}

Le message central de ma thèse est donné au chapitre \ref{chap3} de ma
thèse. Après les efforts théoriques des chapitres \ref{chap1} et
\ref{chap2}, j'y ai montré que l'information des réseaux écologiques
étaient un apport important pour bien interpréter les données de
co-occurrence. En partant initialement de la question ``Est-ce que les
espèces qui interagissent co-occurent différemment que celle qui
n'intéragissent pas'', j'ai compris qu'il n'y avait pas de réponse
tranché mais plutôt une réponse qui dépendait de la nature du réseau.
Grâce à l'analyse de quelques propriétés des réseaux étudiés et d'une
analyse d'un grand nombre de co-occurrence j'ai montré qu'on ne pouvait
pas détecter de signal des intéractions dans les données de
co-occurrence statique. De plus j,I montré qu'affirmer que des espèce
séparé de plus de deux liens dans les réseaux ne pouvaient pas être
distingué d'une co-occurrence aléatoired. Ce résultat sera, je pense,
très utile pour amener une lumière nouvelle sur le débat. Le problème
n'est peut être pas seulement un problème d'échelle spatiale
\citep{Araujo2014, Belmaker2015} mais aussi un problème de la nature du
système étudié. Ce résultats nous indique qu'il faut étudier le système
et comprendre sous quelle condition inclure les interactions est
important ou nous. Le porblème de cette assertion est qu'elle
sous-entend une forme de contingence alors même que nous cherchons des
règles. Il me semble qu'un travail de reflexions sur les groupes est
amané.

\subsection*{Vers une catégorisation}\label{vers-une-catuxe9gorisation}
\addcontentsline{toc}{subsection}{Vers une catégorisation}

De manière tout a fait probante, l'étude de la nature a été un travaille
de groupement por essayer de classé les êtres vivants par des criètère
plus ou moins cohérents. La classification que nous connaissons
maintentantse base sur les lien de parenté entre les êtres vivants. En
plus de cette catégorisation lobale, nous regroupons les animaux de
manière fonctionnelle en écologique et nous parlons aisin de poducteur
primaires, de proie, de prédateur, de généraliste, de
spécialiste\ldots{} Cette terminalogie soulève bien des différences
majeurs mais de manière paradoxale les SDMs dont j'ai souvent parlé dans
mon travil de thèse semble être valables pour toutes les espèces. Biens
entendu dans les faits les chercheurs connaissent le plus souvent les
différences des grands groupes et les approches les plus appropriés pour
tel ou tel groupe. Néanmoins quand on ne reconnait pas dans une forme de
systématisation ces différences. Ainsi, si par exemple, la plupart des
SDMs sont efficaces pour traiter des arbres mais plutôt problématiques
pour traiter des oiseaux, il me semble qu'il faut expliquer pouquoi et
ne pas essayer d'affirmer que les interactions sont importantes ou pas
basé sur un ensemble aprticulier d'exemple bien choisi. En disant cela
je pense qu'il serait souhaitable d'avoir des arguments théorique solide
pour dire quel ou quel type d'espèce il faut prendre en compte tel ou
tel facteur pour bien conpendre. Cette idée peut être batie sur les
traits finctionnels. En 2006, McGill proposait de rebâtir l'écologie des
communautés des traits fonctionelle, ces traits qui mesurent différentes
propriétés des espèces \citep{McGill2006}. Aisin au lieu de se référer à
une catéégorisation de l'esèce par son no taxonomie un ensemble plus
objectof sur la bases desquelles des rgles sont à trouver notamment sur
les stratégies de modélisation des ranges. Et mieux en composition su
des prédiction sur les set de triats sont possibles.

\section*{Les défis à relever dans un monde en
changement}\label{les-duxe9fis-uxe0-relever-dans-un-monde-en-changement}
\addcontentsline{toc}{section}{Les défis à relever dans un monde en
changement}

Des conférences, des mesures, des érosions de écosystèmes, des
extinctions en masse, notre monde en témoigne qu'on a eu la COOP21 qui a
ce jour cherche encore à comprenre. les services écosystèmies
perdus\ldots{}Reconfiguration des écosystèmes naturelle li y a eu
d'autre crise avant. Finalemnt avec du catastrophisme, la question s'est
si nous on ira mal. On est grons pour la taille de la planète peut être
plus suceptibel è l'extinciton que l'on pense. Maid ce ,est pas le
pessismise qui m'importe. - Et si on faisait rien pour le frelon
asiatique ?

\subsection*{Anticiper les changments de
biodiversité}\label{anticiper-les-changments-de-biodiversituxe9}
\addcontentsline{toc}{subsection}{Anticiper les changments de
biodiversité}

La facilité des données de co-occurrence la d.marche des migrations en
cours prédictions parfois exactes parfois juste la migration northwrad
rééxaminer esr semble indiqué qu'il n'y a pas de mgration plus vers le
nors.

\subsection*{Avons-nous des espoirs vains
?}\label{avons-nous-des-espoirs-vains}
\addcontentsline{toc}{subsection}{Avons-nous des espoirs vains ?}

ans son arctile `Don't juge a species on their origin' Mark Davis prend
à revers un sertain nombre d'idée recu et souligne qye les effects des
invedeurs peuvent être positives \citet{Davis2011}.

L'érosion de la biodiversité exergue une certaine nostalgie qui parfois
conduit une forme de fatalisme chez certain experts. Nous travaillons
dans un monde qu nous avons déjà depuis bien longtemps..

L'oiseau autralien Le royaume de la contingence du à l'impact historique
de l'histoire evolutive.

Invaion de vespa ou alors des inovation socilae meilleur défense des
abeilles?

Les conséquences sont compliqués des changements climatiques sont
nombreuses et certaines espèce voir le range grandir d'autre diminuer
pour cds espèce de co existent et donc à un changemnet prononc. de al
morphologoe des communautés alors que le nombre d'espèce peut être peu
affecté \citet{Moritz2008}

Oui mais même sans aller trop vite, il faut refaire

'homogénéité ++ mais avec les espèces invasive le signal est fortemnt
briollé aussi ! Je pense qu'on est a un tournant de la biogoe vers un
chamgemnt de paradigme communaité centré qui ne nit pas les travaux
précédant mais les suit.

\section*{Vers une écologie
prédictive?}\label{vers-une-uxe9cologie-pruxe9dictive}
\addcontentsline{toc}{section}{Vers une écologie prédictive?}

\subsection*{Un défi théorique
majeur}\label{un-duxe9fi-thuxe9orique-majeur}
\addcontentsline{toc}{subsection}{Un défi théorique majeur}

En me confrontant à l'incorporation des intéractions écologiques dans la
TIB (chapitre \ref{chap1}), je me suis aperçu à quelle point il est
difficil de construire des modèles simples, élégant et qui expliquent à
un grand nombre de fait. Je ne suis pas étonné que la TIB soit toujours
utilisée pour un grand nombre d'étude comme point de départ malgré ces
défaut reconnu dans la ré-édition de 2001 de \{\emph{The Theory of
Island Biogeography}\} par Edward 0. Wilson lui même :

\begin{quote}
``The flaws of the book lie in its oversimplification and
incompleteness, which are endemic to most efforts at theory and
synthesis.''
\end{quote}

Intégrer des objets tels que les interactions est finalement quelques
chose de très compliqé et il est importnat que des mathématiciens, des
physiciens partcipent à apporter des outils à l'écologie. Il me semble
aussi qu'il est important que des réflexions soient mené pour des
modèles plus intégrtaifs et essayé des approches, certes simplificatrice
mais qui intégrent finalement différents aspect. Il me semble qu'on peut
schématiser nous avons des modèles de Lotka-Volterra et de l'autre et de
l'autre des modèle plus holiste comme la TIB qui se parle finalement
assez peu. Biensur il peut y avoir un fondement mais une \emph{rupture
de symétrie} mais encore faut il comprende son orgine. Cela serait aussi
un message pour essayer par les deux bouts: de large échelle vers petit
et de petit vers grand d'aller vers davantage d'intégration.

\subsection*{Vers une théorie en intégrative de la
biogéographie}\label{vers-une-thuxe9orie-en-intuxe9grative-de-la-bioguxe9ographie}
\addcontentsline{toc}{subsection}{Vers une théorie en intégrative de la
biogéographie}

L'effort théorique en biogéographie est importnat et 'intégration
ordonnée de concepts clés issus de différents champs de l'écologie
\cite{Thuiller2013} est une clef essentielle pour aller vers des
prédctions de Ainsi, alors que les conditions climatiques et plus
généralement la géographie physique sont classiquement évoquées pour
expliquer la répartition des espèces \cite{Kearney2004}, les
interactions entre espèces sont quant à elles souvent occultées. De
même, bien que les processus évolutifs soient souvent évoqués comme
déterminants majeurs de la diversité des espèces \cite{Rosindell2011},
leurs effets à court terme sont souvent ignorés \cite{Parmesan2006} dans
les scénarios décrivant la biodiversité de demain \cite{Lavergne2010}.
La difficulté principale est alors de produire des modèles (théoriques
en première instance) qui intègrent l'ensemble des processus et les
relations qu'ils entretient \cite{Thuiller2013} tout en gardant une
relative simplicité. Une théorie intégrative en biogéographie pourrait
être le meilleur point d'ancrage pour construire de nouvelles approches
appliquées. Avec une telle théorie en main, nous pourrions aller vers
l'enjeux majeurs de ces dernières années en biogéographie : relâcher les
hypothèses que les modèles classiques de répartitions des espèces
d'aujourd'hui utilisent (notamment en occultant les interactions) pour
prédire la biodiversité de demain \cite{Guisan2011}.

\subsection*{Comment prédire le
hasard}\label{comment-pruxe9dire-le-hasard}
\addcontentsline{toc}{subsection}{Comment prédire le hasard}

Aller vers des contriantes énergétiques mais il est dur qu'on trouvera
des règles fiables sur un système qui bien que régit par des règles
physique assez nien comprise est un moteur de stochasticé..

différentes théories pour différentes échelles ??

Quelles hypothèse pouvons nous faire sur les produits de évolution? Si
on peut supposer qu'il y a des compétition ou la règle est le changemnt
cette même propriété a-t-elle des propriétés sur le long terme. Peut-on
affirmer que les produits de l'évoluton dans un enviroemnt stable amène
à des entités qui optimise l'tilisation de l'énergie. Si oui, que dire
des produites de l'évoltion dans avec variation. Si on peut faire des
hypothèses comment les tester. Dans l'article de

Si l'évolution est imprévisible si au dela d'un certain temps on ne peut
presque rien dire\ldots{} Si la chance de des abeilles européennes
changeait comment prédire cela changement de comportement mais que nous
sommes dand l'incapacité de le prédire que pensé du status de l'écologie
et de l'évoluton en tant que science. Si la composant historique domine
le royaume de la biologie devons-nous nous dsatisfaire de le décire.
L'espoir mais la publication de Ian Hatton êut faire douter de l'absence
de l'absecmed de règel. Comment croire qu'il n'y a pas des principes
d'ordre enerétique la-dessous. Convergence\ldots{}

2014, Hurlbert et Stegen propose une série d'hypothèse pour mettre en
évidence l'impact de l'énergy sur l'évolution la troisième hypothèse est
temps suffisant pour équilibre. Une telle hypothèse une forme de
maximistaion de la production de la biomasse et l'utilisation qui est
peut être. Peut-être que les différents mécanismes en jeu dans les
processu évolutifs amène probablement à une forme de
stationarité\ldots{}

ou une compréhension des contriantes énergétiques

avoir des erreurs quantifiables

\subsection*{Contraintes
énergétiques}\label{contraintes-uxe9nerguxe9tiques}
\addcontentsline{toc}{subsection}{Contraintes énergétiques}

Moi je pars vers ça..

=\textgreater{} des interactions changer de paradigme =\textgreater{}
abstraction des espèces

\textless{}!-- Dans une théorie intégrative de la biogéographie, les
traits fonctionnels peuvent être un pivot très intéressant pour
rassembler les différents concepts que nous avons développés dans les
paragraphes précédents. Les traits peuvent tout d'abord être mis en
relation avec le milieu abiotique. Le taux métabolique ou encore la
sensibilité à la sécheresse sont des indices performant pour décrire la
survie dans un milieu donné \cite{Kearney2004,Engelbrecht2007} que l'on
peut capturer sous forme de traits. Kearney \textit{et al.} 2010 propose
une approche prometteuse dans laquelle, l'environnement physique, la
disponibilité des ressources et la dynamique énergétique sont reliées
par les traits fonctionnelles le tout aboutissant à un modèle de
distribution très mécanistes. La structure d'un réseaux peut également
être dérivée à partir de l'espace des traits. Dans leur méthode proposée
cette année, Gravel \textit{et al.} infèrent les paramètres du modèle de
niche de Williams et Martinez \cite{Williams2000} à partir des relations
de masse du corps entre proie et prédateurs \cite{Gravel2013}. Ils sont
alors en mesure de dériver un réseau global pour un ensemble d'espèce
donné. Enfin, en tant qu'expression phénotypique, les traits
fonctionnels sont soumis aux processus évolutifs. Sur les temps longs,
l'expression de l'évolution résulte en la modification progressive des
traits qui se répercute sur l'ensemble des propriétés qui en découle.
Ainsi la considération d'une modification des traits est une approche
simple et réaliste pour introduire les processus évolutifs et leurs
conséquences \cite{Guill2008,Loeuille2005}.
