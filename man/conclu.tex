\section*{Introduire davantage d'écologie des communautés dans les
modèles de distribution
d'espèce}\label{introduire-davantage-duxe9cologie-des-communautuxe9s-dans-les-moduxe8les-de-distribution-despuxe8ce}
\addcontentsline{toc}{section}{Introduire davantage d'écologie des
communautés dans les modèles de distribution d'espèce}

\subsection*{L'information des
réseaux}\label{linformation-des-ruxe9seaux}
\addcontentsline{toc}{subsection}{L'information des réseaux}

L'effort essentieel de ma thèse porté sur la caractéristion du
questionment sur les interactiosn d'espèces ma conclusio qu'il y a un
porblème d'échelle biologique.

Dans le projet ici présenté, nous proposons de construire des modèles
théoriques plus intégratifs en repartant d'un modèle théorique
classique, celui de la théorie de la biogéographie des îles proposée par
MacArthur et Wilson \cite{MacArthur1967}. Dans un premier temps, nous y
ajoutons les interactions entre espèces et une relation explicite avec
l'environnement abiotique au travers d'une approche communauté centrée
qui étend le modèle classique. Dans un second temps, nous combinons une
approche population centrée et les processus évolutifs pour une
biogéographie insulaire plus mécaniste. Enfin, au regard des enjeux que
soulève le rôle des interactions entre espèces dans la construction de
la biodiversité, nous réfléchissons sur l'inférence d'espèces
interdépendantes.

différentes théories pour différentes échelles ??

De part son pouvoir explicatif et son élégance, le modèle de MacArthur
et Wilson est un point de départ approprié pour construire des modèles
plus intégratifs en intégrant explicitement des processus écologiques et
évolutifs. Cette idée n'est pas nouvelle et les auteurs de la TIB ont
étudié un certain nombre de processus écologiques. Notamment, ils ont
intégré les phénomènes de spéciation \cite{MacArthur1967} et réfléchis
sur l'importance des interactions quant à la répartition des espèces
\cite{MacArthur1984}. Néanmoins, dans le modèle classique, l'ensemble de
ces aspects sont absents, l'idée que les processus écologiques importent
peu aux larges échelles domine. Nous allons, dans ce projet, à
l'encontre de cette idée et proposons de construire des modèles
intégratifs qui étendent la TIB.

\subsection{Interaction des différents
mécanismes}\label{interaction-des-diffuxe9rents-muxe9canismes}

L'interaction de différents L'ensemble des trois éléments jusqu'ici
évoqués (environnement abiotique, interaction, évolution) peuvent
également être étroitement associé. Grant et Grant 2006 rapportent le
cas de la compétition entre trois espèces de pinsons (dits de Darwin)
sur l'ile de Daphne (Galapagos) qui engendre une modification de la
taille de leurs becs. Cette évolution liée à la compétition est elle
même reliée à l'environnement abiotique car, par l'abondance ou
l'absence de précipitations, il détermine la disponibilité des
ressources et donc l'intensité de la compétition \cite{Grant2006}. A
travers cet exemple, nous comprenons l'importance d'inclure l'ensemble
des différents processus pour construire un modèle intégratif en
biogéographie. Un tel modèle serait capable, par exemple, de renseigner
les risques d'exclusion compétitive dans l'exemple décrit par Grant et
Grant.

\subsection*{Vers une théorie en intégrative de la
biogéographie}\label{vers-une-thuxe9orie-en-intuxe9grative-de-la-bioguxe9ographie}
\addcontentsline{toc}{subsection}{Vers une théorie en intégrative de la
biogéographie}

Dans la réédition de 2001 {[}{]} Wilson rappelle que le problème :

\begin{quote}
``The flaws of the book lie in its oversimplification and
incompleteness, which are endemic to most efforts at theory and
synthesis.''
\end{quote}

Suite à mes travaux présenté dans cette thèse, je pense avoir réellemnt
compris les lacunes théoriques de la Biogéography relevés il y a plus de
15 ans par Mark Lomolino {[}@Lomolino2000{]}. En ouverture d'un numéro
spécial dédié à la biogéographie des îles, il insistait sur le beosin
d,introduire davantage d'écologie dans la biogéogrpahie. Je me range de
son côté et de celui de différents auteurs pour voir une modèle en
biogéorgaohie articluer autor de trois composantes fondamentales de la
biogéographie: migration, extinction et évolution. Comment le reste de
l'écologue affecte et influence l'interplay de ces processus\ldots{}

L'effort théorique nécessaire en biogéographie porte sur l'intégration
ordonnée de concepts clés issus de différents champs de l'écologie
\cite{Thuiller2013}. Ainsi, alors que les conditions climatiques et plus
généralement la géographie physique sont classiquement évoquées pour
expliquer la répartition des espèces \cite{Kearney2004}, les
interactions entre espèces sont quant à elles souvent occultées. De
même, bien que les processus évolutifs soient souvent évoqués comme
déterminants majeurs de la diversité des espèces \cite{Rosindell2011},
leurs effets à court terme sont souvent ignorés \cite{Parmesan2006} dans
les scénarios décrivant la biodiversité de demain \cite{Lavergne2010}.
La difficulté principale est alors de produire des modèles (théoriques
en première instance) qui intègrent l'ensemble des processus et les
relations qu'ils entretient \cite{Thuiller2013} tout en gardant une
relative simplicité. Une théorie intégrative en biogéographie pourrait
être le meilleur point d'ancrage pour construire de nouvelles approches
appliquées. Avec une telle théorie en main, nous pourrions aller vers
l'enjeux majeurs de ces dernières années en biogéographie : relâcher les
hypothèses que les modèles classiques de répartitions des espèces
d'aujourd'hui utilisent (notamment en occultant les interactions) pour
prédire la biodiversité de demain \cite{Guisan2011}.

\subsection*{Des contraintes
physiologiques}\label{des-contraintes-physiologiques}
\addcontentsline{toc}{subsection}{Des contraintes physiologiques}

\subsection{L'abstraction des
espèces}\label{labstraction-des-espuxe8ces}

L'abstraction de l'espèce @Poisot2015 pour des questions centrales : -
quelles espèce va interagir avce qui ?\%

\subsection*{Traits fonctionnels}\label{traits-fonctionnels}
\addcontentsline{toc}{subsection}{Traits fonctionnels}

Les traits fonctionnels sont des propriétés mesurables sur les
organismes en relation avec leurs performances et leur rôle dans
l'écosystème \cite{McGill2006}. Les traits étudiés peuvent être de
différentes natures, 1-morphologiques : taille de différentes parties du
corps, position des yeux, taille des oeufs chez les organismes ovipares,
taille des graines pour les végétaux, 2- physiologiques : taux
métaboliques de bases, stœchiométrie (rapport de la concentration entre
divers éléments qui compose l'organismes)
\cite{McGill2006,Albouy2011,Litchman2008}. Un ensemble approprié de ces
propriétés peut être un outil puissant pour décrire un ensemble d'espèce
dans un même espace. Leur proximité dans l'espace des traits est alors
un indice précieux d'une proximité fonctionnelle. Ainsi, à l'aide de 13
traits ecomorphlogiques, Albouy \textit{et al.} 2011 parviennent à
prédire les guildes trophiques de 35 espèces de poissons de la
Méditerranée \cite{Albouy2011}. Edwards \textit{et al.} 2013 montrent
que l'effet saisonnier sur une communauté de phytoplancton dans la
Manche peut être capturé à l'aide de traits décrivant : le taux maximal
de croissance, la compétitivité pour la lumière et l'azote
\cite{Edwards2013}. La distribution des traits fonctionnels au sein de
la biodiversité est aussi une entrée de choix pour réfléchir quand à la
fragilité potentielle des fonctions remplies par les écosystèmes
\cite{Mouillot2013}. \%DG: je comprends cette citation de Mouillot, mais
juste une mise en garde contre ce type de référence. Mouillot se base
sur l'hypothèse que les traits nous informent du fonctionnement, sans
jamais documenter cette relation. Ce qui est souvent le cas, et par
conséquent contribue à bâtir des mythes dans la littérature qui à
l'occasion ne sont pas toujours bien appuyés. L'approche par traits est
un bel exemple, on a édifié rapidement une structure conceptuelle sur
les traits, mais on n'a pas solidement appuyé le concept sur de bonnes
bases empiriques.

L'approche de la biodiversité par les traits fonctionnels est plus
quantitative que l'approche taxonomique et permet de déduire un grand
nombre de propriétés en se passant de la connaissance de leur identité.
Ainsi McGill, dans son article d'opinion de 2006, propose une approche
nouvelle de l'écologie des communautés qui transforme les questions
centrées autour des espèces par des questions qui interrogent la
répartition et la variabilité des traits \cite{McGill2006}. L'emploi des
traits fonctionnels est en fait un appel à une écologie plus mécaniste,
qui se penche sur la physiologie des organismes, en prend les faits les
plus importants (relativement au problème traité) pour les placer dans
un espace de traits commun. Cette approche est aussi en lien avec la
controversée théorie métabolique en écologie
\cite{Brown2004, Price2012}. Dans cette théorie un certain nombre de
grandeurs (comme le taux métabolique) sont reliées à la biomasse
corporelles de l'adulte, fournissant ainsi en un seul trait de
nombreuses relations pour des groupes d'organismes très différents. Par
ces nouvelles approches, l'espérance de s'extraire de la seule identité
des espèces est accrue, l'idée d'avoir des règles générales se
concrétise.

Dans une théorie intégrative de la biogéographie, les traits
fonctionnels peuvent être un pivot très intéressant pour rassembler les
différents concepts que nous avons développés dans les paragraphes
précédents. Les traits peuvent tout d'abord être mis en relation avec le
milieu abiotique. Le taux métabolique ou encore la sensibilité à la
sécheresse sont des indices performant pour décrire la survie dans un
milieu donné \cite{Kearney2004,Engelbrecht2007} que l'on peut capturer
sous forme de traits. Kearney \textit{et al.} 2010 propose une approche
prometteuse dans laquelle, l'environnement physique, la disponibilité
des ressources et la dynamique énergétique sont reliées par les traits
fonctionnelles le tout aboutissant à un modèle de distribution très
mécanistes. La structure d'un réseaux peut également être dérivée à
partir de l'espace des traits. Dans leur méthode proposée cette année,
Gravel \textit{et al.} infèrent les paramètres du modèle de niche de
Williams et Martinez \cite{Williams2000} à partir des relations de masse
du corps entre proie et prédateurs \cite{Gravel2013}. Ils sont alors en
mesure de dériver un réseau global pour un ensemble d'espèce donné.
Enfin, en tant qu'expression phénotypique, les traits fonctionnels sont
soumis aux processus évolutifs. Sur les temps longs, l'expression de
l'évolution résulte en la modification progressive des traits qui se
répercute sur l'ensemble des propriétés qui en découle. Ainsi la
considération d'une modification des traits est une approche simple et
réaliste pour introduire les processus évolutifs et leurs conséquences
\cite{Guill2008,Loeuille2005}.

\subsection*{Des données nouvelles}\label{des-donnuxe9es-nouvelles}
\addcontentsline{toc}{subsection}{Des données nouvelles}

Comme souvent en écologie / science nous avons besoin de données, mais
ce n'est pas une qeustion vaine, L'Accumulation des données doit se
faire avec une certaine normalisation pour utiliser les. Il est souvent
difficile et la conséquence c'est de trouver des difficultés pour
réintégrer des anciennes données @Tingley2009b celle des muséum
@Shaffer1998 Malgré les espoirs des remplacer les ordinateurs pour
formuler les hypothèse, toujours besoin d'un développemnt théorique plus
de que de corrélations essayer d'estimer aujourd'hui en utilisant le
plus près possible la méthode d'hier pour savoir quel biais prbable il y
avait. Ici si on détecte beaucoup plus bas qu'avant avec la même
méthode, alors on peut se dire que le fait que ce soit des fausses
absence est faible. Par contre si on essaye d'avoir des comparaison et
que les résultats sont du à la période de l'année\ldots{} C'est plus
compliqué ! Aller vers des occupancy model

\section*{Les défis à relever dans un monde en
changement}\label{les-duxe9fis-uxe0-relever-dans-un-monde-en-changement}
\addcontentsline{toc}{section}{Les défis à relever dans un monde en
changement}

\subsection*{Une érosion de la biodiversité
affolantes}\label{une-uxe9rosion-de-la-biodiversituxe9-affolantes}
\addcontentsline{toc}{subsection}{Une érosion de la biodiversité
affolantes}

L'érosion de la biodiversité exergue une certaine nostalgie qui parfois
conduit une forme de fatalisme chez certain experts. Relevons la tête il
va falloir trouver les solutions dans le mimétisme ?

Alllant jusqu'à des porblèmes de santé La tique la souris le réservoir
et des hommes des problèmes de productions

Sommes nous en train de biaisé le signal phylogénétique ? (cf article
Thuillier)

\subsection{Avons-nous des espoirs vains
?}\label{avons-nous-des-espoirs-vains}

Le royaume de la contingence du à l'impact historique de l'histoire
evolutive. Alors comment finder des espoinr de généralité quand le
moteur repose sur de la stochasticité Mais cette loi mène à des
prédictions exoecologie Les bactéries mais comment généraliser alors que
l'evolution à afit émerger bon nombre d'organisme qui en soi loin
quoique complèlemnt immbriqué on a plus de miro-organimes que de
cellules\ldots{}

inertie historique comment imaginer des plantes sans mycorrhyze mais
d'autres systeme auraiengt pu marcher En fait quand on pense à la plante
don pense à lMunité de lante + mycorrhuze et quand on pense à un
vertébrés on inclu tout ces bactérie on ne peut certes pas comprendre
comment l'un marche sans l'autre mais pour on a pas besoin de tout
connaître c'est un problème de rupture de symétrie.

Les conséquences sont compliqués des changements climatiques sont
nombreuses et certaines espèce voir le range grandir d'autre diminuer
pour cds espèce de co existent et donc à un changemnet prononc. de al
morphologoe des communautés alors que le nombre d'espèce peut être peu
affecté @Moritz2008

On nous fait miroiter que finalement que l'érosion de la biodiversité
est dramatiques et le ressort actuel pour faire un levier face à cela
c'est les services ecosystémiques qui sont actuelelemet l'argument choc
pour renforcer la production de la nature. Il y a un côté pervers qui
est la financiarisation et la substituabilité l'argent oeut alors être
utilisée pour intervertir ou alors remplacer un type d'écisystème par un
autre ailleurs\ldots{} En fait on a l'impressonq ue c'est pus un
principe de précaution qui erst invoquer et ultimement il est
vraisemblable que la destruction de la nature tel que nous la
connaissons soit dans le future un générateur de conflit\ldots{}. et
uttiment on a a craindre de faire un panete invivable pour nous mêm.
Mais les changement sont des remplacemnt et pour la conservation on peut
se demander les startégie. Dans son arctile `Don't juge a species on
their origin' Mark Davis prend à revers un sertain nombre d'idée recu et
souligne qye les effects des invedeurs peuvent être positives
@Davis2011.

\subsection*{Des écosystèmes
bouleversés}\label{des-uxe9cosystuxe8mes-bouleversuxe9s}
\addcontentsline{toc}{subsection}{Des écosystèmes bouleversés}

Reconfiguration des écosystèmes naturelle li y a eu d'autre crise avant.
Finalemnt avec du catastrophisme, la question s'est si nous on ira mal.
On est grons pour la taille de la planète peut être plus suceptibel è
l'extinciton que l'on pense. Maid ce ,est pas le pessismise qui
m'importe. - Et si on faisait rien pour le frelon asiatique ?

\subsection*{Transients}\label{transients}
\addcontentsline{toc}{subsection}{Transients}

\section*{Vers une écologie
prédictive?}\label{vers-une-uxe9cologie-pruxe9dictive}
\addcontentsline{toc}{section}{Vers une écologie prédictive?}

\subsection*{Un espoir}\label{un-espoir}
\addcontentsline{toc}{subsection}{Un espoir}

La défense des modèle climqtaiur bioclimate enveloppe de Pearson comme
une dpremière approximation utilise se faitt sur 3 exemple de plantes
@Pearson2003

\subsection*{Catégories}\label{catuxe9gories}
\addcontentsline{toc}{subsection}{Catégories}

L'écologie ne traite pas de telle ou telle manière les différentes
espèces. Il y a des champs entier dédier à des classes d'espèces. Par
exemple, on traite mcro-organisme ou metaphyte vs.~metazoaire. Les
échelles et la proximité nous biaise fortement la vision de la
biodiversité. Il est difficile d'appliquer les théories à l'ensemble des
espèces, peut être seulement la théorie de l'évolution mais se qui est
intéressant c'est que ce processus est finalemnt l'essence de la vie.
L'aphrisne d e Dobzansky à mon avis ne devrait simpemtn dire que la
biologie est à regarder à la lumière mais que le vivant est une ensemble
dde moécule organique dont l'organisation est soumise à l'évolution. Il
existe des sepécificités des êtres.

Malgré les apparences, La TIB n'est pas formulée pour l'ensemble des
espèces. Le premier exemple du livre herpetofaune puis les fourmis de la
famille des Ponerinae {[}@MacArthur1967{]}. De même la théorie neutre a
été classique tropical les arbres dans els forêts tropicaux et les
coraux. Biensur ces théories sont liées à la connaissance fine mais
biaisée du vivant (comment avoir une connaissance exhastive du
vivant\ldots{}). Aisin, certaine théorie s'applique à certaine partie du
monde vivant, il y aurait donc une classification à faire ou une
compréhension du côté de pourquoi telle ou telle propriété est Ok pour
telle ou telle partie diu vivant. Cela conduit à une intérogation sur
les règles possibles de composition des écosystèmes.

Wallace n'aurait-il pas eu plus de mal à comprndre les zones
aujourd'hui. Si naïvemnt on réduit aux villes, l'homogénéité ++ mais
avec les espèces invasive le signal est fortemnt briollé aussi ! Je
pense qu'on est a un tournant de la biogoe vers un chamgemnt de
paradigme communaité centré qui ne nit pas les travaux précédant mais
les suit.

Information dans les distributions gecko australien généraliste
\emph{Heteronotia binoei} =\textgreater{}~alors peut être que ça marche
bien mais sur une espèce spécialiste ??

\subsection*{Les produits de
l'évolution}\label{les-produits-de-luxe9volution}
\addcontentsline{toc}{subsection}{Les produits de l'évolution}

Quelles hypothèse pouvons nous faire sur les produits de évolution? Si
on peut supposer qu'il y a des compétition ou la règle est le changemnt
cette même propriété a-t-elle des propriétés sur le long terme. Peut-on
affirmer que les produits de l'évoluton dans un enviroemnt stable amène
à des entités qui optimise l'tilisation de l'énergie. Si oui, que dire
des produites de l'évoltion dans avec variation. Si on peut faire des
hypothèses comment les tester. Dans l'article de

Si l'évolution est imprévisible si au dela d'un certain temps on ne peut
presque rien dire\ldots{} Si la chance de des abeilles européennes
changeait comment prédire cela changement de comportement mais que nous
sommes dand l'incapacité de le prédire que pensé du status de l'écologie
et de l'évoluton en tant que science. Si la composant historique domine
le royaume de la biologie devons-nous nous dsatisfaire de le décire.
L'espoir mais la publication de Ian Hatton êut faire douter de l'absence
de l'absecmed de règel. Comment croire qu'il n'y a pas des principes
d'ordre enerétique la-dessous. Convergence\ldots{}

2014, Hurlbert et Stegen propose une série d'hypothèse pour mettre en
évidence l'impact de l'énergy sur l'évolution la troisième hypothèse est
temps suffisant pour équilibre. Une telle hypothèse une forme de
maximistaion de la production de la biomasse et l'utilisation qui est
peut être. Peut-être que les différents mécanismes en jeu dans les
processu évolutifs amène probablement à une forme de
stationarité\ldots{}

\subsection*{Quels types de prédictions pouvons-nous
faire?}\label{quels-types-de-pruxe9dictions-pouvons-nous-faire}
\addcontentsline{toc}{subsection}{Quels types de prédictions
pouvons-nous faire?}

Quels objet qu'est ce qui peut faire l'obejt d'une prédiction / qu'est
ce qui ne peut pas?

\subsubsection{Une question d'échelle?}\label{une-question-duxe9chelle}

L'écologie porte sur l'ensemble du monde vivant quelques soiten leur
taille mais les différent champs ne sont pas toutes relatoves à la
m\^{}me échelle alors il y a bien els échelles de temps, les echelles
spatiales mais il y a le lével d'organisation. Il est bien inportant de
comprendre cela !

Un scéhma avec des variables qui émergenet ave différemts paramères et
quelques éxemelpme de théorie! (DEB Evolution foodweb\ldots{}) et
l'action de

Repartition des especes des passges histroqiere dans l'origin des
espèces et dans Wallace. Le principe même de l'écologie (la definition
de ecologie).On arrive à l'idée de ;la niche. Exemple histriques. Dans
son ouvrage, le grand biogéographe Wallace reconait en introduction le
caractère facinant de la réaortition de la biodiversité des îles avec
des faot intriguant wuant à la faune et la flore. Ainsi il constate
qu'il peut y avir plus deux différence entre île très éloigné et deux
île s très proche. Il écrit que la faune et la flore sont plus
dissimilaire entre ldeles deux piles des Galapagos Bali et Lombik
qu'entre Hokaido (Yesso) et La grand bretagne ouy encore la Nouvelle
Zéland et l'Australie,

Exemple classique de grinnel et des Trasher + evolution avec les
charcter displacement.

Nous accumulons des évidences quand aux impact du changement
anthropique. A diiférentes échelles la diminution de la biodiversité,
changemnt en compoisiton @Taranu2015 @DeRoos2008

La biogéographie avec au moins 3 problèmes d'échelles =\textgreater{}
spatiale =\textgreater{} temporelle plus on augmente plus l'enpreinte
historiques est forte =\textgreater{} grands evenemnt géologique
(lacitaion mouvement des plques) biogéogrpahies historiqyes mais aussi
forme un pool d'espèces =\textgreater{} Mais aussi l'échelle taxonomique
: la relaton aire espèce est décrite à l'intérieru des taxons les
relations allométriques à l'inérieur des taxons E O Wilson a commencé à
rappporter des relation sur les formis les exemples du livre sont
herpeta faun (reptile plus amphibien) mecanisme =\textgreater{}
diversité de milieu

\subsubsection{Prédire des
communautés}\label{pruxe9dire-des-communautuxe9s}

\subsection*{Les dangers d'aller trop
vite}\label{les-dangers-daller-trop-vite}
\addcontentsline{toc}{subsection}{Les dangers d'aller trop vite}

\begin{quote}
There is also a danger that predictions grow faster than our
understanding of ecological systems, resulting in a gap between the
scientists generating the predictions and stakeholders using them
\end{quote}

``Predictive ecology in a changing world'' {[}@Mouquet2015{]}

\subsection{Contraintes
énergétiques}\label{contraintes-uxe9nerguxe9tiques}

Moi je pars vers ça..

=\textgreater{} des interactions changer de paradigme =\textgreater{}
abstraction des espèces
