\section*{Que peut-on prédire ?}\label{que-peut-on-pruxe9dire}
\addcontentsline{toc}{section}{Que peut-on prédire ?}

Quantifié les flux les fréquences aléatoire et contraintres les
extinctions

Questionner des modèle comme ceux prennet tout pour prédire \ldots{}.
{[}@Harfoot2014{]}

\section{Catégorie}\label{catuxe9gorie}

Il existe en fait une catégoristaion non déclarée par un de fait. La TIB
bne s'intéresse pas au île auand MacArthur s'intéresse au arbres, il en
fait des îlesm un cadre de font. Pourrions nous avoir des groupes
neutres (coraux arbres) et d,utre pas (OK pour théories neutres), si
findé sur un poinds relatifs des processus c'est super good ! Théorie
sur la composition ou un trait ! mais il fait savoir ce qui chnage la
compositions des traits !! On ignore le mycorrhix il y aura très
probablement un résaeux de cette tforme la avec telle conséquence
economique

Et si on faisait rien pour le frelo et si le comportemnt des abeilles
européennes changeait comment prédire cela\ldots{}

\section{Quelles type de prédicions pouvons nous faire
?}\label{quelles-type-de-pruxe9dicions-pouvons-nous-faire}

Bon objet et ce qui la concerne ou pas

problème de reflexion sur l'unité pertinente

des convergences contraintes physiques / intellegicence pour se
soustraire à la niche.. / la chitine

\subsection{Une question d'échelle}\label{une-question-duxe9chelle}

L'écologie porte sur l'ensemble du monde vivant quelquees soiten leur
taille mais les différent champs ne sont pas toutes relatoves à la
m\^{}me échelle alors il y a bien els échelles de temps, les echelles
spatiales mais il y a le lével d'organisation. Il est bien inportant de
comprendre cela !

Un scéhma avec des variables qui émergenet ave différemts paramères et
quelques éxemelpme de théorie! (DEB Evolution foodweb\ldots{}) et
l'action de

Repartition des especes des passges histroqiere dans l'origin des
espèces et dans Wallace. Le principe même de l'écologie (la definition
de ecologie).On arrive à l'idée de ;la niche. Exemple histriques. Dans
son ouvrage, le grand biogéographe Wallace reconait en introduction le
caractère facinant de la réaortition de la biodiversité des îles avec
des faot intriguant wuant à la faune et la flore. Ainsi il constate
qu'il peut y avir plus deux différence entre île très éloigné et deux
île s très proche. Il écrit que la faune et la flore sont plus
dissimilaire entre ldeles deux piles des Galapagos Bali et Lombik
qu'entre Hokaido (Yesso) et La grand bretagne ouy encore la Nouvelle
Zéland et l'Australie,

Exemple classique de grinnel et des Trasher + evolution avec les
charcter displacement.

Nous accumulons des évidences quand aux impact du changement
anthropique. A diiférentes échelles la diminution de la biodiversité,
changemnt en compoisiton @Taranu2015 @DeRoos2008

La biogéographie avec au moins 3 problèmes d'échelles =\textgreater{}
spatiale =\textgreater{} temporelle plus on augmente plus l'enpreinte
historiques est forte =\textgreater{} grands evenemnt géologique
(lacitaion mouvement des plques) biogéogrpahies historiqyes mais aussi
forme un pool d'espèces =\textgreater{} Mais aussi l'échelle taxonomique
: la relaton aire espèce est décrite à l'intérieru des taxons les
relations allométriques à l'inérieur des taxons E O Wilson a commencé à
rappporter des relation sur les formis les exemples du livre sont
herpeta faun (reptile plus amphibien) mecanisme =\textgreater{}
diversité de milieu

\subsection{Des classes d'esèces ?}\label{des-classes-desuxe8ces}

Wallace n'aurait-il pas eu plus de mal à comprndre les zones
aujourd'hui. Si naïvemnt on réduit aux villes, l'homogénéité ++ mais
avec les espèces invasive le signal est fortemnt briollé aussi !

Je pense qu'on est a un tournant de la biogoe vers un chamgemnt de
paradigme communaité centré qui ne nit pas les travaux précédant mais
les suit.

La défense des modèle climqtaiur bioclimate enveloppe de Pearson comme
une dpremière approximation utilise se faitt sur 3 exemple de plantes
@Pearson2003

\subsection{Prédire des
communautés}\label{pruxe9dire-des-communautuxe9s}

=\textgreater{} des interactions changer de paradigme

On nous fait miroiter que finalement que l'érosion de la biodiversité
est dramatiques et le ressort actuel pour faire un levier face à cela
c'est les services ecosystémiques qui sont actuelelemet l'argument choc
pour renforcer la production de la nature. Il y a un côté pervers qui
est la financiarisation et la substituabilité l'argent oeut alors être
utilisée pour intervertir ou alors remplacer un type d'écisystème par un
autre ailleurs\ldots{} En fait on a l'impressonq ue c'est pus un
principe de précaution qui erst invoquer et ultimement il est
vraisemblable que la destruction de la nature tel que nous la
connaissons soit dans le future un générateur de conflit\ldots{}. et
uttiment on a a craindre de faire un panete invivable pour nous mêm.
Mais les changement sont des remplacemnt et pour la conservation on peut
se demander les startégie. Dans son arctile `Don't juge a species on
their origin' Mark Davis prend à revers un sertain nombre d'idée recu et
souligne qye les effects des invedeurs peuvent être positives
@Davis2011.

\section*{Vers une biogéographie
intégrative}\label{vers-une-bioguxe9ographie-intuxe9grative}
\addcontentsline{toc}{section}{Vers une biogéographie intégrative}

\subsection{Les données}\label{les-donnuxe9es}

Comme souvent en écolog/ie / science nous avons besoin de données, mais
ce n'est pas une qeustion vaine, L'Accumulation des données doit se
faire avec une certaine normalisation pour utiliser les. Il est souvent
difficile et la conséquence c'est de trouver des difficultés pour
réintégrer des anciennes données @Tingley2009b celle des muséum
@Shaffer1998 Malgré les espoirs des remplacer les ordinateurs pour
formuler les hypothèse, toujours besoin d'un développemnt théorique plus
de que de corrélations essayer d'estimer aujourd'hui en utilisant le
plus près possible la méthode d'hier pour savoir quel biais prbable il y
avait. Ici si on détecte beaucoup plus bas qu'avant avec la même
méthode, alors on peut se dire que le fait que ce soit des fausses
absence est faible. Par contre si on essaye d'avoir des comparaison et
que les résultats sont du à la période de l'année\ldots{} C'est plus
compliqué ! Aller vers des occupancy model

\subsection{L'abstraction des
espèces}\label{labstraction-des-espuxe8ces}

\subsection*{Traits fonctionnels}\label{traits-fonctionnels}
\addcontentsline{toc}{subsection}{Traits fonctionnels}

Les traits fonctionnels sont des propriétés mesurables sur les
organismes en relation avec leurs performances et leur rôle dans
l'écosystème \cite{McGill2006}. Les traits étudiés peuvent être de
différentes natures, 1-morphologiques : taille de différentes parties du
corps, position des yeux, taille des oeufs chez les organismes ovipares,
taille des graines pour les végétaux, 2- physiologiques : taux
métaboliques de bases, stœchiométrie (rapport de la concentration entre
divers éléments qui compose l'organismes)
\cite{McGill2006,Albouy2011,Litchman2008}. Un ensemble approprié de ces
propriétés peut être un outil puissant pour décrire un ensemble d'espèce
dans un même espace. Leur proximité dans l'espace des traits est alors
un indice précieux d'une proximité fonctionnelle. Ainsi, à l'aide de 13
traits ecomorphlogiques, Albouy \textit{et al.} 2011 parviennent à
prédire les guildes trophiques de 35 espèces de poissons de la
Méditerranée \cite{Albouy2011}. Edwards \textit{et al.} 2013 montrent
que l'effet saisonnier sur une communauté de phytoplancton dans la
Manche peut être capturé à l'aide de traits décrivant : le taux maximal
de croissance, la compétitivité pour la lumière et l'azote
\cite{Edwards2013}. La distribution des traits fonctionnels au sein de
la biodiversité est aussi une entrée de choix pour réfléchir quand à la
fragilité potentielle des fonctions remplies par les écosystèmes
\cite{Mouillot2013}. \%DG: je comprends cette citation de Mouillot, mais
juste une mise en garde contre ce type de référence. Mouillot se base
sur l'hypothèse que les traits nous informent du fonctionnement, sans
jamais documenter cette relation. Ce qui est souvent le cas, et par
conséquent contribue à bâtir des mythes dans la littérature qui à
l'occasion ne sont pas toujours bien appuyés. L'approche par traits est
un bel exemple, on a édifié rapidement une structure conceptuelle sur
les traits, mais on n'a pas solidement appuyé le concept sur de bonnes
bases empiriques.

L'approche de la biodiversité par les traits fonctionnels est plus
quantitative que l'approche taxonomique et permet de déduire un grand
nombre de propriétés en se passant de la connaissance de leur identité.
Ainsi McGill, dans son article d'opinion de 2006, propose une approche
nouvelle de l'écologie des communautés qui transforme les questions
centrées autour des espèces par des questions qui interrogent la
répartition et la variabilité des traits \cite{McGill2006}. L'emploi des
traits fonctionnels est en fait un appel à une écologie plus mécaniste,
qui se penche sur la physiologie des organismes, en prend les faits les
plus importants (relativement au problème traité) pour les placer dans
un espace de traits commun. Cette approche est aussi en lien avec la
controversée théorie métabolique en écologie
\cite{Brown2004, Price2012}. Dans cette théorie un certain nombre de
grandeurs (comme le taux métabolique) sont reliées à la biomasse
corporelles de l'adulte, fournissant ainsi en un seul trait de
nombreuses relations pour des groupes d'organismes très différents. Par
ces nouvelles approches, l'espérance de s'extraire de la seule identité
des espèces est accrue, l'idée d'avoir des règles générales se
concrétise.

Dans une théorie intégrative de la biogéographie, les traits
fonctionnels peuvent être un pivot très intéressant pour rassembler les
différents concepts que nous avons développés dans les paragraphes
précédents. Les traits peuvent tout d'abord être mis en relation avec le
milieu abiotique. Le taux métabolique ou encore la sensibilité à la
sécheresse sont des indices performant pour décrire la survie dans un
milieu donné \cite{Kearney2004,Engelbrecht2007} que l'on peut capturer
sous forme de traits. Kearney \textit{et al.} 2010 propose une approche
prometteuse dans laquelle, l'environnement physique, la disponibilité
des ressources et la dynamique énergétique sont reliées par les traits
fonctionnelles le tout aboutissant à un modèle de distribution très
mécanistes. La structure d'un réseaux peut également être dérivée à
partir de l'espace des traits. Dans leur méthode proposée cette année,
Gravel \textit{et al.} infèrent les paramètres du modèle de niche de
Williams et Martinez \cite{Williams2000} à partir des relations de masse
du corps entre proie et prédateurs \cite{Gravel2013}. Ils sont alors en
mesure de dériver un réseau global pour un ensemble d'espèce donné.
Enfin, en tant qu'expression phénotypique, les traits fonctionnels sont
soumis aux processus évolutifs. Sur les temps longs, l'expression de
l'évolution résulte en la modification progressive des traits qui se
répercute sur l'ensemble des propriétés qui en découle. Ainsi la
considération d'une modification des traits est une approche simple et
réaliste pour introduire les processus évolutifs et leurs conséquences
\cite{Guill2008,Loeuille2005}.

L'abstraction de l'espèce @Poisot2015 pour des questions centrales : -
quelles espceace av interagir avce qui ?\% Une chance pour voir des
communautés chnager et des communatés compltement affecté et en tirer
des conclusion ou alors le contraire des inférences des règles
valableque dans les milieux perturbés\ldots{} qui ont leur
règles\ldots{}

\subsection*{des prédictions fiables?}\label{des-pruxe9dictions-fiables}
\addcontentsline{toc}{subsection}{des prédictions fiables?}

\subsection*{Les dangers d'aller trop
vite}\label{les-dangers-daller-trop-vite}
\addcontentsline{toc}{subsection}{Les dangers d'aller trop vite}

\begin{quote}
There is also a danger that predictions grow faster than our
understanding of ecological systems, resulting in a gap between the
scientists generating the predictions and stakeholders using them
\end{quote}

``Predictive ecology in a changing world'' {[}@Mouquet2015{]}

\section*{Un monde en changement : entre espoir et
illusion}\label{un-monde-en-changement-entre-espoir-et-illusion}
\addcontentsline{toc}{section}{Un monde en changement : entre espoir et
illusion}

\subsection*{Une érosion de la biodiversité
affolantes}\label{une-uxe9rosion-de-la-biodiversituxe9-affolantes}
\addcontentsline{toc}{subsection}{Une érosion de la biodiversité
affolantes}

L'érosion de la biodiversité exergue une certaine nostalgie qui parfois
conduit une forme de fatalisme chez certain experts. Relevons la tête il
va falloir trouver les solutions dans le mimétisme ?

Alllant jusqu'à des porblèmes de santé La tique la souris le réservoir
et des hommes des problèmes de productions

\subsection*{Un monde biaisé?}\label{un-monde-biaisuxe9}
\addcontentsline{toc}{subsection}{Un monde biaisé?}

Sommes nous en train de biaisé le signal phylogénétique ? (cf article
Thuillier)

\subsection{Avons-nous des espoirs vains
?}\label{avons-nous-des-espoirs-vains}

Le royaume de la contingence du à l'impact historique de l'histoire
evolutive. Alors comment finder des espoinr de généralité quand le
moteur repose sur de la stochasticité Mais cette loi mène à des
prédictions exoecologie Les bactéries mais comment généraliser alors que
l'evolution à afit émerger bon nombre d'organisme qui en soi loin
quoique complèlemnt immbriqué on a plus de miro-organimes que de
cellules\ldots{}

inertie historique comment imaginer des plantes sans mycorrhyze mais
d'autres systeme auraiengt pu marcher En fait quand on pense à la plante
don pense à lMunité de lante + mycorrhuze et quand on pense à un
vertébrés on inclu tout ces bactérie on ne peut certes pas comprendre
comment l'un marche sans l'autre mais pour on a pas besoin de tout
connaître c'est un problème de rupture de symétrie.

Les conséquences sont compliqués des changements climatiques sont
nombreuses et certaines espèce voir le range grandir d'autre diminuer
pour cds espèce de co existent et donc à un changemnet prononc. de al
morphologoe des communautés alors que le nombre d'espèce peut être peu
affecté @Moritz2008

\subsection{DEB}\label{deb}

Le travail de Gotelli \textit{et al.} est également un exemple de
démarche intégrative où un nombre important de processus peuvent être
inclus via un système de combinaison de scénarios et tester par
simulations stochastiques \cite{Gotelli2009}. Enfin, en construisant des
réseaux basés sur la cooccurrence des espèces, Araújo \textit{et al.}
revisitent le problème de l'interdépendance des espèces
\cite{Araujo2011} : ils s'interrogent sur la résistance des réseaux de
cooccurrence obtenus face aux futurs changement climatiques, ils mettent
ainsi en évidence des risques accrus de perte des espèces les moins
connectés (celles qui cooccurent moins). Ces travaux témoignent de la
volonté d'une biogéographie intégrative.

C'est impressionnant de voir comment un auteur en repartant de simple
considération telle que la taile le volume peut arriver à construire une
théorie à la fois simple, fondée et predictive. mettant de la cohérence
dansune accumulation de fait.

=\textgreater{} problème SDMS quand inférencefait sur les données
d'espèces la force c'est d'avoir des mesures ++ et indépendante quelquee
part c'est vrai mais la source d'inforation est très brouillé et on peut
se demander se que l'on peut obtenir comme infornation\ldots{}.

Nous contraignons énormément les ranges d'espèces alors nous sort de
tout ça\ldots{}

L'ajout des interactions dans un modèle incluant l'environnement
abiotique interroge la relation que les deux processus entretiennent. Si
les espèces n'ont pas les mêmes performances dans différents milieux du
fait de leur physiologie, pour les mêmes espèces considérées, les
réseaux n'ont pas de raison d'être identiques d'un milieu à un autre.
C'est sur ce fait que Poisot \textit{et al.} 2012 ont proposé une mesure
de dissimilarité des réseaux \cite{Poisot2012}. Defossez \textit{et al.}
montrent que les interactions négatives entre l'hêtre commun
(\textit{Fagus Sylvaitca}) et les micro-organismes du sol diminuent avec
l'altitude \cite{Defossez2011}. Ainsi, les contraintes biotiques sont à
relier à l'environnement \cite{Brooker2006,Canham2006} et un modèle
intégratif doit donner un cadre cohérent à ces rétroactions entre
processus. Enfin, l'importance des interactions est à mettre en relation
avec l'échelle considérée \cite{Peterson2011}. Pour deux espèces en
interaction, plus l'échelle d'étude est large, moins les effets des
interactions locales sont susceptibles d'être capturés, le pouvoir
explicatif de la présence d'une espèce sur l'autre peut être alors
discutable \cite{Araujo2007}. Comprendre quels sont les processus à
prendre en compte aux différentes échelles spatio-temporelles et
comprendre comment le changement d'échelle affecte nous prédictions est
aussi un véritable challenge en biogéographie \cite{Martinez2012}.

A large espes répartition de la biodiversité on quantifie la différence
depuis les mesures classiques. Simpson, alpha gamma beta qui sont
étendues au réseau @Poisot2012. Mais quand on chnage d'echelle on arrive
rarement à quelques choses de concluant pour l'integration des
interactions. Pourtant il ya des exemples convaicant comme celui de
Gitelli.

Le travail le plus dur devient d'utiliser un ensmeble de connaisance pur
déterminer des cartes de zone à risuqe mais la qualité des cartes est
théorie dépendant mais comprendre eavce pus de fiabiité les prochaines
zone ou \emph{Vespa} sera\ldots{}

Information dans les distributions gecko australien généraliste
\emph{Heteronotia binoei} =\textgreater{}~alors peut être que ça marche
bien mais sur une espèce spécialiste ??
