\section*{Intéractions écologiques et distribution des
espèces}\label{intuxe9ractions-uxe9cologiques-et-distribution-des-espuxe8ces}
\addcontentsline{toc}{section}{Intéractions écologiques et distribution
des espèces}

\subsection*{Des modèles théoriques à
développer}\label{des-moduxe8les-thuxe9oriques-uxe0-duxe9velopper}
\addcontentsline{toc}{subsection}{Des modèles théoriques à développer}

Dans son appel pour un renouvellement de la théorie de la biogéographie
des îles, Mark Lomolino soulignait le besoin d'intégrer davantage de
processus écologiques et évolutifs autour des trois processus
fondamentaux de la biogéographie : colonisation, extinction et évolution
\citep{Lomolino2000}. Au chapitre \ref{chap1}, je me suis confronté
frontalement à ce problème en proposant une démarche pour incorporer les
interactions et les contraintes abiotiques dans la TIB. J'ai proposé
d'utiliser un cadre mathématique général reliant ces facteurs aux
probabilités d'extinction et de colonisation. Il me semble que cette
approche est simple et prometteuse, je la considère aujourd'hui comme
une extension de la TIB en ce sens que la théorie classique devient une
cas particulier : les interactions et les contraintes abiotiques n'ont
pas d'influence sur les taux de colonisation et d'extinction.

Le modèle proposé au chapitre \ref{chap1} suggère aussi le potentiel de
réfléchir en terme de probabilité d'assemblage \citep{Cazelles2015a}
plutôt que de considérer les espèces individuellement. A la lecture du
livre de Robert MacArthur publié en 1972, j'ai ressenti que cette idée
était présente mais pas formulalisée \citep{macarthur1972geographical}.
Partir des assemblages pour comprendre les présence d'espèces posent un
porblème technique très concret: le nombre de communautés à envisager
devient rapidement très important (pour \(n\) espèces, ce sont \(2^n\)
assemblages possibles), ce qui limite la mise en application du modèle
développé au chapitre \ref{chap1} dans sa formulation actuelle. Cela
étant dit, il est possible que des moyens émergent pour réduire en
compléxité et qu'il soit progressivement transformé en une méthode
d'inférence statique efficace.

En me confrontant à l'incorporation des intéractions écologiques dans la
TIB (chapitre \ref{chap1}), je me suis aperçu à quel point il est
délicat de construire des modèles simples, élégants expliquant à un
grand nombre de faits. En conséquences, je ne suis pas étonné que la TIB
soit toujours abondamment utilisée comme point de départ de nombreuses
études \citep{Warren2015} et cela en dépit de ces défauts reconnus dans
la ré-édition de 2001 de \emph{The Theory of Island Biogeography} par
Edward 0. Wilson lui même :

\begin{quote}
The flaws of the book lie in its oversimplification and incompleteness,
which are endemic to most efforts at theory and synthesis.
\end{quote}

L'objet \emph{interactions} n'est pas simple à manipuler : à l'échelle
de la communauté, les interactions ne peuvent pas être traitées
isolémment, elles forment des réseaux. Il y a un champ de la
mathématique entier dédié à l'étude de tels objets appelé graphes et la
théorie qui traite de ces objets est utilisée pour apprhender des
réseaux de toutes sortes qu'ils soient sociaux ou neuronnaux. D'autres
champs de la biologie utilisent ces objets, les neurosciences par
exemple, et pointent également les difficultés à comprendre les systèmes
caractérisés par l'interdépendence de ses unités. L'écologie des réseaux
bénéficient très directement des travaux de mathématiciens et de
physiciens dont elles retirent des outils performants et de plus en plus
pointus, ce qui ajoute à la difficulté du sujet une complexité
technique.

En parallèle des questionnements très précis que soulèvent différents
champs de l'écologie, il me semble également important que des
réflexions soient menées pour aller vers des modèles plus intégratifs.
Une part importante de l'effort doit être dédiée à des approches
simplifiées et davantage intégratives. En forçant un peu le trait, en
écologie nous avons d'un côté des modèles qui avec très peu de
populations considérées engendrent des dynamiques complexes voir
chaotiques \citep[dont l'existence est validée
expérimentalement][]{Costantino1997b, Fussmann2000} et de l'autre des
modèles comme celui de la TIB qui, avec une équation différentielle
simple, propose une vision profonde de la biogéographie
\citep{MacArthur1967}. Je pense qu'il qu'il est tout aussi pertient
d'essayer de partir de l'échelle la plus large pour aller vers des
échelles plus petites que de mener la démarche inverse. Il est par
ailleurs tout aussi possible que les deux objets finaux à prédire
c'est-à-dire l'abondance relative de populations en interaction et la
composition spécifique sur des larges échelles spatio-temporelles ne
puissent être prédits de la même façon \citep[ce qui serait une forme de
\emph{rupture de symétrie}][]{Anderson1972}. Quoi qu'il en soit, c'est
bien en essayant d'utilser la première approche que j'ai mieux cerné
quelles pouvaient être les traces laissées par les interactions
écologiques sur les distributons d'espèces.

\subsection*{Des théories pour mieux apréhender les données de
co-occurrence}\label{des-thuxe9ories-pour-mieux-apruxe9hender-les-donnuxe9es-de-co-occurrence}
\addcontentsline{toc}{subsection}{Des théories pour mieux apréhender les
données de co-occurrence}

Le chapitre \ref{chap2} bien que théorique, est un pas important en
direction de l'application de mes reflexions à des données empiriques.
Il y est en effet question de données de co-occurrence et de réseaux
écologiques. Les données d'occurrence constituent une source de
réflexion importante pour les biogéographes sur lequel se concentre
l'effort de développement méthodologique du domaine
\citep{Elith2006, Phillips2006, Pollock2014}. Les données de
co-occurrence sont issues de la considération simultannée de données
d'occurence de plusieurs espèces sur un même gradient biogéographique.
Exploiter ces données permet, par exemple, d'envisager la structure des
assemblages de demain \citep{Albouy2012}. En proposant une réflexion de
l'impact des interactions écologiques sur les données de co-occurrence,
j'ai essayé d'améliorer la compréhension de la nature de l'information
que pouvaient contenir les données de co-occurrence. Ce travail de
compréhension du lien qu'il existe entre les processus écologiques et
les données de distributions analysées est crucial pour orienter le
développement des outils sur lesquels sont construits les scénarios de
changement de la biodiversité. De manière générale, il s'agit de
comprendre du lien qu'il existe entre la distribution d'une espèce et sa
niche hutchinsonienne \citep{Pulliam2000, Godsoe2010a}. En utilisant un
modèle de probabilité simple et la version trophique de la TIB
\citep{Gravel2011}, j'ai découvert des attentes théoriques précises sur
les données de co-occurrence et j'ai montré que l'empreinte des
interactions écologiques sur les données de co-occurrence n'est pas
appréciable notament lorsque les intéractions sont nombreuses.

L'article présenté au chapitre \ref{chap3} proposent de tester la
théorie du chapitre \ref{chap2}. En analysant des données de co-occurece
pour des systèmes dont les interactions étaient documentées, j'ai montré
que celles-ci laissent des traces visibles dans les données statiques de
co-occurrence. La détection de signaux de co-occurrence imputables aux
lien écologiques liant les espèces est cepandant possible que sous
certaines conditions: lorsque les espèce interagissent directement,
lorsque le nombre d'interactions est limité. De manière cohérente, la
distribution d'un prédateur spécialiste est très corrélée avec celle de
sa proie alors qu'un prédateur généraliste voit sa distribution
partiellement corrélée avec un grand nombre de distributions, celle de
ces proies, ce qui rend difficile la présence d'un signal clair dans la
co-occurrence d'un généraliste avec une de ces proies. Un signal peut
néanmoins exister lorsque l'on éxamine la corrélation de la distribution
de ce prédateur et la répartition géographique jointe de l'ensemble de
ces proies.

En travaillant sur les co-occurrences avec des données de distribution
d'espèce en interaction, j'ai aussi pointé du doight un problème
important de l'application des SDMs au regard des réseaux écologiques.
La co-occurrence forte de deux espèces est souvent interprétée comme le
temoin de la similarité de leurs besoins physiologiques, ce qui justifie
d'utiliser des projections à l'échelle de l'espèce pour prédire des
communautés \citep{Rehfeldt2006, Albouy2012}. Cela dit, en partant de ce
principe là, lorsque l'on prend pour espace explicatif seulement les
variables abiotiques, l'occurrence des espèces seulement des variables
climatiques, il est vraismblable que nous capturions une part de
l'impact des interactions dans la distributon sans pour autant le voir.
Nous avons montré au chapitre \ref{chap3} que l'utilisation de SDMs pour
obtenir des co-occurrences intégrant les contraintes abiotiques
affaiblissait considérablement le signal observé sur les données de
co-occurrence brutes. L'interprétation immédiate consiste à dire que la
co-occurrence est conrtainte par le variables pédo-climatiques
abiotiques. Néanmoins, le fait que même les associations les plus fortes
(pour les prédateurs spécialistes et leur proie) disparaissent et qu'un
modèle simple basé sur la présence de proies soit plus performant que
certains SDM, semble indiqué qu'une portion de l'effet des interactions
et comem nous ne sommes pas en mesure de connaître précisemment cette
part, il se peut que l'association soit pas très bien reflétée dans les
prédicitons basée sur les SDMs. C'est bien la fusion méthodiques des
deux informations qui doit permettre d'aller vers des approches
systématiques \citep{Meier2010}. Dans les cas précis d'un prédateur et
ses proies\footnote{cela est aussi valable pour un pollinisateur et les
  plantes qu'ils pollinisent ou encore pour un parasite et ses hotes.},
il y a un lien évident entre les distributions: le prédateur est
nécessairement limité par la distribution conjointe de ces proies
\citep{Holt2009, Shenbrot2007}. Ainsi, la reconnaissance de cette
réalité doit être au coeur d'un renouvellemnent des apporches pour
prédire des espèce en réseaux.

Du chapitre \ref{chap2} au chapitre \ref{chap3}, j'ai souligné l'intérêt
des développemnets théoriques pour mieux comprendre des données
empiriques. En partant initialement de la question \emph{est-ce que les
espèces qui interagissent co-occurent différemment que celle qui
n'intéragissent pas}, j'ai compris qu'il n'y avait pas de réponse
tranchée, mais plutôt une réponse qui dépendait de la nature du réseau.
Ce résultat sera, je pense, très utile pour amener une lumière nouvelle
sur le débat qui anime la communauté des biogéographes autour de la
question du rôle des interactions dans la distribution aux larges
échelles spatiale. En méditant sur ce chapitre, j'ai également bien
compris comment le choix d'un espace explicatif donné pouvait amener à
des conclusion qui demandait une alternative. Bien que dans les
dernières années avec l'essort des JSDMs il y a une attention
particulière, il faut apporter davantage de biologie pour bien comprndre
les données que nous traitons et notamment rapidement lever l'hypothèse
d'indépendance \citep{Elith2006}. Je suis convaincu que le problème des
intéractions n'estpas seulement question d'un problème d'échelle
spatiales\citep{Araujo2014, Belmaker2015}, mais c'est aussi une question
qui concerne la nature du système étudié. Mes résultats sont seulement
un premier indice fort en ce sens et soulignent l'intérêt d'étudier le
système pour conclure la nature des facteurs qui sont à prendre en
compte. Pour aller plus loin dans ma réflexion, il faudrait, je pense,
que nous parvenions à une caracctérisation des systèmes pour lesquels
les interactions sont ou ne sont pas important afin que l'on puisse
avoir des règles efficaces pour savoir quelles types d'approches est
pertienent pour quel type de système. C'est une étape importante et
longue pour aller vers des prédictions robustes qui sont très
aujourd'hui plus que nécessaires. En particulier je pense que
L'intégration systématiques des co-occurrence à travers les JSDMs tels
qu'ils sont présentés aujoud'hui ne permetrra pas toujours de comprendre
ce qu'il se passe \citep{Ovaskainen2010, Pollock2014, Warton2015b}.

\section*{Vers une écologie
prédictive?}\label{vers-une-uxe9cologie-pruxe9dictive}
\addcontentsline{toc}{section}{Vers une écologie prédictive?}

\subsection*{Les défis à relever dans un monde en
changement}\label{les-duxe9fis-uxe0-relever-dans-un-monde-en-changement}
\addcontentsline{toc}{subsection}{Les défis à relever dans un monde en
changement}

Érosion de la biodiversité, extinctions de masses, perte de service
écosystémiques, la liste est longue des boulversements que nous pouvons
attribuer aux activités anthropiques. Nous sommes cinquième Rapport
d'évaluation, dans une période clef ou l'on tente de réagir avec la
COP21 qui s'est tenue è Paris en décembre 2015, la mise en place de
l'IPBES
(\emph{Intergovernmental Platform on Biodiversity and Ecosystem Services},
qui se veut un équivalent du IPCC) \citep{Diaz2015a}. La mise en place
est certe pour articluer la et faciliter la trasmission des donn.es
suitifique avec leur insertitude au décideurs politiques. Le porblème
poir la climatilogiques est incertitudes et aussi incertitude de
politique en plus donc plusierurs scénarios d'émission mais si l'on
regrader avec nous on rajoute une incertitude lié à lnature de l'objet
et l'état de notre connaissance, ce qu'on peut dire sur les communautés
de demian semble bien maigre.

Nous assistons à une recomposition des communautés. Lorque l'on parle de
sixième extinction c'est que nous avons des taux record d'extinction
\citep{Thomas2004}. Biensur cela pose des grandes questions sur comme
savoir quel partie du vivant est davatage touché \citep{Thuiller2011},
mais d'un point de vue on est dans une ériode de focntionnenement
particulier qu'on peut voir comme ue grande expérience mais aussi comme
m moment où des théories solides seraient le sbienvenue. Nous sommes
dans un moment de crise et nous manquons de théorie et de conpréhsin des
écosytèmes qui sont présentemment en changement. Dans son arctile `Don't
juge a species on their origin' Mark Davis prend à revers un sertain
nombre d'idée recu et souligne que les effects des invedeurs peuvent
être positives \citet{Davis2011}. Devons accepté le frelon asiatique et
l'homogéisation des milieux? Est-ce Laisser et accompagner les

Et si on faisait rien pour le frelon asiatique ? Peut être que qu'une
partie de l'entomofaune distparaitrait, peut être que les abeilles
domestiques deviendraient pus efficace et finirais par stabiiser sa
populatiom. Dans tous les cas, au moins au point de vue
\citep{Villemant2011}. Récemment une suprenante étude sur le Diamant
mandarin (\emph{Taeniopygia guttata}), un oiseau commun du centre de
l'Autralie, qu'au dessus dessu de 26°C un champ particulier du mâle pour
allerqui induit une différence à des oiseau plus petit et à une meilel
ferticilité \citep{Mariette2016}. Comolexit. des systèmes biologoqes à
prendre et cMest surprise sont fialemnt plutôt la règle et l'excetion et
donc une modestie dans la tâche de modélisé la biodiversité mondiale
\citep{Mouquet2015}.

\subsection*{Des règles en écologie et
évolution?}\label{des-ruxe8gles-en-uxe9cologie-et-uxe9volution}
\addcontentsline{toc}{subsection}{Des règles en écologie et évolution?}

Quelles hypothèse pouvons nous faire sur les produits de évolution? Si
on peut supposer qu'il y a des compétition ou la règle est le changemnt
cette même propriété a-t-elle des propriétés sur le long terme. Peut-on
affirmer que les produits de l'évoluton dans un enviroemnt stable amène
à des entités qui optimise l'tilisation de l'énergie. Si oui, que dire
des produites de l'évoltion dans avec variation. Si on peut faire des
hypothèses comment les tester. Dans l'article de \citep{Rabosky2015} les
allémtries offre-t-elle un espoir peut-on facilement résuidre la
diversité des réseaux \citep{Eklof2013} grace de l 'allométrie aussi
\citep{Schneider2012}?.

Il est plus facile de s'ppuyer que sur des correlatons d'autant plus que
si des correlatons fortes il existent une esplication peut alors voir le
jour. avoir des erreurs quantifiables mieux dessiner ce qui est suremnet
plus déterminsite de ceux qui l'ai moins Ca ne fat pas une théorie on
peut esp.reer que c'est une bonne apporximation.

Si l'évolution est imprévisible si au dela d'un certain temps on ne peut
presque rien dire\ldots{} Si la chance de des abeilles européennes
changeait comment prédire cela changement de comportement mais que nous
sommes dand l'incapacité de le prédire que pensé du status de l'écologie
et de l'évoluton en tant que science. Si la composant historique domine
le royaume de la biologie devons-nous nous dsatisfaire de le décire.
L'espoir mais la publication de Ian Hatton êut faire douter de l'absence
de l'absecmed de règel. Comment croire qu'il n'y a pas des principes
d'ordre enerétique la-dessous. Convergence\ldots{}

2014, Hurlbert et Stegen propose une série d'hypothèse pour mettre en
évidence l'impact de l'énergy sur l'évolution la troisième hypothèse est
temps suffisant pour équilibre. Une telle hypothèse une forme de
maximistaion de la production de la biomasse et l'utilisation qui est
peut être. Peut-être que les différents mécanismes en jeu dans les
processu évolutifs amène probablement à une forme de
stationarité\ldots{}

Biensur il y un certain nombre de chose comment ne pas oender que le
lègue de la TIB n'est aps quelque chose mais et l'emsmeble des théories
est souvent resreint à un chmape à une catégorei et comme moi j'ai
montrés que des système oour lesquels les interaction sosnt plus ou
moins importnates, je pense qu,'il y a un un promier travails ed
evatégoraisation.

De manière tout a fait probante, l'étude de la nature a été un travaille
de groupement por essayer de classé les êtres vivants par des criètère
plus ou moins cohérents. La classification que nous connaissons
maintentantse base sur les lien de parenté entre les êtres vivants. En
plus de cette catégorisation lobale, nous regroupons les animaux de
manière fonctionnelle en écologique et nous parlons aisin de poducteur
primaires, de proie, de prédateur, de généraliste, de
spécialiste\ldots{} Cette terminalogie soulève bien des différences
majeurs mais de manière paradoxale les SDMs dont j'ai souvent parlé dans
mon travil de thèse semble être valables pour toutes les espèces. Biens
entendu dans les faits les chercheurs connaissent le plus souvent les
différences des grands groupes et les approches les plus appropriés pour
tel ou tel groupe. Néanmoins quand on ne reconnait pas dans une forme de
systématisation ces différences. Ainsi, si par exemple, la plupart des
SDMs sont efficaces pour traiter des arbres mais plutôt problématiques
pour traiter des oiseaux, il me semble qu'il faut expliquer pouquoi et
ne pas essayer d'affirmer que les interactions sont importantes ou pas
basé sur un ensemble aprticulier d'exemple bien choisi. En disant cela
je pense qu'il serait souhaitable d'avoir des arguments théorique solide
pour dire quel ou quel type d'espèce il faut prendre en compte tel ou
tel facteur pour bien conpendre. Cette idée peut être batie sur les
traits finctionnels. En 2006, McGill proposait de rebâtir l'écologie des
communautés des traits fonctionelle, ces traits qui mesurent différentes
propriétés des espèces \citep{McGill2006}. Aisin au lieu de se référer à
une catéégorisation de l'esèce par son no taxonomie un ensemble plus
objectof sur la bases desquelles des rgles sont à trouver notamment sur
les stratégies de modélisation des ranges. Et mieux en composition su
des prédiction sur les set de triats sont possibles.

De même peut être que des hypothèse eambietieurse, dans des que le tems
à cerie à aller vers des systèmes énergétique Aller vers des contriantes
énergétiques mais il est dur qu'on trouvera des règles fiables sur un
système qui bien que régit par des règles physique assez nien comprise
est un moteur de stochasticé..

\subsection*{Vers une théorie en intégrative de la
biogéographie}\label{vers-une-thuxe9orie-en-intuxe9grative-de-la-bioguxe9ographie}
\addcontentsline{toc}{subsection}{Vers une théorie en intégrative de la
biogéographie}

J'ai montré la cohérence de la reflexion théorique pour mieux comprendre
et aller dcherché dan sles données un espoir d'aller plus loin qui
biensureest de meiller données pour aller plus loun.

L'effort théorique en biogéographie doit se faire autour être
intégration ordonnée de concepts clés issus de différents champs de
l'écologie \citep{Thuiller2013} est une clef essentielle pour aller vers
des prédctions de Ainsi, alors que les conditions climatiques et plus
généralement la géographie physique sont classiquement évoquées pour
expliquer la répartition des espèces \cite{Kearney2004}, les
interactions entre espèces sont quant à elles souvent occultées. De
même, bien que les processus évolutifs soient souvent évoqués comme
déterminants majeurs de la diversité des espèces \cite{Rosindell2011},
leurs effets à court terme sont souvent ignorés \cite{Parmesan2006} dans
les scénarios décrivant la biodiversité de demain \cite{Lavergne2010}.
La difficulté principale est alors de produire des modèles (théoriques
en première instance) qui intègrent l'ensemble des processus et les
relations qu'ils entretient \cite{Thuiller2013} tout en gardant une
relative simplicité. Une théorie intégrative en biogéographie pourrait
être le meilleur point d'ancrage pour construire de nouvelles approches
appliquées. Avec une telle théorie en main, nous pourrions aller vers
l'enjeux majeurs de ces dernières années en biogéographie : relâcher les
hypothèses que les modèles classiques de répartitions des espèces
d'aujourd'hui utilisent (notamment en occultant les interactions) pour
prédire la biodiversité de demain \cite{Guisan2011}.

Il est vraissembleble que nous ne pourrions pas nous extraire
totalemnent et qu'il n'est opas question de tout expliquer
\citep{Gravel2011a}

En s'appuantddur un champ bien dessiner il afiat être précis en
biogéogrpaheo Comme un prmier pas plus loin que mes travax le chapitre
\ref{chap4} vien apporter un pas vers le développent dd'un théorie
métabllolqieude la vers laquelle e veux aporter ma contribution dans les
prochaines année.s

\ref{chap4} vers une théori métablocique métaboloique toujours pour
lever les interacroj biensur comment émerge mais regarder la nature
mêmeé
